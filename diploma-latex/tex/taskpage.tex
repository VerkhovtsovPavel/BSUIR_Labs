\newpage
{
    % Вставка текста
    \newcommand{\lineunderscore}[1][]{%
        \uline{#1}\uline{\hspace*{\fill}}%
    }

    \newcommand{\lineunderscorec}[1][]{%
        \uline{\hspace*{\fill}}%
        \lineunderscore[#1]%
    }

    % Установка основного размера шрифта для страницы: 12pt
    \makeatletter\let\newcommand\renewcommand\input{size12.clo}

    \newgeometry{top=1.25cm,bottom=1.25cm,right=1cm,left=2cm,twoside}
    \thispagestyle{empty}
    \setlength{\parindent}{0em}


    \begin{center}
        Министерство образования Республики Беларусь\\
        Учреждение образования\\
        БЕЛОРУССКИЙ ГОСУДАРСТВЕННЫЙ УНИВЕРСИТЕТ \\
        ИНФОРМАТИКИ И РАДИОЭЛЕКТРОНИКИ\\[1em]

    \begin{minipage}{\textwidth}
        \begin{flushleft}
            \begin{tabular}{ p{0.20\textwidth}p{0.31\textwidth}p{0.20\textwidth}p{0.20\textwidth} @{} }
                Факультет & \lineunderscore[КСиС] & Кафедра & \lineunderscore[ПОИТ] \\
                Специальность & \lineunderscore[1-40 01 01] & Специализация & \lineunderscore[03]
            \end{tabular}
        \end{flushleft}
    \end{minipage}\\[1em]

    \begin{minipage}{\textwidth}
        \begin{flushright}
            \begin{tabular}{p{0.40\textwidth}}
                УТВЕРЖДАЮ \\[0.5em]
                \underline{\hspace*{7em}} Н.В.~Лапицкая \\
                <<\underline{\hspace*{4ex}}>> \underline{\hspace*{7em}} \the\year~г.
            \end{tabular}
        \end{flushright}
    \end{minipage}\\[1em]

    \textbf{ЗАДАНИЕ} \\
    \textbf{по дипломному проекту (работе) студента} \\
    \vspace{1em}
    \textbf{\lineunderscorec[Верховцова Павла Андреевича]} \\
    {\small (фамилия, имя, отчество) }

    \end{center}

    1. Тема проекта (работы):
    \lineunderscore[Программное средство определения психологических характеристик по анализу образцов почерка]\\
    \lineunderscore\\
    утверждена приказом по университету от <<\uline{~~06~~}>> \uline{~~февраля~~} \the\year~г. \No{} \uline{~~207-c~~}

    \vspace{1em}

    2. Срок сдачи студентом законченного проекта (работы): \lineunderscorec[1 июня \the\year~года]

    \vspace{1em}

    3. Исходные данные к проекту (работе):
    \lineunderscore[Тип операционной системы~-- Linux; Языки программирования~-- Java, Scala, Перечень выполняемых функций: а) сегментация рукописного текста; б) выделения характеристик почерка; в) определение психологических характеристик по характеристикам почерка; г) контроль доступа к данным. Назначение разработки: автоматизация процесса определения психологических параметров личности по анализу образцов почерка]

    \vspace{1em}

    4. Содержание пояснительной записки (перечень подлежащих разработке вопросов): \\
    \lineunderscore[Введение]\\
    \lineunderscore[1 Обзор предметной области]\\
    \lineunderscore[2 Анализ требований к программному средству]\\
    \lineunderscore[3 Проектирование программного средства]\\
    \lineunderscore[4 Разработка программного средства]\\
    \lineunderscore[5 Тестирование программного средства]\\
    \lineunderscore[6 Методика использования программного средства]\\
    \lineunderscore[7 Технико-экономическое обоснование разработки ПС]\\
    \lineunderscore[Заключение]\\
    \lineunderscore[Список использованных источников]\\
    \lineunderscore[Приложение А Исходный код программного средства]

    \clearpage
    \thispagestyle{empty}

    5. Перечень графического материала (с точным указанием обязательных чертежей): \\
    \lineunderscore[Результаты работы программного средства. Плакат - формат А1, лист 1.]\\
    \lineunderscore[Диаграмма классов программного средства. Плакат - формат А1, лист 1.]\\
    \lineunderscore[Диаграмма последовательности программного средства. Плакат - формат А1, лист 1.]\\
    \lineunderscore[Модуль выделения признаков почерка. Схема программы - формат А1, лист 1.]\\
    \lineunderscore[Модуль контроля доступа. Схема программы - формат А1, лист 1.]\\
    \lineunderscore[Сегментация рукописного текста. Схема алгоритма - формат А1, лист 1.]\\
    \lineunderscore

    \vspace{1em}

    6. Содержание задания по технико-экономическому обоснованию:\\
    \lineunderscore[Расчет экономической эффективности от разработки программного средства определения психологических характеристик по анализу образцов почерка]\\
    \lineunderscore

    Задание выдала: \uline{\hspace*{6em}} / Т.\,А.~Рыковская /

    \vspace{1em}

    %\vfill

    \begin{center}
      \textbf{КАЛЕНДАРНЫЙ ПЛАН}
    \end{center}

    \begin{tabular}{
        | m{0.51\textwidth}
        | >{\centering}m{0.09\textwidth}
        | >{\centering}m{0.20\textwidth}
        | >{\centering\arraybackslash\hspace{0pt}}m{0.10\textwidth}|
    }
        \hline \centering{Наименование этапов дипломного проекта (работы)} & Объем этапа, \% & Срок выполнения этапов & Примечание \\
        \hline Анализ предметной области, & & & \\
        \hline разработка технического задания & 15-20\% & 01.02--14.02 & \\
        \hline Разработка функциональных требований, & & & \\
        \hline проектирование архитектуры программы & 15-20\% & 15.02--06.03 & \\
        \hline Разработка схемы программы, алгоритмов, & & & \\
        \hline схемы данных & 15-20\% & 07.03--27.03 & \\
        \hline Разработка программного средства & 15-20\% & 28.03--24.04 & \\
        \hline Тестирование и отладка & 10\% & 25.04--08.05 & \\
        \hline Оформление пояснительной записки & & & \\
        \hline и графического материала & 20\% & 09.05-31.05 & \\
        \hline
    \end{tabular}

    \vspace{2em}

    Дата выдачи задания \lineunderscorec[~~1 февраля 2016~~] \hspace{2ex} Руководитель \hfill{} \uline{\hspace*{4em}} / А.\,В.~Хмелева /

    \vspace{1em}

    Задание принял к исполнению  \uline{\hspace*{4em}} / П.\,А.~Верховцов /

    \restoregeometry
}
\clearpage
