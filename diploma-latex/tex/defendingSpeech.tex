\section{Речь}
\subsection{Введение}
В современном мире наблюдается тенденция к глобальной автоматизации производства, управления и учета. Системы учета кадров, клиентов, товаров и бухгалтерии, и множество других применений находит автоматизация на сегодняшний момент.
На фоне этого вопросы доверия к управленческому и производственному персоналу с каждым днем все острее, так как все чаще возникают ситуации когда человек является самым медленным элементов производственной цепочки.

Графология и почерковедение – это дисциплины, утверждающии о наличии связи между почерком и индивидуальными особенностями личности. Оперирую такими признаками как наклон, интервал между строками, форма полей эксперты могут установить психологическую устойчивость, склонность к агрессии, коммуникабельность и другие качества интересные работодателю.

\subsection{Архитектура}
Программное средство реализовано с использованием архитектуры микросервисов, что позволяет обновлять и расширять компоненты системы без необходимости ее полной остановки или перезагрузки. Так же данный подход позволяет добиться горизонтальной масштабируемости путем увеличения количества модулей одного типа и позволяет географически разнести компоненты системы. 

Разработанное программное средство состоит из следующих модулей (\emph{Диаграмма классов}):
\begin{itemize}
    \item модуль выделения признаков почерка;
    \item модуль определения параметров личности;
    \item модуль контроля доступа;
    \item модуль управление образцами почерка;
    \item модуль доступа к базе данных.
\end{itemize}

\subsection{Сегментация}
Перед началом сегментации изображения необходимо выполнить его предварительную обработку (бинаризация, удаление шумов);
Алгоритм сегментации реализованный в проекте является параллельным(\emph{Алгоритм сегментации}), что позволило значительно уменьшите время сегментации. 

Модель акторов - модель параллельных вычислений, основанная на взаимодействии изолированных примитивов, взаимодействующих по средствам получения и отправки сообщений. 

Благодаря отсутствую прямое взаимодействия акторов между собой, т.к. вся коммуникация осуществляется при помощи передачи сообщений, становится возможным полностью избежать блокировок потоков исполнения. В совокупности с использованием неизменяемых структур данных, механизм сообщений делает всю работу системы акторов априорно асинхронной и неблокирующей. Позволяет добиться прироста производительности сопоставимого с количеством ядер процессоров в среде исполнения, чего невозможно добиться при использовании классической модели параллельности на основе потоков и блокировках при доступе к общему изменяемому состоянию.

\subsection{Пользователи}

В разработанном программном средстве предусмотрено две роли "Пользователь" и "Администратор" (\emph{Use Case}).
"Администратор" может пакетно загружать образцы изображений и их обработка начинается автоматически, данный функционал в основном используется для загрузки обучающей выборки.

\subsection{Выделение признаков}

Следующим этапом работы программ является выделения признаков почерка.

Программное средство оперирует следующими признаками:
\begin{itemize}
  \item наклон символов;
  \item наклон строк;
  \item интервал между символами;
  \item интервал между словами;
  \item интервал между строками;
  \item частота текста;
  \item сила нажима.
\end{itemize}

Что позволяет получить разностороннюю информацию для анализа.
Так же стоит отметить что для ускорения обработки образца, начало выделение признака начинается как-только готовы все исходные данные(\emph{Модуль выделения признаков почерка}), например определение интервала между строками начинается сразу после сегментации изображения на строки.

Результаты представлены на \emph{Результаты работы - 1}

\subsection{Классификация}
Следующий этапом работы является определение характеристик личности.

Всего определяется 3 характеристики
\begin{itemize}
  \item темперамент;
  \item лидерские качества;
  \item работоспособность.
\end{itemize}

Всего 16 классов образцов.

Для классификации использоваться классификатор на основе метода опорных векторов.

Обучающая выборка представляет собой примерно 1500 образцов почерка от 500 авторов.
Распределения образцов выборки по классам представлено на (\emph{Результаты работы - 2})

Все классы представлены в выборке примерно в разных пропорциях.

\subsection{Тестирование}
Для проверки корректности работы ПС было разработано 23 тест-кейса покрывающие все функционал. В ходе тестирования не было выявило дефектов. 

\subsection{Вывод}
Разработанное программное средство может быть использовано экспертами-графологами для предварительного анализа и подготовки данных, а так же отделами кадров при отборе кандидатов.