\newcommand{\tableHead}{\hline Название тест-кейса & Шаги & Ожидаемый результат & Результат \\ \hline}

\section{Тестирование программного средства}
\label{sec:testing}

Разработанное программное средство представляет собой набор программных модулей и сервер базы данных. Тестирование программного средства представляет собой проверку работоспособности самих разработанных модулей независимо от сервера базы данных. 
Перед подготовки программного средства к тестированию необходимо выполнить следующие действия:
\begin{enumerate}
  \item развернуть все модули программного средства;
  \item сконфигурировать доступ к базе данных;
  \item сконфигурировать адреса сервисов выделения признаков и определения психологических характеристик в модуле распределения нагрузки.
\end{enumerate}

Для оценки правильности работы программного средства было проведено тестирование. В данном разделе приведены тест-кейсы и их результаты в вид таблиц.

\begin{longtable}{| >{\raggedright}p{0.25\textwidth}
                  | >{\raggedright}p{0.25\textwidth}
                  | >{\raggedright}p{0.25\textwidth}
                  | >{\raggedright\arraybackslash}p{0.15\textwidth}|}
  \caption{Тестирование операций с образцами почерка}
  \label{table:testing:db}\\
  \endfirsthead
  \caption*{Продолжение таблицы \ref{table:testing:db}}\\
  \tableHead
  \endhead

  \tableHead
    Получение списка образцов почерка пользователя (авторизированный) &
    1) ввести верный логин; \\
    2) ввести верный пароль; \\
    3) нажать кнопку "Войти".
    &
    Откроется главная страница приложения со списком ранее загруженных образцов почерка.
    &
    Тест пройден \\ \hline

   Получение списка образцов почерка пользователя (неавторизированный) &
    1) ввести верный логин; \\
    2) ввести верный пароль; \\
    3) нажать кнопку "Войти" и сохранить id пользователя; \\
    4) выполнить в командной строке команду \emph{curl URL"=приложения/users/id"=пользователя/samples/}.
    &
    HTTP ответ содержит код 401 Unauthorized, список образцов не отправлен.
    &
    Тест пройден \\ \hline

    Получение списка образцов почерка другого пользователя &
    1) ввести верный логин; \\
    2) ввести верный пароль; \\
    3) нажать кнопку "Войти" и сохранить id пользователя; \\
    4) авторизоваться в приложении другим пользователем;
    5) ввести в стоку браузера \emph{URL"=приложения/users/id"=первого"=пользователя/samples/} и нажать Ввод.
    &
    HTTP ответ содержит код 403 Forbidden, список образцов не отправлен. 
    &
    Тест пройден \\ \hline

   Добавление образца почерка &
   1) авторизоваться в приложении; \\
   2) выбрать изображение допустимого формата и размера для добавления; \\
   3) нажать кнопку "Загрузить" и дождаться окончания загрузки.
   &
   Откроется страница просмотра образца почерка с возможностью выделения признаков и определения характеристик личности.
   &
   Тест пройден \\ \hline

   Добавление образца почерка (неверный формат) &
   1) авторизоваться в приложении; \\
   2) выбрать изображение недопустимого формата, например txt.
   &
   Появится сообщений "Файл недопустимого формата, используйте jpg(jpeg), bmp или png".
   &
   Тест пройден \\ \hline

   Добавление образца почерка (нулевой размер) &
   1) авторизоваться в приложении; \\
   2) выбрать изображение допустимого формата с нулевым размером. 
   &
   Появится сообщений "Файл недопустимого размера".
   &
   Тест пройден \\ \hline

   Добавление образца почерка (размер более 10 MB) &
   1) авторизоваться в приложении; \\
   2) выбрать изображение допустимого формата с размером более 10 MB.
   &
   Появится сообщений "Файл недопустимого размера".
   &
   Тест пройден \\ \hline

   Удаление образца почерка (подтвердить удаление) &
   1) авторизоваться в приложении; \\
   2) выбрать образец из ранее загруженных; \\
   3) нажать кнопку "Удалить образец"; \\
   4) нажать кнопку "Подтвердить".
   &
   Откроется главная страница приложения со списком ранее загруженных образцов почерка, список не будет содержать удаленный образец.
   &
   Тест пройден \\ \hline

   Удаление образца почерка (отменить удаление) &
   1) авторизоваться в приложении; \\
   2) выбрать образец из ранее загруженных; \\
   3) нажать кнопку "Удалить образец"; \\
   4) нажать кнопку "Отмена".
   &
   Откроется страница просмотра образца почерка.
   &
   Тест пройден \\ \hline

   Пакетное добавление образцов почерка &
   1) авторизоваться в приложении как администратор; \\
   2) выбрать изображения допустимого формата и размера для добавления; \\
   3) нажать кнопку "Загрузить" и дождаться окончания загрузки; \\
   4) выбрать любой образец из только что загруженных и дождаться окончания обработки.
   &
   Откроется страница просмотра образца почерка со значениями выделенных признаков и характеристиками личности.
   &
   Тест пройден \\ \hline

\end{longtable}

\begin{longtable}{| >{\raggedright}p{0.25\textwidth}
                  | >{\raggedright}p{0.25\textwidth}
                  | >{\raggedright}p{0.25\textwidth}
                  | >{\raggedright\arraybackslash}p{0.15\textwidth}|}
  \caption{Тестирование модуля контроля доступа}
  \label{table:testing:accesses}\\
  \endfirsthead
  \caption*{Продолжение таблицы \ref{table:testing:accesses}}\\
  \tableHead
  \endhead

  \tableHead
   Авторизация пользователя с неверным логином и паролем &
   1) ввести неверный логин; \\
   2) ввести неверный пароль; \\
   3) нажать кнопку "Войти".
   &
   Появится сообщение "Введен неверный логин или пароль".
   &
   Тест пройден \\ \hline

   Авторизация пользователя с верным логином и неверным паролем паролем &
   1) ввести верный логин; \\
   2) ввести неверный пароль; \\
   3) нажать кнопку "Войти".
   &
   Появится сообщение "Введен неверный логин или пароль".
   &
   Тест пройден \\ \hline

   Авторизация пользователя с верным логином и паролем паролем &
   1) ввести верный логин; \\
   2) ввести верный пароль; \\
   3) нажать кнопку "Войти".
   &
   Откроется главная страница приложения со списком ранее загруженных образцов почерка и элементами для загрузки новых.
   &
   Тест пройден \\ \hline

   Регистрация пользователя &
   1) нажать кнопку "Регистрация"; \\
   2) ввести уникальный логин; \\
   3) ввести корректный пароль; \\
   4) нажать кнопку "Зарегистрироваться".
   &
   Появится сообщений "Регитсрация прошла успешно". Откроется страница авторизации.
   &
   Тест пройден \\ \hline

   Регистрация пользователя с неуникальным логинов &
   1) нажать кнопку "Регистрация"; \\
   2) ввести логин ранее зарегистрированного пользователя; \\
   3) ввести корректный пароль; \\
   4) нажать кнопку "Зарегистрироваться".
   &
   Появится сообщение "Пользователь с таких логином уже зарегистрирован".
   &
   Тест пройден \\ \hline

   Регистрация пользователя с некорректным паролем (без цифр) &
   1) нажать кнопку "Регистрация"; \\
   2) ввести уникальный логин; \\
   3) ввести пароль не содержащий цифр; \\
   4) нажать кнопку "Зарегистрироваться".
   &
   Появится сообщение "Пароль должен содержать не менее 8 символов в верхнем и нижнем регистрах и цифры".
   &
   Тест пройден \\ \hline

   Регистрация пользователя с некорректным паролем (нижний регистр) &
   1) нажать кнопку "Регистрация"; \\
   2) ввести уникальный логин; \\
   3) ввести пароль содержащий символы только нижнего регистра; \\
   4) нажать кнопку "Зарегистрироваться".
   &
   Появится сообщение "Пароль должен содержать не менее 8 символов в верхнем и нижнем регистрах и цифры".
   &
   Тест пройден \\ \hline

   Регистрация пользователя с некорректным паролем (короткий) &
   1) нажать кнопку "Регистрация"; \\
   2) ввести уникальный логин; \\
   3) ввести пароль длинной 6 символов; \\
   4) нажать кнопку "Зарегистрироваться".
   &
   Появится сообщение "Пароль должен содержать не менее 8 символов в верхнем и нижнем регистрах и цифры".
   &
   Тест пройден \\ \hline

\end{longtable}

\clearpage

\begin{longtable}{| >{\raggedright}p{0.25\textwidth}
                  | >{\raggedright}p{0.25\textwidth}
                  | >{\raggedright}p{0.25\textwidth}
                  | >{\raggedright\arraybackslash}p{0.15\textwidth}|}
  \caption{Тестирование модулей выделения признаков почерка определения параметров личности}
  \label{table:testing:features_personal}\\
  \endfirsthead
  \caption*{Продолжение таблицы \ref{table:testing:features_personal}}\\
  \tableHead
  \endhead

  \tableHead

   Выделение признаков образца почерка &
   1) авторизоваться в приложении; \\
   2) выбрать образец из ранее загруженных; \\
   3) нажать кнопку "Выделить признаки" и дождаться окончания обработки.
   &
   Откроется страница просмотра образца почерка со значениями выделенных признаков.
   &
   Тест пройден \\ \hline

   Определения параметров личности (признаки почерка выделены) &
   1) авторизоваться в приложении; \\
   2) выбрать образец из ранее загруженных с выделенными признаками; \\
   3) нажать кнопку "Определить параметры личности" и дождаться окончания обработки.
   &
   Откроется страница просмотра образца почерка со значениями выделенных признаков и характеристиками личности.
   &
   Тест пройден \\ \hline

   Определения параметров личности (признаки почерка не выделены) &
   1) авторизоваться в приложении; \\
   2) выбрать образец из ранее загруженных без выделенных признаков почерка; \\
   3) нажать кнопку "Определить параметры личности".
   &
   Появится сообщение "Предварительно будет выполнено выделение признаков" по окончанию обработки откроется страница просмотра образца почерка со значениями выделенных признаков и характеристиками личности.
   &
   Тест пройден \\ \hline
\end{longtable}

\begin{longtable}{| >{\raggedright}p{0.25\textwidth}
                  | >{\raggedright}p{0.25\textwidth}
                  | >{\raggedright}p{0.25\textwidth}
                  | >{\raggedright\arraybackslash}p{0.15\textwidth}|}
  \caption{Тестирование дополнительных функций}
  \label{table:testing:other}\\
  \endfirsthead
  \caption*{Продолжение таблицы \ref{table:testing:other}}\\
  \tableHead
  \endhead

  \tableHead
   Переключение языков приложения (русский -> английский) &
   1) авторизоваться в приложении пользователей с установленным русским языком; \\
   2) переключите язык на английский.
   &
   Страница перезагрузится, язык интерфейса будет изменен на английский.
   &
   Тест пройден \\ \hline

   Переключение языков приложения (английский -> русский) &
   1) авторизоваться в приложении пользователей с установленным английским языком; \\
   2) переключите язык на русский.
   &
   Страница перезагрузится, язык интерфейса будет изменен на русский.
   &
   Тест пройден \\ \hline

   \hline
\end{longtable}

\clearpage
