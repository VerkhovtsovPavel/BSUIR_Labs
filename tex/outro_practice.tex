\sectioncentered*{Вывод}
\addcontentsline{toc}{section}{Вывод}
\label{sec:practice_outro}

В ходе практике были проанализированны существующие аналоги разрабатываемого программого средсва и изучены литературные источники, на основании анализа были составлены и специфицированны тпебования к разбабатываемому ПС. В качестве метода семгентации текста используется статический метод так он дает достойный результат и обладает хорошими временными характеристиками, а в качестве метода класификации используется метод опорных векторов из-за быстрого времени обучения и распознования и наличия достаточной обучающей вырорки для его использования. На основании требований к программому средству в качесве основного языка программирования был выбран Scala, как современный, мультипарадигмальный, компилируемый язык что обеспечит высокую производительность вместе с выразительностью и компактностью кода. Так же будут использоваться фреймворк Akka для реализации паралельной обработки данных и фреймворк Spray для реализации веб-сервера, данные фреймворки были выбраны из-за использованые неблокирующих операций и использования модели акторов что обспечивает общий подход к разработке и упрощает сопровождение. На основании спецификации требований и используемых технологии была разработана общая архитектура будущего программного средства состоящая из моделя выделения признаков почерка, модуля определения параметров личности, модуля контроля доступа и модуля доступа к базе данных.