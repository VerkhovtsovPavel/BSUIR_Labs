\section{Анализ полученных результатов}
\label{sec:summary}

В данном диссертационной работы были проанализированы существующие методы и подходы в областях автоматизации графологического анализа, диагностики неврологических отклонений и биометрической аутентификации на основе анализа образцов почерка. А так же современные системы проектирования, разработки и сопровождения программных средств и систем. Современные прикладные библиотеки и фреймворки хорошо зарекомендовавшие себя в сфере разработки программного обеспечения и только набирающие популярность.

В ходе работы был установлен исчерпывающий набор признаков почерка позволяющий производить независимый анализ в каждой из трех областей. Не мало важно так жен было избежать дублирования и исключить из списка параметры в высокой степенью зависимости.
Результирующий список признаков имеет следующий вид:
\begin{itemize}
  \item длительность написания;  
  \item количество линий;
  \item длинна по горизонтали;
  \item длинна по вертикали;
  \item площадь;
  \item общая длинна;
  \item максимальное ускорение;
  \item минимальное ускорение;
  \item длительность написание по вертикали;
  \item длительность написание по горизонтали;
  \item наклон символов;
  \item наклон строк;
  \item интервал между символами;
  \item интервал между словами;
  \item интервал между строками;
  \item частота текста.
\end{itemize}

Важно отметить некоторые части разработанного программного средства, в частности модуль определения неврологических отклонений, требует вычисления дополнительных параметров таких как среднеквадратичное отклонение от скелета образца, тем не менее, так как параметр является частным и требуются достаточно существенные вычислительные и временные затраты на его нахождение в результирующий набор он не входит.

В ходе работы был апробированн ряд решений в области классификации признаков, таких как количество классов, типы параметров, в частности решался вопрос использования признака максимальное ускорение как непрерывной величины либо разбиение его на категории в процессе классификации признаков почерка. Были опробованы различные решения в области построения скелета образца  

В ходе работы были проанализированные особенности распределения каждого признака почерка, как при их написании одним человеком, так и разными людьми, что легло в основу коэффициентов слагаемых при биометрической аутентификации, а так же выборе типов входов, непрерывное значение, категория при определении характеристик личности. 

Стоит отметить достаточно большой процент ошибок второго рода при использовании метода биометрической аутентификации, это может быть связано с неполной проработкой модели коэффициентов или излишне высоким порогом аутентификации. Однако уменьшение вышеописанного порога неизбежно приведет в увеличению ошибок первого рода, что недопустимо учитывая специфику области информационной безопасности.

\newcommand{\tableHead}{\hline Тип признака & Название признака & Отклонение от обучающей выборки \\ \hline}

Разброс рассчитанных признаков почерка, представлен в таблице~\ref{table:summary:feauture_error}.
\begin{longtable}[l]{| >{\raggedright}m{0.27\textwidth}
                     | >{\raggedright}m{0.32\textwidth}
                     | >{\centering\arraybackslash}m{0.33\textwidth}|}
  \caption{Исходные данные}
  \label{table:summary:feauture_error} \\
  \endfirsthead
  \caption*{Продолжение таблицы \ref{table:summary:feauture_error}}\\
   \tableHead
  \endhead
    \tableHead

    Наклон & & 10\% \\ \hline
    & наклон символов & 11\% \\ \hline
    & наклон строк & 9\% \\ \hline
    Интервал & & 7\% \\ \hline
    & интервал между символами & 10\% \\ \hline
    & интервал между словами & 7\% \\ \hline
    & интервал между строками & 4\% \\ \hline
    Длинна & & 3\% \\ \hline
    & длинна по горизонтали & 3\% \\ \hline
    & длинна по вертикали & 3\% \\ \hline
    & общая длинна & 3\% \\ \hline
    Прочие & & 5\% \\ \hline
    & количество линий & 4\% \\ \hline
    & частота текста & 3\% \\ \hline
    & площадь & 6\% \\ \hline
\end{longtable}
Как видно ошибка вычисление признаков не превышает 11\% что позволяет судить о качестве полученных данных и доверии к прогнозам построенных на их основе. 

Как было описано выше так был про анализированн разброс признаков в разных образцах почерка одного человека, результаты представлены в таблице~\ref{table:summary:personal_dispersion}.   

\begin{longtable}[l]{| >{\raggedright}m{0.27\textwidth}
                     | >{\raggedright}m{0.32\textwidth}
                     | >{\centering\arraybackslash}m{0.33\textwidth}|}
  \caption{Исходные данные}
  \label{table:summary:personal_dispersion} \\
  \endfirsthead
  \caption*{Продолжение таблицы \ref{table:summary:personal_dispersion}}\\
  \hline Тип признака & Название признака & Отклонение \\ \hline
  \endhead
  \hline Тип признака & Название признака & Отклонение \\ \hline

    Наклон & & 8\% \\ \hline
    & наклон символов & 10\% \\ \hline
    & наклон строк & 6\% \\ \hline
    Интервал & & 11\% \\ \hline
    & интервал между символами & 15\% \\ \hline
    & интервал между словами & 9\% \\ \hline
    & интервал между строками & 8\% \\ \hline
    Длинна & & 7\% \\ \hline
    & длинна по горизонтали & 9\% \\ \hline
    & длинна по вертикали & 7\% \\ \hline
    & общая длинна & 8\% \\ \hline
    Время & & \\ \hline
    & длительность написания & 25\% \\ \hline
    & максимальное ускорение & 15\% \\ \hline
    & минимальное ускорение & 3\% \\ \hline
    & длительность написание по вертикали & 17\%\\ \hline
    & длительность написание по горизонтали & 17\%\\ \hline
    Прочие & & 5\% \\ \hline
    & количество линий & 1\% \\ \hline
    & частота текста & 5\% \\ \hline
    & площадь & 9\% \\ \hline
\end{longtable}
Основываясь на данных из таблицы были рассчитаны коэффициенты для биометрической аутентификации. Стоит отметить что не смотря на низкое значение отклонения такие параметры как количество линий и минимальное ускорение не могут иметь большой вес, так так имеют так же и низкое значение отклонения среди разных людей. Использование большого веса подобных коэффициентов приведет к возрастанию количества ошибок второго рода.

Выбор в качестве архитектуры приложения монолитной архитектуры доказал свою целесообразность, задержки между вызовами метод модулей минимальны, скорость обработки и передачи образцов почерка между модулями приложения достигается путем сведения к минимуму передачи данные по сети Интернет.

Как итог стоит отметить что тема работы полностью раскрыта, задачи исследования выполнены, а  поставленная цель достигнута.