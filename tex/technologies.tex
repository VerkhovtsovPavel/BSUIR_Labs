\section{Используемые технологии}
\label{sec:techs:intro}

Выбор технологий является важным предварительным этапом разработки сложных информационных систем. Платформа и язык программирования, на котором будет реализована система, заслуживает большого внимания, так как множество исследований показали, что выбор языка программирования значительно влияет на производительность труда программистов и качество создаваемого ими кода~\cite[c.~59]{mcconnell_2005}.

На выбор технологий повлияли следующие факторы:
\begin{itemize}
\item программное средство должно быть выполнено в виде клиент"=серверного приложения;
\item разрабатываемое ПО должно работать на операционных системах Linux, MacOS и Windows;
\item разработчик имеет опыт работы с объектно"=ориентированными и функциональными языками программирования.
\end{itemize}

Основываясь на выше перечисленных факторах было принято решение использовать в качестве основного языка программирования Scala как современный, активно набирающий популярность язык поддерживающий функциональную и объектно"=ориентированную парадигмы программирования.

Для реализации серверной часть приложения будет использоваться набор прикладных библотек (фреймворк) Spray, предназначенного реализации веб-приложений.

Для обеспечения параленьной обработки данных будет использоваться набор прикладных библотек (фреймворк) Akka реализующий акторную модель параленьных вычислений. 

Далее приводится подробная характеристика используемых технологий и языков программирования.


\subsection{Язык программирования Scala}
\label{sub:techs:scala}
Scala — мультипарадигмальный язык программирования, спроектированный кратким и типобезопасным для простого и быстрого создания компонентного программного обеспечения, сочетающий возможности функционального и объектно-ориентированного программирования~\cite{wiki_scala}.

Python поддерживает несколько парадигм программирования, в том числе структурное, объектно-ориентированное, функциональное, императивное и аспектно-ориентированное. Основные архитектурные черты — динамическая типизация, автоматическое управление памятью, полная интроспекция, механизм обработки исключений, поддержка многопоточных вычислений и удобные высокоуровневые структуры данных. Код в Python организовывается в функции и классы, которые могут объединяться в модули (они в свою очередь могут быть объединены в пакеты).

Эталонной реализацией Python является интерпретатор CPython, поддерживающий большинство активно используемых платформ. Он распространяется под свободной лицензией Python Software Foundation License, позволяющей использовать его без ограничений в любых приложениях, включая проприетарные. Есть реализации интерпретаторов для JVM (с возможностью компиляции), MSIL (с возможностью компиляции), LLVM и других. Проект PyPy предлагает реализацию Python на самом Python, что уменьшает затраты на изменения языка и постановку экспериментов над новыми возможностями.

Python -- активно развивающийся язык программирования, новые версии (с добавлением/изменением языковых свойств) выходят примерно раз в два с половиной года. Вследствие этого и некоторых других причин на Python отсутствуют стандарт ANSI, ISO или другие официальные стандарты, их роль выполняет CPython.

Отличительные особенности языка Python:
\begin{itemize}
  \item Простой. Python -- простой и минималистичный язык. Чтение хорошей программы на Python очень напоминает чтение английского текста. Такая псевдо-кодовая природа Python является одной из его самых сильных сторон. Она позволяет сосредоточиться на решении задачи, а не на нюансах использования самого языка.
  \item Свободный и открытый. Python – это пример свободного и открытого программного обеспечения – FLOSS (Free/Libré and Open Source Software). Пользователь имеет право свободно распространять копии этого программного обеспечения, читать его исходные тексты, вносить изменения, а также использовать его части в своих программах. В основе свободного ПО лежит идея сообщества, которое делится своими знаниями. Это одна из причин, по которым Python так хорош: он был создан и постоянно улучшается сообществом, которое хочет сделать его лучше.
  \item Язык высокого уровня. При написании программы на Python разработчику не нужно отвлекаться на такие низкоуровневые детали, как управление памятью и т.п.
  \item Портируемый. Благодаря своей открытой природе, Python был портирован на множество платформ. Программы на Python могут запускаться на любой из этих платформ без каких-либо изменений (если программа не использует системно-зависимые функции). Python можно использовать в GNU/Linux, Windows, FreeBSD, Macintosh, Solaris, OS/2, Amiga, AROS, z/OS, Palm OS, QNX, VMS, Psion, Acorn RISC OS, VxWorks, PlayStation, Sharp Zaurus, Windows CE и множестве других платформ.
  \item Интерпретируемый. Python не требует компиляции в бинарный код. Программа выполняется из исходного текста. Python сам преобразует этот исходный текст в некоторую промежуточную форму, называемую байткодом, а затем переводит его на машинный язык и запускает. Всё это заметно облегчает использование Python, поскольку нет необходимости заботиться о компиляции программы, подключении и загрузке нужных библиотек и т.д. Вместе с тем, это делает программы на Python намного более переносимыми.
  \item Объектно-ориентированный. Python поддерживает как процедурно-ориентированное, так и объектно-ориентированное программирование. Python предоставляет простые, но мощные средства для ООП, особенно в сравнении с такими большими языками программирования, как C++ или Java.
  \item Расширяемый. Если необходимо добиться очень высокой производительности некоторой части программы или использовать некоторые другие возможности более низкоуровневых языков, можно разработать эту часть программы на C или C++, а затем вызывать её из программы на Python.
  \item Встраиваемый. Python можно встраивать в программы на C/C++, чтобы предоставлять возможности написания сценариев их пользователям.
  \item Обширные библиотеки. Стандартная библиотека Python просто огромна. Она может помочь в решении самыхразнообразных задач, связанных с использованием регулярных выражений, генериро-ванием документации, проверкой блоков кода, распараллеливанием процессов, база-ми данных, веб-браузерами, CGI, FTP, электронной почтой, XML, XML-RPC, HTML, WAV файлами, криптографией, GUI и другими системно-зависимыми вещами. Всё это доступно абсолютно везде, где установлен Python. В этом заключается философия Python <<Всё включено>>.Кроме стандартной библиотеки, существует множество других высококачественных биб-лиотек, доступных в каталоге пакетов.
\end{itemize}

\subsubsection{Объектно-ориентированное программирование}
Дизайн языка Python построен вокруг объектно-ориентированной модели программирования. Реализация ООП в Python является элегантной, мощной и хорошо продуманной, но вместе с тем достаточно специфической по сравнению с другими объектно-ориентированными языками~\cite{wiki_python, byte_of_python}.

Особенности Scala с точки зрения объектно"=ориентированной парадигмы:
\begin{itemize}
  \item статическая сильная полная типизация с автоматическим выведением типов;
  \item наследование, в том числе использование примисей (Traits);
  \item полиморфизм;
  \item инкапсуляция;
  \item конструкторы, деструкторы;
  \item все операторы (+, -, *) являются методами;
  \item гибкое управление доступом к полям и методам;
  \item метапрограммирование;
  \item объекты компаньны (инкапсилируют статическая поля и методы).
\end{itemize}

\subsubsection{Функциональное программирование}
Особенности Scala с точки зрения функциональной парадигмы:
\begin{itemize}
  \item функции высших порядков;
  \item функции объект первого класса;
  \item оптимизация хвостовой рекурсии;
  \item сопостовление с образцов;
  \item поддержка неизменяемых структур данных;
  \item функциональные комбинаторы и композиции;
  \item частичное применение функции;
  \item <<ленивые>> вычисления;
  \item структурное переиспользование неизменяемых коллекций;
\end{itemize}

\subsection{Sandia National Laboratories PUF Analysis Tool (SPAT)}
\label{sec:techs:spat}
Sandia National Laboratories PUF Analysis Tool -- программное средство с открытым исходным кодом, созданное Sandia Corporation для симуляции физически неклонируемых функций, наглядной демонстрации и оценки их статистических свойств. Этот инструмент был создан главным образом для использования в образовательных целях.

Функции данной программы не ограничиваются только лишь симуляцией, она с равным успехом может анализировать статистические свойства реальных физических PUF.
Программа представляет собой набор модулей:
\begin{itemize}
  \item модуль взаимодействия с физическими устройствами;
  \item модуль симуляции PUF (может быть расширен пользователем);
  \item модуль анализа свойств PUF независимо от реализации (физическая или симуляция);
  \item графический интерфейс.
\end{itemize}
SPAT может быть использована для визуализации:
\begin{itemize}
  \item влияния шума на выходной сигнал PUF;
  \item метрик среднего значени шума в микросхеме;
  \item динамики расстояния Хэмминга выходного сигнала в зависимости от динамики входного;
  \item свойств случайности для данного чипа, таких как энтропия.
\end{itemize}
\begin{figure}[ht]
    \centering
    \includegraphics[width=0.7\textwidth]{spat.png}
    \caption{Окно программы SPAT с результатами статистического анализа виртуальной RO-PUF}
    \label{fig:techs:spat}
\end{figure}
