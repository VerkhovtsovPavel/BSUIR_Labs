\section{Методика использования программного средства}
\label{sec:manpage}

Программное средство определения параметров личности по анализу образцов почерка работает как веб-приложение. Ниже приведено описание вариантов использования и конфигурации модулей приложения.

\subsection{Руководство администратора}
\label{sec:manpage:admin_man}
Модули разрабатываемого программного средства разворачиваются как независимые компоненты, что позволяет добиться большой гибкости.

Конфигурация модулей осуществятся с помощью файла \emph{application.conf} расположенного в папке \emph{resources} в каталоге с файлами модуля. Для описания конфигурации используется формат HOCON позволяющий представить конфигурацию в легко-читаемом форме. 

\lstinputlisting[
    style=commonstyle,
    caption=Пример файла конфигурации модуля,
    label=lst:manpage:app_conf
]{src/application.conf}

Варьируя параметры \emph{host} и \emph{port} можно добиться развертывания модуля на произвольном адресе.

Помимо параметров развертывания и адресов других модулей файл конфигурации модуля доступа к данным содержит блок отвечающий за информацию о базе данных.

\lstinputlisting[
    style=commonstyle,
    caption=Пример блока конфигурации доступа к базе данных,
    label=lst:manpage:db_conf
]{src/mongodb.conf}

Наиболее важными параметрами являются \emph{connection-string}, \emph{user-name} и \emph{password}. С помощью параметра \emph{connection-string} указывается адрес сервера баз данных и имя базы, а параметра \emph{user-name} и \emph{password} отвечают за авторизационные данные. Так же при низкой скорость интернет соединения может возникнуть необходимость увеличить значение параметра \emph{connection-timeout}. 

\subsection{Руководство пользователя}
\label{sec:manpage:client_man}
Данный раздел содержит инструкции по использованию основных возможностей программного средства и позволят конечным пользователям быстрее научиться эффективно взаимодействовать с ним.