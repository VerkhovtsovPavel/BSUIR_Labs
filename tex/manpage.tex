\section{Методика использования программной системы}
\label{sec:manpage}

Программное средство распознавания уникальных неклонируемых идентификаторов цифровых устройств является комплексным продуктом и рассчитано на работу по схеме клиент-сервер. Ниже приведен возможный вариант развертывания и конфигурации программной системы.

\subsection{Настройка сервера}
\label{sec:manpage:server_setup}
Модуль проверки подлинности (сервер) может запускаться в одиночном режиме как обычное Python-приложение. Однако, такой подход не является рекомендованным и не обеспечивает достаточную секретность информации и стабильность системы. Сервер аутентификации рассчитан на работу в связке с WSGI-сервером и веб-сервером. Возможная схема их подключения показана на рисунке \ref{fig:manpage:nginx_proxy}.

\begin{figure}[!h]
    \centering
    \includegraphics[width=0.65\textwidth]{nginx_proxy.png}
    \caption{Схема возможного развертывания приложения в связке с WSGI-сервером и HTTP-сервером}
    \label{fig:manpage:nginx_proxy}
\end{figure}


Для обеспечения лучшей совместимости программных компонентов рекомендуется развертывание на устройствах под управением операционной системы семейства Linux. В листингах \ref{lst:manpage:gunicorncfg} и \ref{lst:manpage:nginxcfg} приведен пример рекомендуемой конфигурации менеджера процессов \emph{systemd} для запуска WSGI-сервера \emph{gunicorn} и настроек HTTP-сервера \emph{nginx} соответственно.


Программное средство рассчитано на работу по протоколу HTTPS в целях обеспечения секретности передаваемой информации. Для развертывания внутри закрытой сети достаточно использования самоподписанного (\emph{self-signed}) SSL/TLS"=сертификата, однако, если сценарий использования предполагает прохождение информационных потоков через публичные сети, например, Интернет, крайне рекомендуется использование валидного сертификата, подписанного доверенным центром сертификации. В обоих случаях необходимо правильная настройка сетевого экрана, для исключения перехвата данных или осуществления атаки отказа в обслуживании (DDoS) злоумышленником. При использовании вышеописанного сценария развертывания, достаточно разрешения входящего и исходящего трафика по стандартному HTTPS-порту 443.


\lstinputlisting[
    style=commonstyle,
    caption=Пример конфигурации systemd для запуска приложения в связке с WSGI-сервером gunicorn,
    label=lst:manpage:gunicorncfg
]{src/systemdgunicorn.conf}

\lstinputlisting[
    style=commonstyle,
    caption=Пример конфигурации nginx в качестве прокси-сервера,
    label=lst:manpage:nginxcfg
]{src/nginx.conf}

В целом, настройка и развертывание серверной части ничем не отличается от развертывания любого другого Python/WSGI-приложения, поэтому связанные с ними нюансы не являются частью данного руководства.

\subsection{Настройка модуля контроля доступа}
\label{sec:manpage:client_setup}
Модуль контроля доступа (Prover) является посредником между устройством и сервером аутентификации. Он не хранит секретных данных, однако может взаимодействать с реальными устройствами, поэтому он также должен быть развернут в безопасном окружении. Модуль контроля доступа сообщается с модулем проверки подлинности по протоколу HTTPS, как показано на рисунке \ref{fig:manpage:nginx_proxy}.


\subsection{Пример использования}
Программное средство поддерживает два этапа взаимодействия устройства с информационной системой -- регистрацию и аутентификацию. Для обеих целей в состав ПС входят утилиты \emph{register\_device.py} и \emph{authenticate\_device.py}. Они предоставляют базовый функционал и носят скорее демонстрационный характер, так как ПС рассчитана на интеграцию с существующей системой безопасности и предоставляет API для реализации нужд этой системы. Исходный код утилит представлен в листингах \ref{lst:manpage:authdev} и \ref{lst:manpage:regdev}.

\lstinputlisting[
    style=commonstyle,
    caption=Вспомогательный скрипт аутентификации утсройства,
    label=lst:manpage:authdev
]{src/authenticate_device.py}

\lstinputlisting[
    style=commonstyle,
    caption=Вспомогательный скрипт регистрации утсройства,
    label=lst:manpage:regdev
]{src/register_device.py}

\subsection{Расширение функциональности программного средства}
Программное средство может быть расширено для обеспечения поддержки различных типов устройств и подключений. Базовая поставка программной системы включает программную симуляцию PUF-устройства типа арбитр и класс (драйвер) для работы с ним. Добавление новых драйверов происходит путем переопределения базового класса devices.Device и реализации метода создания виртуального объекта устройства и собственного метода функции PUF, предназначенного для передачи входного сигнала устройству и получения выходного.

\lstinputlisting[
    style=commonstyle,
    caption={Пример переопределения класса Device для работы с программной реализацией PUF-арбитра},
    label=lst:architecture:software_arbiter
]{src/fulllisting/software_arbiter.py}
