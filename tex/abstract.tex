\sectioncentered*{Автореферат}

% всего страниц + ведомость

% Краткое введение

% Цели - Задачи - Объект - Предмет

\todo{Вычитать}
Опубликованность результатов диссертации

По теме диссертации опубликовано 3 печатных работ, из них 1 статья в рецензируемом издании, 2 работы в сборниках трудов и материалов международных конференций.

\todo{Вычитать, заменить на свое}
Структура и объем диссертации

Диссертация состоит из введения, общей характеристики работы, четырех глав, заключения, списка использованных источников, списка публикаций автора и приложения. В первой главе представлен анализ предметной области, выявлены основные существующие проблемы в рамках тематики исследования, показаны направления их решения. Вторая глава посвящена разработке архитектуры ПО и алгоритмов для систем вибрационного контроля, обеспечивающих непрерывную регистрацию и определение амплитудно-фазовых параметров. В третьей главе предложены методы формирования диагностических признаков и определения информативно значимых параметров для СППР по оценке ТС сложных механизмов на основе вейвлет-анализа и спектрального анализа. В четвертой главе предложена практическая реализация ПО для многоканальной системы вибрационного контроля и поддержки принятия решений, представлены результаты экспериментальных исследований метрологических характеристик и практического применения разработанной системы.

 % Объем работы составляет \totpages~с., \totfig{}~рис., \tottab{}~табл., \totref{}~источников.

ОСНОВНОЕ СОДЕРЖАНИЕ

Во введении определена область и указаны основные направления исследования, показана актуальность темы диссертационной работы, дана краткая характеристика исследуемых вопросов, обозначена практическая ценность     работы.

Разработка программной системы велась на языке Scala, с использованием библиотек Akka-Http для реализации веб"=интерфейса и Akka для обеспечения асинхронной обработки данных. 

Результатом диссертационной работы стала программное средство реализующее операции выделения признаков рукописного текста и определения психологические характеристики личности, предоставляющего защиту от несанкционированного доступа к данным пользователей.

\clearpage