\sectioncentered*{Автореферат}

% всего страниц + ведомость
\newcommand{\totpages}{\number\numexpr\getpagerefnumber{LastPage} + 1}

\todo[inline]{Переделать в автореферат}
\begin{center}
    Пояснительная записка \totpages~с., \totfig{}~рис., \tottab{}~табл., \totref{}~источников.
    \\ 
    \MakeUppercase{графология, метод опорных векторов, сегментация рукописного текста, машинное обучение}
\end{center}

Объектом исследования диссертационной работы является психологические характеристики личности.

Целью данной работы является разработка программного средства позволяющего установить психологические характеристики личности на основании анализа образцов почерка.

Проведен анализ методов сегментации рукописного текста, выделения признаков рукописного текста и классификации признаков рукописного текста.

Разработка программной системы велась на языке Scala, с использованием библиотек Spray для реализации веб"=интерфейса и Akka для обеспечения асинхронной обработки данных. 

Результатом диссертационной работы стала программное средство реализующее операции выделения признаков рукописного текста и определения психологические характеристики личности, предоставляющего защиту от несанкционированного доступа к данным пользователей.

\clearpage