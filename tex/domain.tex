\section{Обзор предметной области}
\label{sec:domain:intro}

В данном разделе будет произведён обзор программных средств аналогичных разрабатываемому в рамках дипломного проекта - определение психологических параметров человека по образцу почерка, а так же литературных источников. Проанализированы преимущества и недостатки различных подходов к выделению и классификации признаков рукописного текста.

\subsection{Графология}
\label{sub:domain:grafologic}
\emph{Графология} - это учение, постулирующее наличие устойчивой связи месту почерком и индивидуальными особенностями личности.

Идея использования почерка для выявления психологических параметров личности впервые была предложена в 1622 в книге итальянского профессора Камилло Бальдо <<Как узнать природу и качества человека, взглянув на букву, которую он написал>> ~\cite{kamillo_grafology}. Первым кто систематизировал знания стал Фландрэна аббат Мишон в 1872 году. Он проанализировал большое количество работ по графологии и образцов почерка и в своей книге <<Система графологии>> предложил \emph{метод Мишона}, он основывался на анализе штрихов, букв, слов, свободных движений, строк и пр.~\cite{mishon_grafology}

Начиная с середины 20 века графология начала рассматриваться как псевдонаучное учение~\cite{graphology_wiki}. По результатам исследования профессиональным графологам не удалось достоверно оценить трудовые способности человека. В среднем профессиональные графологи давали такую же по степени достоверности оценку, как и люди «с улицы»~\cite{neter_shakhar_psevdograph}~\cite{king_koehler_psevdograph}. В десятках исследований было показано отсутствие связи особенностей почерка с трудовыми способностями человека.

Тем не менее графология широко используется в современной практике отбора кадров~\cite{graphology_psyfactor}.

Основные признаки почерка, которые анализирует графологическая экспертиза:
\begin{enumerate}
  \item размер букв (очень маленькие, маленькие, средние, крупные);
  \item наклон букв (левый наклон, легкий наклон влево, правый наклон, резкий наклон вправо);
  \item направление почерка: (строчки ползут вверх, строчки прямые,  строчки ползут вниз);
  \item размашистость и сила нажима: (легкая, средняя, сильная, очень сильная);
  \item характер написания слов (склонность к соединению букв и слов, склонность к отдалению букв друг от друга, смешанный стиль);
  \item общая оценка (почерк старательный, почерк неровный, почерк небрежный, почерк неразборчивый).
\end{enumerate}

Перечисленные параметры почерка являются устойчивыми, но все же присутствует естественные отклонения параметров (длина, ширина, толщина, угол) от средних значений. Вариация становится наиболее заметной при изменение психологического состояния человека, например при страхе, беспокойстве, алкогольном опьянении.

\subsection{Анализ аналогов}
\label{sub:domain:analogs}

\subsubsection{ScriptAlyzeR}
\label{sub:domain:analogs:neuro_script} 

Программное средство <<ScriptAlyzeR>> является частью семейства программных средств для работы с рукописным текстов компании <<NeuroScript>> и представляет собой декстопное приложение для операционных систем Windows~\cite{analogs_scriptAlyzer}.

Основными возможностями ПС являются:
\begin{itemize}
  \item отслеживание положения, давления, ориентации с частотой 100-200 Гц;
	\item поддержка отслеживания руки, стилуса и мыши;
	\item измеряйте координацию одновременно две рук;
	\item отображение результатов в реальном времени;
	\item изменение толщины линии. Визуальная и звуковая обратная связь;
	\item искажение визуальная обратная связь. Поворот, перекос и отражение на мониторе компьютера в режиме реального времени;
	\item моделирование. Генерация рукописных цифровых данных с шумом и известными характеристиками штрихов;
	\item проверка непротиворечивости.
	\item анализ результатов. Статистика результатов с визуализацией;
	\item многостраничная записи. Разделение текст на слова и штрихи;
	\item внешние приложения. Полная интеграция с вашими собственными модулями с использованием сценариев MATLAB® или скомпилированных программ;
	\item оптически сканированные изображения.;
\end{itemize}

Основными недостатками ПС являются:
\begin{itemize}
  \item Поддержка только ОС семейства Windows (XP, 7, 8)
  \item Платное использование (799+ US\$)
\end{itemize}

Согласно утверждениям разработчиков ПС может быть использовано для оценки моторных функций, диагностики неврологических отклонений, а так же тестирования на состояние алкогольного опьянения.

\begin{figure}[ht]
    \centering
    \label{fig:domain:analogs:neuro_script}
    \includegraphics[width=0.7\textwidth]{figures/neuroscript.png}
    \caption{Программное средство <<ScriptAlyzeR>>}
\end{figure}

\subsubsection{Graphology}
\label{sub:domain:analogs:graphology} 

Программное средство <<Graphology>> является приложение для операционной системы Android разработанным компанией <<LH Apps>>~\cite{analogs_graphology}.

Программное средство <<Graphology>> предназначена для анализа почерка и определения характеристик личности. Алгоритм работы программы основан на обширных исследованиях и был создан при консультации профессиональных экспертов графологии.

Основными возможностями ПС являются:
\begin{itemize}

  \item Поддержка ОС Android;
  \item Многофакторная оценка параметров личности (почерк, подпись, рисунки);
  \item Выполнение анализа без доступа в интернет.
\end{itemize}

Основными недостатками ПС являются:
\begin{itemize}
  \item Поддержка только ОС Android;
  \item Поддержка только английского языка;
  \item Для ввода образцов почерка используется экран смартфона, что приводит к искажению в написании символов при низком разрешении и без использование стилуса.
\end{itemize}

\begin{figure}[ht]
    \centering
    \label{fig:domain:analogs:graphology}
    \includegraphics[width=0.7\textwidth]{figures/graphology_analog.jpeg}
    \caption{Программное средство <<Graphology>>}
\end{figure}

\subsubsection{Signature Analysis}
\label{sub:domain:analogs:signature_analysis} 

Программое средство <<Signature Analysis>> является приложение для операционной системы Android разработанным компанией <<Beyond Consultancy Services>>~\cite{analogs_signature_analysis}.

Программное средство <<Signature Analysis>> предназначена определения характеристик личности по образцу подписи. В разработке участвовал графолог с многолетним опытом, выступающей в качестве консультанта многих крупных компаний.

Основными возможностями ПС являются:
\begin{itemize}
  \item Поддержка ОС Android;
  \item Широкий спектр анализируемых параметров подписи (скорость, давление, длины, направления);
\end{itemize}

Основными недостатками ПС являются:
\begin{itemize}
  \item Поддержка только ОС Android;
  \item Платный анализ каждой подписи (0,83 US\$);
  \item Для работы необходимо интернет соединение;
  \item Для ввода образцов почерка используется экран смартфона, что приводит к искажению в написании символов при низком разрешении и без использование стилуса.
\end{itemize}

\begin{figure}[ht]{}
    \centering
    \label{fig:domain:analogs:signature_analysis}
    \includegraphics[height=0.5\textheight]{figures/analog_signature_analysis.png}
    \caption{Программное средство <<Signature Analysis>>}
\end{figure}

\subsubsection{My Graphology}
\label{sub:domain:analogs:my_graphology}

Программное средство <<My Graphology>> является приложение для операционной системы Android разработанным компанией <<PENS>>~\cite{analogs_my_graphology}.

Основными возможностями ПС являются:
\begin{itemize}
  \item Поддержка ОС Android;
  \item Использовать для ввода экран или фотографию почерка;
  \item Выполнение анализа без доступа в интернет.
\end{itemize}

Основными недостатками ПС являются:
\begin{itemize}
  \item Поддержка только ОС Android;
  \item В разработке не участвовали эксперты графологи;
  \item Поддержка только испанского языка интерфейса.
\end{itemize}

\begin{figure}[ht]
    \centering
    \label{fig:domain:analogs:my_graphology}
    \includegraphics[width=0.55\textwidth]{figures/analog_my_graphology.jpeg}
    \caption{Программное средство <<My Graphology>>}
\end{figure}

\subsubsection{GRAPHOLOGY signature analysis}
\label{sub:domain:analogs:graphology_sign_analysis}

Программное средство <<GRAPHOLOGY signature analysis>> является приложение для операционной системы Android разработанным компанией <<DokThor>>~\cite{analogs_graphology_sign_analysis}. Программное средство <<GRAPHOLOGY signature analysis>> предназначена определения характеристик личности по образцу подписи.

Основными возможностями ПС являются:
\begin{itemize}
  \item Поддержка ОС Android;
  \item Выполнение анализа без доступа в интернет;
  \item Предоставление характеристик по личности по 5 основным критериям.
\end{itemize}

Основными недостатками ПС являются:
\begin{itemize}
  \item Поддержка только ОС Android;
  \item В разработке не участвовали эксперты графологи;
  \item Механизм ввода подписи неочевиден;
  \item Для ввода образцов почерка используется экран смартфона, что приводит к искажению в написании символов при низком разрешении и без использование стилуса.
\end{itemize}

\begin{figure}[ht]
    \centering
    \label{fig:domain:analog:graphology_sign_analysis}
    \includegraphics[height=0.5\textheight]{figures/analog_graphology_sign_analysis.png}
    \caption{<<GRAPHOLOGY signature analysis>>}
\end{figure}

\subsubsection{Graphology Lite}
\label{sub:domain:analogs:graphology_lite}

Программное средство <<Graphology Lite>> является приложение для операционной системы Android разработанным компанией <<Hyperborea>>~\cite{analogs_graphology_sign_analysis}.

Основными возможностями ПС являются:
\begin{itemize}
  \item Поддержка ОС Android;
  \item Выполнение анализа без доступа в интернет;
  \item Использовать для ввода экран или фотографию почерка.
\end{itemize}

Основными недостатками ПС являются:
\begin{itemize}
  \item Поддержка только ОС Android;
  \item В разработке не участвовали эксперты графологи;
  \item Бесплатная версия позволяет произвести анализ только одного образца. Платная версия стоит 1.05 US\$.
\end{itemize}

\begin{figure}[ht]
    \centering
    \label{fig:domain:analogs:graphology_lite}
    \includegraphics[height=0.5\textheight]{figures/analog_graphology_lite.png}
    \caption{Программое средство <<Graphology Lite>>}
\end{figure}

\subsection{Анализ литературных источников}
\label{sub:domain:literary_sources}

Публикации и научные статьи на темы схожие с темой данной работы можно условно разделить по следующим признакам:
\begin{itemize}
  \item метод сегментации изображения;
  \item метод классификации признаков изображения.
\end{itemize}



\subsection{Постановка задачи}
В результате выполнения дипломного проекта должно быть разработано программное средство определения психологических параметров личности по образцу подчетка, реализующее процедуры выбеления признаков почерка из изображения и их классификацию, а так же механизм авторизации для обеспечения секретности данных. К разрабатываемому программному средству предъявляются следующие требования:
\begin{itemize}
\item разрабатываемое ПО должно работать на операционных системах Linux, MacOS и Windows;
\item программное средство должно быть выполнено в виде клиент"=серверного приложения;
\item программное средство должно поддерживать русской язык интерфейса;
\item программное средство должно поддерживать работу как в режиме выделения признаков рукописного текста, так и в режиме их классификации;
\item программное средство должно предусматривать механизм регистрации, аутентификации и авторизации пользователей.
\end{itemize}
