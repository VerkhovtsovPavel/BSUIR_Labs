\begin{center}
{\bfseries КРАТКОЕ ВВЕДЕНИЕ}
\end{center}
\label{sec:intro}

Применение информационных систем в коммерческих компании стало необходимым условием выживания на рынке, а для социальных и государственных организаций и частных некоммерческих проектов использование информационных систем является единственным способом предоставить уровень сервиса соответствующий современным стандартам. 

При глобальном применении информационных технологий встает вопрос об организации и разграничении доступа к информации обрабатываемой системой, так как часть информации может представлять коммерческую тайну или содержать персональные данные.

На фоне глобального применения информационных технологий вопросы подбора квалифицированного и исполнительного персонала встают как никогда остро, так как все чаще возникают ситуации когда человек является самым медленным и ненадежным элементов производственной цепочки, а ошибки могут обернуться для компании не только потерями прибыли, но репутации.

В рамках данной работы рассматриваются возможность автоматизации методов графологии, биометрической аутентификации и анализа неврологических отклонений на основе образцов почерка для определения психологических характеристик человека, вероятных неврологических отклонений, а так же проведение аутентификации.
Разрабатываемое программное средство может показать разнообразные способы использования набора признаков почерка в целях определения характеристик личности, биометрической аутентификации и определения неврологических отклонений.
\clearpage