\sectioncentered*{Заключение}
\addcontentsline{toc}{section}{Заключение}
\label{sec:outro}

В данном дипломном проекте были проанализированы существующие аналоги разрабатываемого программного средства и изучены литературные источники, на основании анализа были составлены и специфицированны требования к разрабатываемому ПС. Тема проекта полностью раскрыта, поставленная цель достигнута, а задачи исследования выполнены.

В качестве метода сегментации текста используется статический метод так он дает достойный результат и обладает хорошими временными характеристиками, а в качестве метода классификации используется метод опорных векторов из-за быстрого времени обучения и распознавания и наличия достаточной обучающей выборки для его использования.

На основании требований к программному средству в качестве основного языка программирования был выбран Scala, как современный, мультипарадигмальный, компилируемый язык что обеспечит высокую производительность вместе с выразительностью и компактностью кода. Так же был использован фреймворк Akka для реализации параллельной обработки данных и фреймворк Spray для реализации веб-сервера. Данные библиотеки были выбраны из-за использования неблокирующих операций и модели акторов что обеспечивает общий подход к разработке и упрощает сопровождение.

На основании спецификации требований и используемых технологий была разработана обобщенная архитектура будущего программного средства состоящая из моделя выделения признаков почерка, модуля определения параметров личности, модуля контроля доступа и модуля доступа к базе данных.

После разработки было осуществлено тестирование программного средства на соответствие заявленным функциональным и нефункциональным требованиям.

Было разработано руководство пользователя содержащее инструкции по использованию основных функций ПО, а так же инструкции по реконфигурации для администраторов.

В рамках дипломного проекта спроектировано, создано и протестировано программное средство определения параметров личности по образцам почерка. Однако, за рамками рассматриваемой темы осталось множество альтернативных параметров почерка и методов классификации, включая комбинацию методов, а также смежных областей применения, таких как установления авторства рукописного текста. Вышеперечисленное может стать основой для будущих научных исследований.