\sectioncentered*{Заключение}
\addcontentsline{toc}{section}{Заключение}
\label{sec:outro}

В данном дипломном проекте был рассмотрен вопрос использования физически неклонируемых функций в качестве субъектов аутентификации. В рамках дипломного проекта была разработана программное средство с клиент-серверной архитектурой, предоставляющее возможности для удалённой проверки подлинности цифровых устройств и рассчитанное на интеграцию в новые или существующие системы информационной или физической безопасности. В разработанном проекте были использованы современные теоретические наработки, технологии машинного обучения и статистического моделирования, а также современные программные инструменты и языки программирования. Реализованные в проекте функций обеспечивают достаточный уровень секретности передаваемых данных, а также отвечают высоким требованиям производительности и стабильности программного средства.

В итоге, тема дипломного проекта была раскрыта, а в его рамках создано комплексное многофункциональное программное средство. Однако, за рамками рассматриваемой темы осталось множество альтернативных протоколов аутентификации и их реализаций, а также аппаратная реализация физически неклонируемой функции и алгоритмов, использованных в протоколе. В дальнейшем планируется как можно более приблизить программное средство к использованию с реальными цифровыми устройствами для решения задач не только проверки подлинности, но и смежных, таких как защита интеллектуальной собственности.
