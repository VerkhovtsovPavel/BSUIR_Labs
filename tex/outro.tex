\sectioncentered*{Заключение}
\addcontentsline{toc}{section}{Заключение}
\label{sec:outro}

В данном диссертационной работе были проанализированы существующие аналоги разрабатываемого программного средства и изучены литературные источники, на основании анализа были составлены и специфицированны требования к разрабатываемому ПС. Тема работы полностью раскрыта, поставленная цель достигнута, а задачи исследования выполнены.

В качестве метода сегментации текста используется статический метод так он дает достойный результат и обладает хорошими временными характеристиками, для задач определения неврологических отклонений и биометрической аутентификации использовались статистические методы, в качестве метода классификации используется метод опорных векторов из-за быстрого времени обучения и распознавания и наличия достаточной обучающей выборки для его использования, а в качестве алгоритма построения скелета образца метод уточнения областей, так как учитываю структуру и особенности входных образцов почерка он дает оптимальный результат скелетизации за минимальное время при сравнении с другими алгоритмами.

На основании требований к программному средству в качестве основного языка программирования был выбран Scala, как язык обеспечивающий высокую производительность вместе с выразительностью и компактностью кода. Фреймворки Akka и Akka-Http были выбраны из-за использования неблокирующих операций и модели акторов что обеспечивает общий подход к разработке и упрощает сопровождение.

На основании спецификации требований и используемых технологий была разработана обобщенная архитектура будущего программного средства состоящая из моделя выделения признаков почерка, модуля определения характеристик личности, модуль биометрической аутентификации, модуль определения неврологических отклонений, модуля контроля доступа и модуля доступа к базе данных.

После разработки было осуществлено тестирование программного и разработана инструкции по реконфигурации и администрированию.

В рамках диссертационной работы было спроектировано, создано и протестировано программное средство анализа почерка на основе траектории линий в психологии, медицине и информационной безопасности, а так же проанализированны полученные результаты. 

Полученные результаты могут лечь в основу новых научных работ или быть расширены путем добавления ролевой модели доступа, дополнительных признаков почерка или реализацией мобильного клиента.