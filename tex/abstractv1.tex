\sectioncentered*{Реферат}
\thispagestyle{empty}
%%
%% ВНИМАНИЕ: этот реферат не соответствует СТП-01 2013
%% пример оформления реферата смотрите здесь: http://www.bsuir.by/m/12_100229_1_91132.docx
%%
\noindent БГУИР ДП 1-40 01 01 03 013 ПЗ
\\

\textbf{Булаш,\,М.В.} Программное средство распознавания уникальных неклонируемых идентификаторов цифровых устройств : пояснительная записка к дипломному проекту / М.В.~Булаш. - Минск : БГУИР, 2016. - \pageref*{LastPage}~с.
\\

Пояснительная записка \pageref*{LastPage}~с., \totfig{}~рис., \tottab{}~табл., \totref{}~источников.
\\

\MakeUppercase{цифровые устройства, физически неклонируемые функции, моделирование физической системы, машинное обучение, аутентификация, контроль доступа}
\\

\emph{Цель проектирования}: Исследование способов аутентификации цифровых устройств, оборудованных ФНФ, в информационных системах. Создание программного средства распознавания уникальных неклонируемых идентификаторов, предназначенного для осуществления аутентификации и контроля доступа на основе использования специальных компонентов устройств - физически неклонируемых функций.

\emph{Технологии и алгоритмы}: Разработка программной системы велась на языке Python, с использованием библиотек Twisted, Werkzeug и веб-фреймворка Klein для обеспечения асинхронного сетевого взаимодействия. В процессе работы использовались среда разработки Sublime Text 3 с расширениями SublimePythonIDE. В ходе работы исследованы и применены алгоритмы математического моделирования физических систем и коррекции цифровых сигналов.

\emph{Результаты работы}: cсоздана программная система, реализующая протокол аутентификации физических устройств в информационной системе, и предоставляющая достаточный уровень защиты от несанкционированного доступа к секретным данным.

\emph{Область применения результатов}: Программный продукт позиционируется как основа для построения более сложной системы, например, системы контроля и управления доступом.


\clearpage
