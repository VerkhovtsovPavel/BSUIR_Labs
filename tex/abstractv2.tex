\sectioncentered*{Реферат}

% всего страниц + ведомость
\newcommand{\totpages}{\number\numexpr\getpagerefnumber{LastPage} + 1}

\begin{center}
    Пояснительная записка \totpages~с., \totfig{}~рис., \tottab{}~табл., \totref{}~источников.
    \\
    \MakeUppercase{цифровые устройства, физически неклонируемые функции, моделирование физической системы, машинное обучение, аутентификация, контроль доступа}
\end{center}

Объектом исследования дипломного проекта является возможность и необходимые условия практического использования физически неклонируемых функций (ФНФ) в системах физической криптографии, аутентификации и контроля доступа. В частности, рассматриваются способы аутентификации цифровых устройств, оборудованных ФНФ, в информационных системах.

Цель дипломного проектирования - разработка программной реализации протокола аутентификации цифровых устройств. Подробно рассмотрены и реализованы механизмы взаимодействия собственно устройства и сервера аутентификации на этапе регистрации устройства в информационной системе и на этапе проверки его подлинности.
Так как целью является именно реализация взаимодействия, устройство, оборудованное ФНФ решено выполнить в виде программной эмуляции в целях упрощения тестирования.

Разработка программной системы велась на языке Python, с использованием библиотек Twisted, Werkzeug для обеспечения асинхронного сетевого взаимодействия. В процессе работы использовались среда разработки Sublime Text 3 с расширениями SublimePythonIDE. Пояснительная записка оформлена с помощью технологии LaTeX и пакета утилит LatexTools. В ходе работы исследованы и применены алгоритмы математического моделирования физических систем и коррекции цифровых сигналов.

Результатом дипломного проекта стала легковесная программная система, реализующая протокол аутентификации физических устройств в информационной системе, и предоставляющая достаточный уровень защиты от несанкционированного доступа к секретным данным. Программный продукт позиционируется как основа для построения более сложной системы, например, системы контроля и управления доступом.

Разработанное приложение является экономически эффективным, оно полностью оправдывает средства, вложенные в его разработку.

\clearpage