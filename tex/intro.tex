\sectioncentered*{Введение}
\addcontentsline{toc}{section}{Введение}
\label{sec:intro}

В современном мире наблюдается тенденция к глобальной автоматизации производства управления и учета. Роботы на предприятиях, система учета кадров, клиентов, товаров и бухгалтерии, и множество других применений находит автоматизация на сегодняшний момент.

На фоне этого вопросы доверия к управленческому и производственному персоналу, а так же вопросы оптимизации человеческого труда с каждым днем все острее, так как все чаще возникают ситуации когда человек
является самым медленным элементов цепочки.

Графология - это дисциплина, утверждающая о наличии строгой связи между почерком и индивидуальными особенностями личности. Оперирую такими признаками как наклон, интервал между строками, форма полей и д.р. графологи могут установить психологическую устойчивость, склонность к агрессии, умение работать в команде и другие качества потенциально интересные работодателю.

Целью данного проекта является исследования способов автоматизации определения психологических параметров личности по образцу почерка. Создание программного средства определения психологических характеристик по анализу образцов почерка.  
Объектом исследования дипломного проекта является возможность и условия использования графологических методов в информационных системах.
Предметом исследования является автоматизация графологических методов направленных на установление параметров личности.

Задачами исследования являются:
\begin{enumerate}
  \item изучить научно-методическую и справочную литературу по вопросу графологии и графологических методов;
  \item сравнить существующие подходу к автоматизации графологических методов;
  \item обосновать практическую пользу применение исследуемого предмета;
  \item выявить требования в программному средству;
  \item разработать программное средство автоматизации определения психологических параметров личности по образцу почерка;
  \item сравнить полученные данные с имеющимися.
\end{enumerate}
изучить научно-методическую и справочную литературу по вопросу графологии и графологических методов

В рамках данного проекта рассматриваются возможность автоматизации использования методов графологии для определения индивидуальных параметров человека с использованием технологий компьютерного зрения и машинного обучения.

Разрабатываемое программное средство может быть использовано отделами кадров при отборе кандидатов, а так же руководством компаний при назначении работника на руководящую должность.