\sectioncentered*{Введение}
\addcontentsline{toc}{section}{Введение}
\label{sec:intro}

\todo[inline]{Расширить в учетом неврологии и информационной безопасности}
В современном мире наблюдается тенденция к глобальной автоматизации производства, управления и учета. Роботы на предприятиях, система учета кадров, клиентов, товаров и бухгалтерии, и множество других применений находит автоматизация на сегодняшний момент.
На фоне этого вопросы доверия к управленческому и производственному персоналу с каждым днем все острее, так как все чаще возникают ситуации когда человек является самым медленным элементов производственной цепочки.

\todo[inline]{Добавить описание микрографии и аутентификации}
Графология – это дисциплина, утверждающая о наличии связи между почерком и индивидуальными особенностями личности. Оперирую такими признаками как наклон, интервал между строками, форма полей графологи могут установить психологическую устойчивость, склонность к агрессии, коммуникабельность и другие качества интересные работодателю.

\todo[inline]{Обновить цель, предмет, объект}
Целью данной работы является разработка программного средства определения психологических характеристик по анализу образцов почерка.
Объектом исследования диссертационной работы является психологические характеристики личности.
Предметом исследования является автоматизация графологических методов направленных на установление параметров личности.

\todo[inline]{Добавить пункты с оставшимися применениями}
Задачами исследования являются:
\begin{enumerate}
  \item изучить научно-методическую и справочную литературу по вопросу графологии и графологических методов;
  \item провести сравнительный анализ существующие подходу к автоматизации выделения признаков почерка;
  \item обосновать практическую пользу применения исследуемого \mbox{предмета};
  \item определить требования в программному средству;
  \item разработать программное средство автоматизации биометрической идентификации по образцу почерка;
  \item разработать программное средство автоматизации определения неврологических заболеваний по образцу почерка;
  \item разработать программное средство автоматизации определения психологических параметров личности по образцу почерка.
\end{enumerate}

В рамках данного проекта рассматриваются возможность автоматизации использования методов графологии для определения индивидуальных параметров человека с использованием машинного обучения.
Разрабатываемое программное средство может быть использовано отделами кадров при отборе кандидатов, а так же руководством компаний при назначении работника на руководящую должность.