\sectioncentered*{Введение}
\addcontentsline{toc}{section}{Введение}
\label{sec:intro}

В современном мире наблюдается повсеместное ускорение производство, улучшение качества товаров и услуг засчет внедрения информационных систем учета и анализа различной степени сложности.

Применение информационных систем в коммерческих компании стало необходимым условием выживания на рынке, а для социальных и государственных организаций и частных некоммерческих проектов использование информационных систем является единственным способом предоставить уровень сервиса соответствующий современным стандартам. 

При глобальном применении информационных технологий встает вопрос об организации и разграничении доступа к информации обрабатываемой системой, так как часть информации может представлять коммерческую тайну или содержать персональные данные.

На фоне глобального применения информационных технологий вопросы подбора квалифицированного и исполнительного персонала встают как никогда остро, так как все чаще возникают ситуации когда человек является самым медленным и ненадежным элементов производственной цепочки, а ошибки могут обернуться для компании не только потерями прибыли, но репутации.

Графология -- это дисциплина, утверждающая о наличии связи между почерком и индивидуальными особенностями личности. Оперирую такими признаками как наклон, интервал между строками, форма полей графологи могут установить психологическую устойчивость, склонность к агрессии, коммуникабельность и другие качества интересные работодателю.

Биометрическая аутентификация -- это процедура сбора биометрического параметра с последующей возможностью предъявления пользователем своего уникального биометрического параметра и процесс сравнения его со всей базой имеющихся данных, с целью установления личности или причастности к группе лиц.

Микрография -- это нарушение письма выражающееся в нечеткости контуров и уменьшение букв. Наблюдается при паркинсоническом синдроме или депрессии, иногда при кататонических состояниях.

Целью данной работы является разработка программного средства анализа образцов почерка на основе анализа траектории линии.
Объектом исследования диссертационной работы является набор характеристик почерка достаточный для проведения графологического анализа, биометрической аутентификации и определения неврологических заболеваний.
Предметом исследования является автоматизация методов графологии, биометрической аутентификации и определения неврологических заболеваний.

Задачами исследования являются:
\begin{enumerate}
  \item изучить научно-методическую и справочную литературу по вопросу графологии и графологических методов;
  \item провести сравнительный анализ существующие подходу к автоматизации выделения признаков почерка;
  \item обосновать практическую пользу применения исследуемого \mbox{предмета};
  \item определить требования в программному средству;
  \item разработать программное средство автоматизации биометрической аутентификации по образцу почерка;
  \item разработать программное средство автоматизации определения неврологических отклонений по образцу почерка;
  \item разработать программное средство автоматизации определения психологических характеристик личности по образцу почерка.
\end{enumerate}

В рамках данной работы рассматриваются возможность автоматизации методов графологии, биометрической аутентификации и анализа неврологических заболевания на основе образцов почерка для определения психологических характеристик человека, вероятных неврологических заболеваний, а так же проведение аутентификации.
Разрабатываемое программное средство может показать разнообразные способы использования набора параметров почерка в целях определения характеристик личности, биометрической аутентификации и определения неврологических заболеваний.