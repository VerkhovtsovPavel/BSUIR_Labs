\section{Разработка программного средства}

В данном раздела описывается процесс разработки программного средства.
\subsection{Обучение классификатора}
Правильная классификация черт личности является, пожалуй, основной задачей данного проекта, результат зависит от предподготовки данных, процедуры оценки результата и финального качества обучения.

\begin{figure}[h]
    \centering
    \includegraphics[width=0.7\textwidth]{figures/SVM_flow.png}
    \caption{Алгоритм обучение}
    \label{fig:develoipment:svm_flow}
\end{figure}

Общая схема подготовки, обучения и использования классификатора представлена на рисунке~\ref{fig:develoipment:svm_flow}.

После обучения классификатора можно переходить к классификации образов в листинге~\ref{listing:development:classification} представлен метод классификации нового образца почерка на основе обученной модели. Всего в программном средстве различается 16 типов личности определяющих характеристики.

\lstinputlisting[
    style=commonstyle,
    caption=Метод классификации личности по параметрам почерка,
    label=listing:development:classification
]{src/evaluate_single_instance.java}

Для обучения классификатора будет использоваться обучающая выборка <<IAM Handwriting Database>>~\cite{IAM_handwriting_database} состоящая примерно из 1200 образцов текста с заранее выделенными параметрами почерка либо на основании выделенных параметров можно рассчитать параметры используемые в данной работе. Образцы почерка представлены изображениями в формате png, а параметры XML документом, пример параметров представлен в листинге~\ref{listing:development:sample_set}.
\lstinputlisting[
    style=commonstyle,
    caption=Пример XML-документа описывающего параметры почерка,
    label=listing:development:sample_set
]{src/sample_set.xml}

\subsection{Иерархия классов}
\begin{figure}[!ht]
    \centering
    \includegraphics[width=1\textwidth]{figures/classes-fdp.png}
    \caption{Иерархия классов}
    \label{fig:develoipment:class_fdp}
\end{figure}

Общая диаграмма иерархии классов представлена на рисунке~\ref{fig:develoipment:class_fdp}. Все классы отвечающие за обработку запросов и данных, в частности \emph{ModelActor}, \emph{ParserActor}, \emph{ServiceActor} являются наследниками класса \emph{Actor} библиотека Akka. Это свидетельствует о том что обработка информации на всех уровнях приложения происходит асинхронно.

Классы \emph{ItemNotFound}, \emph{Forbidden}, \emph{Unauthorized} являются наследниками класса \emph{Response} и представляют собой обертки для HTTP ответов на различные ошибочные ситуации на стороне клиента.

\subsection{Маршрутизация запросов}
Библиотека Spray, описанная в раздела~\ref{sec:techs:spray}, имеет очень выразительный внутренний язык описания разбора, обработки и маршрутизации запросов к сервису. Модуль доступа к данным является связующий звеном между модулями приложения и базой данных. Поскольку одной из задач программного средства является предотвращение не авторизированного доступа к данным. Каждый обработчик запросов содержит секцию отвечающую за проверку прав доступа.
\lstinputlisting[
    style=commonstyle, 
    caption=Маршрутизация запросов к серверу в модуле доступа к данным,
    label=listing:development:db_rout
]{src/routing.scala}
Конструкция \emph{verify} отвечает за проверку прав доступа, а \emph{path} отвечает за маршрутизацию и разбор параметров запросов. 

\subsection{Хранение данных}
Для хранения учетных и пользовательских данных используется СУБД MongoDB. Условно все данные можно разделить на образцы изображений, листинг~\ref{listing:development:json:sample} и описание пользовательских данных, листинг~\ref{listing:development:json:user}. Для хранения используется формат JSON что упрощает преобразование информации при общения сервисов между собой и с клиентом. 

\lstinputlisting[
    style=commonstyle,
    caption=Пример JSON-документа описывающего образец текста,
    label=listing:development:json:sample
]{src/sample_item.json}

\lstinputlisting[
    style=commonstyle,
    caption=Пример JSON-документа описывающего пользователя,
    label=listing:development:json:user
]{src/user_item.json}

\subsection{Сборка и развертывание}
Для получения последнюю версию исходного кода программного средства используйте команду <<git clone>> в корневом каталоге ПС. Разрешение зависимостей, сборка и развертывание программного средства выполняется с помощью автоматической системы сборки sbt~\cite{sbt}, поэтому перед началом сборки необходимо установить данный инструмент. Программное средство в своей работе использует ряд библиотек. При помощи команды <<sbt>> в корневом каталоге ПС будет произведены загрузка всех необходимых библиотек включая компилятор языка Scala, а так же компиляция модуля программного средства.
Для развертывания модуля программного средства необходимо указать параметры развертывания в файле \emph{application.conf}, подробнее об этом рассказывается в главе~\ref{sec:manpage:admin_man}, и выполнить команду <<sbt run>>.

В данном разделе была рассмотрена общая структура обучение классификатора на основе метода опорных векторов, диаграмма иерархии классов разработанного программного средства, формата хранения внутрих данных приложения, таких как учетные записи пользователей и образцы почерка, а так же внешние данные, обучающую выборку. На данном этапе разработка и отладка программное средства закончена и можно переходи к его тестированию.