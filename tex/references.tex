% Зачем: Изменение надписи для списка литературы
% Почему: Пункт 2.8.1 Требований по оформлению пояснительной записки.
\renewcommand{\bibsection}{\sectioncentered*{Список использованных источников}}
\phantomsection\pagebreak% исправляет нумерацию в документе и исправляет гиперссылки в pdf
\addcontentsline{toc}{section}{Список использованных источников}

% Зачем: Печать списка литературы. База данных литературы - файл bib/database.bib
\bibliography{bib/database}

\bigskip
\textbf{Список публикаций соискателя}
\bigskip

1-А. Верховцов, П. А. Диагностика неврологических заболеваний на основе образцов почерка / П. А. Верховцов // Компьютерные системы и сети: материалы 54-й научной конференции аспирантов, магистрантов и студентов, Минск, 23 – 27 апреля 2018 г. / Белорусский государственный университет информатики и радиоэлектроники. – Минск, 2018. – С. 54 – 55.

2-А. Верховцов, П.А. Методы и подходы автоматизации графологического анализа / П.А. Верховцов // Актуальные направления научных исследований XXI века: теория и практика. Т. 5. № 10 (36). / Воронежский государственный лесотехнический университет им. Г.Ф. Морозова. – Воронеж, 2017. – С. 99-102.

3-А. Верховцов, П.А. Аутентификация на основе рукописной подписи / П.А. Верховцов // Актуальные направления научных исследований XXI века: теория и практика / Воронежский государственный лесотехнический университет им. Г.Ф. Морозова. – Воронеж, 2018.