\section{Функциональные требования}
\label{sec:freq}

Основываясь на требованиях изложенных в разделе \ref{sec:domain:requirements} и диаграмме вариантов использования разработана спецификация требований к разрабатываемому ПС.

\begin{figure}[ht]
\centering
    \includegraphics[scale=0.53]{figures/use_case.png}  
    \caption{Диаграмма вариантов использования}
  \label{fig:freg:usecase}
\end{figure}

\subsection{Регистрация пользователя}
\label{sec:freq:reg}
\begin{itemize}
	\item возможность регистрации должна быть доступна из пользовательского веб-интерфейса;
	\item пароль при регистрации должен проверяться на сложность(должен содержать не менее 8 символов в верхнем и нижнем регистрах и цифры);
	\item авторизационные данные пользователя должны передаваться по защищенному соединению;
\end{itemize}

\subsection{Авторизация пользователя}
\label{sec:freq:auth}
\begin{itemize}
	\item авторизационные данные пользователя должны передаваться по защищенному соединению;
	\item пароль пользователя не должен передаваться в явном виде (вычисления хеша в браузере);
 	\item сообщение об ошибке при вводе неверное логина или пароля не должно сообщать что именно введено неправильно.
\end{itemize}

\subsection{Просмотр сохраненных образцов почерка}
\label{sec:freq:show}
\begin{itemize}
	\item список ранее загруженных образцов доступен на главной странице пользователя;
	\item изображение и дополнительная информация(признаки почерка, параметры личности) должны загружаться только после выбора изображения из списка;
	\item пользователь имеет возможность запустить процесс выделения признаков почерка;
	\item пользователь имеет возможность запустить процесс параметров личности, если процесс выделения признаков почерка еще не был выполнен, и по окончанию начнется процесс определения параметров личности;
\end{itemize}

\subsection{Удаление сохраненных образцов почерка}
\label{sec:freq:delete}
\begin{itemize}
	\item возможность удаления сохраненных образцов почерка должна быть доступна из пользовательского веб-интерфейса страница просмотра образца;
	\item при удалении должно запрашиваться подтверждение действия пользователя с сообщением о последствиях;
	\item реальное удаление образца происходит через неделю после подтверждения удаления пользователем (возможность восстановить файл при ошибочном удалении).
\end{itemize}

\subsection{Добавление нового образца почерка}
\label{sec:freq:add}
\begin{itemize}
	\item возможность добавить новый образец доступна на главной странице пользователя;
	\item возможно добавить образцы в форматах jpg(jpeg), bmp и png;
	\item добавляемый образце должен иметь размер более 0 и менее 10 MB;
	\item после добавления нового образца пользователь переходит на страницу просмотра образца;
\end{itemize}

\subsection{Выделение признаков образца почерка}
\label{sec:freq:extract_features}
\begin{itemize}
	\item возможность выделение признаков образца почерка должна быть доступна из пользовательского веб-интерфейса страница просмотра образца;
	\item до завершения выделения признаков на странице отображается индикатор отработки в виде вращающегося круга.
\end{itemize}

\subsection{Определение параметров личности}
\label{sec:freq:psiho_analysis}
\begin{itemize}
	\item возможность определение параметров личности должна быть доступна из пользовательского веб-интерфейса страница просмотра образца;
	\item до завершения определение параметров личности на странице отображается индикатор отработки в виде вращающегося круга.
\end{itemize}

\subsection{Пакетное добавление образцов почерка}
\label{sec:freq:package_add}
\begin{itemize}
	\item возможность пакетного добавить новый образцов доступна на главной странице администратора;
	\item процессы выделения признаков и параметров личности начинаются автоматически;
	\item до завершения обработки изображений на странице отображается индикатор отработки в виде вращающегося круга.
\end{itemize}

\subsection{Пользовательский интерфейс программного средства}
\begin{itemize}
	\item пользовательский интерфейс представляет собой Web-страницу;
	\item пользовательский интерфейс поддерживает русский и английский языки;
	\item возможность смены языка доступна пользователю на любой странице;
	\item поддержка Google Chrome и Firefox последних версий;
\end{itemize}

Поддержка операционных систем Windows, Linux, MacOS обеспечивается использование <<тонкого клиента>> и ограничена только поддержкой данными системами версий браузеров.
Разрабатываемое программное средство будет бесплатным для использования без ограничений на количество обрабатываемых и хранимых файлов.