\sectioncentered*{Общая характеристика работы}
\addcontentsline{toc}{section}{ОБЩАЯ ХАРАКТЕРИСТИКА РАБОТЫ}
\label{sec:general_overview}

\newcommand{\totpages}{\number\numexpr\getpagerefnumber{LastPage}}

\textbf{Цель и задачи исследования}
\bigskip

Целью данной работы состоит в исследовании подходов анализа психологического и медицинского состояния человека и информационной безопасности на основе образцов почерка и создании средства автоматизации подходов.

Задачами исследования являются:
\begin{enumerate}
  \item изучить научно-методическую и справочную литературу по вопросу графологии и графологических методов;
  \item провести сравнительный анализ существующие подходу к автоматизации выделения признаков почерка;
  \item обосновать практическую пользу применения исследуемого \mbox{предмета};
  \item определить требования в программному средству;
  \item разработать программное средство автоматизации биометрической аутентификации по образцу почерка;
  \item разработать программное средство автоматизации определения неврологических отклонений по образцу почерка;
  \item разработать программное средство автоматизации определения психологических характеристик личности по образцу почерка.
\end{enumerate}

Объектом исследования диссертационной работы является образцы \mbox{почерка}.

Предметом исследования является автоматизация методов анализа образцов почерка на основе траектории линий.

Основной гипотезой, положенной в основу диссертационной работы, является возможность использования компьютеров общего назначения для задач ввода, обработки и анализа образцов почерка, для проведения биометрической аутентификации, определения неврологических отклонений и определения психологических характеристик личности. 

\bigskip
\textbf{Личный вклад соискателя}
\bigskip

Результаты, приведенные в диссертации, получены  соискателем лично. Вклад научного руководителя А. В. Хмелевой, заключается в формулировке целей и задач исследования.

\bigskip
\textbf{Опубликованность результатов диссертации}
\bigskip

По теме диссертации опубликовано 3 печатных работ, из них 1 статья в рецензируемом издании, 2 работы в сборниках трудов и материалов международных конференций.

\bigskip
\textbf{Структура и объем диссертации}
\bigskip

Диссертация состоит из общей характеристики работы, введения, пяти глав, заключения, списка использованных источников, списка публикаций автора и приложения. В первой главе представлен обзор предметной области, выявлены основные существующие подходы, методы и алгоритмы в рамках тематики исследования, а так же выявлены проблемы и недоставки существующих программных средств. Производится выбор и обоснование алгоритмов, средств разработки, языка программирования, набора прикладных библиотек. Вторая глава посвящена формированию требований к будущему программному средству, на основание результатов полученных в прошлой главе. В третьей главе производится описание архитектуры разрабатываемого программного средства, описывается структура модулей программного средства, определяется список признаков почерка и особенности реализации алгоритмов выбранных в первой главе для анализа неврологических отклонений, определения психологических характеристик и биометрической аутентификации. В четвертой главе описана практическая реализация программного средства анализа почерка на основе траектории линий в психологии, медицине и информационной безопасности. В пятой главе представлены результаты исследования.

Общий объем работы составляет \totpages~страниц, из которых основного текста - 54 страницы, \totfig{}~рисунков, на 9 страницах, \tottab{}~таблицы на 2 страницах, список использованных источников из \totref{}~наименования на 6 страницах и 1 приложение на 6 страницах.