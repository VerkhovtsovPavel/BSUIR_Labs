\sectioncentered*{Вывод}
\addcontentsline{toc}{section}{Вывод}
\label{sec:practice_outro}

В ходе практике были проанализированны существующие аналоги разрабатываемого программного средства и изучены литературные источники, на основании анализа были составлены и специфицированны требования к разрабатываемому ПС. В качестве метода сегментации текста используется статический метод так он дает достойный результат и обладает хорошими временными характеристиками, а в качестве метода классификации используется метод опорных векторов из-за быстрого времени обучения и распознавания и наличия достаточной обучающей выборки для его использования. На основании требований к программному средству в качестве основного языка программирования был выбран Scala, как современный, мультипарадигмальный, компилируемый язык что обеспечит высокую производительность вместе с выразительностью и компактностью кода. Так же будут использоваться фреймворк Akka для реализации параллельной обработки данных и фреймворк Spray для реализации веб-сервера, данные фреймворки были выбраны из-за использования неблокирующих операций и модели акторов что обеспечивает общий подход к разработке и упрощает сопровождение. На основании спецификации требований и используемых технологии была разработана обобщенная архитектура будущего программного средства состоящая из моделя выделения признаков почерка, модуля определения параметров личности, модуля контроля доступа и модуля доступа к базе данных.