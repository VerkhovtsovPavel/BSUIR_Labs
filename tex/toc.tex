% Зачем: Содержание пишется полужирным шрифтом, по центру всеми заглавными буквами
% Почему: Пункт 2.2.7 Требований по оформлению пояснительной записки.
\renewcommand \contentsname {\centerline{\bfseries\large{\MakeUppercase{содержание}}}}

% Зачем: Не захламлять основной файл
% Примечание: \small\selectfont злостный хак, чтобы уменьшить размер шрифта в ToC
{
\normalsize\selectfont
\tableofcontents
\newpage
}

\todo[inline,color=green]{Заменить везде фразу проект на работы}
\todo[inline,color=green]{Найти `'jpeg`', `'изображение`', 'пиксель', 'яркость' \ldots и удалить либо заменить на форматы используемые в магистерской}
\todo[inline,color=green]{Провести авто-проверку орфографии}
\todo[inline,color=green]{Исключить части про сегментацию}
\todo[inline,color=green]{Исключить части про Tesseract}
\todo[inline,color=green]{Добавить новые литературные источники из new-litsources.txt}