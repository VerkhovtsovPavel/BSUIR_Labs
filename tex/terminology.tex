\sectioncentered*{Определения и сокращения}

В представленной пояснительной записке используются следующие определения и сокращения:

Аутентификация -- процедура проверки подлинности идентификатора, предъявленного сущностью для получения доступа к ресурсу.

ПО -- программное обеспечение

ПС -- программное средство

ОС -- операционная система

ООП -- объектно-ориентированное программирование

ФП -- функциональное программирование

БД -- база данных

СУБД -- Система управления базами данных

JVM -- Java Virtual Machine (виртуальная машина для языка Java)

LLVM -- Low Level Virtual Machine (универсальная система анализа, трансформации и оптимизации программ, реализующая виртуальную машину с RISC-подобными инструкциями)

API -- Application Programming Interface (интерфейс прикладного программирования, интерфейс программирования приложений)

JSON -- JavaScript Object Notation (текстовый формат хранения и передачи данных, основанный на языке программирования JavaScript)

JWT -- JSON Web Token (открытый стандарт создания и верификации маркеров доступа)

HOCON -- Human-Optimized Config Object Notation (формат конфигурационных файлов ориентированный на человеко-читаемость, основанный на JSON)

Sbt -- Simple Build Tool (система автоматической сборки для проектов, написанных на языках Scala, Java и С++)

Plug-and-Play -- технология, предназначенная для быстрого определения и конфигурации устройств в компьютере и других технических устройствах.

PKI -- Public Key Infrastructure (набор средств, распределённых служб и компонентов, в совокупности используемых для поддержки криптографических задач на основе закрытого и открытого ключей)

ACID -- Atomicity Consistency Isolation Durability  (описывает требования к транзакционной системе например, к СУБД, обеспечивающие наиболее надёжную и предсказуемую её работу)

\clearpage