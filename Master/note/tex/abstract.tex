\begin{center}
{\bfseries ОБЩАЯ ХАРАКТЕРИСТИКА РАБОТЫ}
\end{center}

\textbf{Цель и задачи исследования}
\bigskip

Целью данной работы является исследование подходов анализа психологического и медицинского состояния человека и информационной безопасности на основе образцов почерка и создание средства автоматизации подходов.

Задачами исследования являются:
\begin{enumerate}
  \item изучить научно-методическую и справочную литературу по вопросу графологии и графологических методов;
  \item провести сравнительный анализ существующих подходу к автоматизации выделения признаков почерка;
  \item обосновать практическую пользу применения исследуемого \mbox{предмета};
  \item определить требования в программному средству;
  \item разработать программное средство автоматизации биометрической аутентификации по образцу почерка;
  \item разработать программное средство автоматизации определения неврологических отклонений по образцу почерка;
  \item разработать программное средство автоматизации определения психологических характеристик личности по образцу почерка.
\end{enumerate}

Объектом исследования диссертационной работы являются образцы \mbox{почерка.}

Предметом исследования является автоматизация методов анализа образцов почерка на основе траектории линий.

Основной гипотезой, положенной в основу диссертационной работы, является возможность использования компьютеров общего назначения для задач ввода, обработки и анализа образцов почерка, для проведения биометрической аутентификации, определения неврологических отклонений и определения психологических характеристик личности. 

\bigskip
\textbf{Личный вклад соискателя}
\bigskip

Результаты, приведенные в диссертации, получены  соискателем лично. Вклад научного руководителя А. В. Хмелевой,  заключается в формулировке целей и задач исследования.

\bigskip
\textbf{Опубликованность результатов диссертации}
\bigskip

По теме диссертации опубликовано 3 печатные работы, из них 1 статья в рецензируемом издании, 2 работы в сборниках трудов и материалов международных конференций.

\bigskip
\textbf{Связь работы с приоритетными направлениями научных исследований и запросами реального сектора экономики}
\bigskip

Работа выполнялась в соответствии с научно-техническим заданием и планом работ кафедры «Программное обеспечение информационных технологий» по теме  «Разработка моделей, методов, алгоритмов, повышающих показатели проектирования, внедрения и  эксплуатации программных средств для перспективных платформ обработки информации, решения интеллектуальных задач,  работы с большими массивами данных и внедрение в современные обучающие комплексы» (ГБ № 16-2004, № ГР 20163588, научный руководитель НИР – Н. В. Лапицкая).

\bigskip
\textbf{Структура и объем диссертации}
\bigskip

Диссертация состоит из общей характеристики работы, введения, пяти глав, заключения, списка использованных источников, списка публикаций автора и приложения. В первой главе представлен обзор предметной области, выявлены основные существующие подходы, методы и алгоритмы в рамках тематики исследования, а так же выявлены проблемы и недоставки существующих программных средств. Производится выбор и обоснование алгоритмов, средств разработки, языка программирования, набора прикладных библиотек. Вторая глава посвящена формированию требований к будущему программному средству, на основание результатов полученных в первой главе. В третьей главе производится описание архитектуры разрабатываемого программного средства, описывается структура модулей программного средства, определяется список признаков почерка и особенности реализации алгоритмов, выбранных в первой главе для анализа неврологических отклонений, определения психологических характеристик и биометрической аутентификации. В четвертой главе описана практическая реализация программного средства анализа почерка на основе траектории линий в психологии, медицине и информационной безопасности. В пятой главе представлены результаты исследования.

Общий объем работы составляет 77~страниц, из которых основного текста - 54 страницы, 25~рисунков на 9 страницах, 2~таблицы на 2 страницах, список использованных источников из 75~наименований на 6 страницах и 1 приложение на 6 страницах.

\begin{center}
{\bfseries ОСНОВНОЕ СОДЕРЖАНИЕ}
\end{center}

Во \textbf{введении} определена область и указаны основные направления исследования, показана актуальность темы диссертационной работы, дана краткая характеристика исследуемых вопросов, обозначена практическая ценность работы.

В \textbf{первой главе} производится обзор литературных источников. Проанализированы преимущества и недостатки различных подходов к выделению, обработке и классификации признаков рукописного текста, а также программных средств, аналогичных разрабатываемому соискателем в рамках данной работы. В результате анализа было принято решение использовать метод опорных векторов для классификации характеристик личности на основе признаков почерка~[1], статические методы для сегментации образцов на строки, слова и символы, статистические методы на основе радиально-базисной функции Гаусса для задач биометрической аутентификации~[2] и среднеквадратичное отклонение от скелета образца для определения неврологических отклонений~[3].

\textbf{Вторая глава} посвящена анализу требований к разрабатываемому программному средству, основываясь на данных, полученных в первой главе. В результате анализа были выявлены следующие функции программного средства:
\begin{itemize}
	\item регистрация пользователя;
	\item авторизация пользователя;
	\item просмотр сохраненных образцов почерка;
	\item удаление сохраненных образцов почерка;
	\item добавление нового образца почерка;
	\item выделение признаков образца почерка;
	\item определение психологических характеристик личности;
	\item биометрическая аутентификации пользователя;
	\item определение неврологических отклонений.
\end{itemize}

Приведенный набор функций позволяет полностью соответстветствовать задачам работы и современным стандартам разработки программного обеспечения.

В \textbf{третьей главе} описана архитектура программного средства, определен набор модулей программного средства:
\begin{itemize}
    \item модуль выделения признаков почерка;
    \item модуль определения характеристик личности;
    \item модуль биометрической аутентификации;
    \item модуль определения неврологических отклонений;
    \item модуль контроля доступа;
    \item модуль доступа к базе данных.
\end{itemize}

Так, в данной главе приводится исчерпывающий набор признаков рукописного текста, достаточный для проведения всех трех видов анализа.

В \textbf{четвертой главе} приведена реализация программного средства. Приведена схема классификатора и описана структура обучающей выборки. Описаны основные классы и их взаимодействие. Приведена схема базы данных и примеры JSON-объектов, передаваемых между сервером и клиентом.

В \textbf{пятой главе} производится анализ результатов, полученных в процессе выполнения работы, оцениваются целесообразности выбора архитектурных и алгоритмических решений. Приводятся численные результаты оценки работы программы и их обоснование.

Результатом диссертационной работы стало программное средство, реализующее операции выделения признаков рукописного текста, определения психологические характеристики личности, определение неврологических отклонений и биометрической аутентификации.

\clearpage

\begin{center}
{\bfseries ЗАКЛЮЧЕНИЕ}
\end{center}

\textbf{Основные научные результаты диссертации}
\bigskip

1. В ходе работы был установлен исчерпывающий набор признаков почерка, позволяющий производить независимый анализ в каждой из трех областей.

2. В ходе работы были проанализированны особенности распределения каждого признака почерка, как при их написании одним авторам, так и разными авторами, что позволило расчитать коэффициенты признаков почерка.

3. В ходе работы был апробирован ряд решений в области классификации признаков, в частности, исследовалось оптимальное количество классов, набор признаков, созависимость признаков в наборе. Также по ряду признаков проводился анализ оптимального подхода к формированию входных данных классификатора.

\bigskip
\textbf{Элементы научной новизны}
\bigskip

1. Использование категории вместо непрерывного значения для ряда признаков почерка позволило повысить качество классификации, и как следствие, увеличить точность определения психологических характеристик.

2. Совместное использование среднеквадратичного отклонения от скелета образца почерка для выявления тремора и набора признаков почерка для выявления проявлений микрографии позволило повысить точность определения неврологических отклонений.

3. Решение на основе радиально-базисной функции Гаусса и набора коэффициентов, расчитанных на основе распределений признаков почерка для одного и разных авторов, для задач биометрической аутентификации было аппробированно и показало свою состоятельность.

\bigskip
\textbf{Рекомендации по практическому использованию результатов}
\bigskip

1. Полученные результаты формируют теоретическую и практическую базу для разработки программного обеспечения анализа почерка на основе траектории линий в психологии, медицине и информационной безопасности.

2. Полученные результаты могут быть использованы для дальнейшего развития существующих систем.

3. Результаты работы могут использоваться отделами кадров на предприятиях различной формы собственности для оценки психологических характеристик претендентов на должность.

4. Результаты работы могут применяться пользователя (частными лицами) для самостоятельной диагностики неврологических отклонений и принятии решения о посещении специалиста невролога.

\clearpage

\begin{center}
{\bfseries СПИСОК ОПУБЛИКОВАННЫХ РАБОТ}
\end{center}

1. Верховцов, П.А. Методы и подходы автоматизации графологического анализа / П.А. Верховцов // Актуальные направления научных исследований XXI века: теория и практика. Т. 5. № 10 (36). / Воронежский государственный лесотехнический университет им. Г.Ф. Морозова. – Воронеж, 2017. – С. 99-102.

2. Верховцов, П.А. Аутентификация на основе рукописной подписи / П.А. Верховцов // Актуальные направления научных исследований XXI века: теория и практика / Воронежский государственный лесотехнический университет им. Г.Ф. Морозова. – Воронеж, 2018.

3. Верховцов, П.А. Диагностика неврологических заболеваний на основе образцов почерка / П. А. Верховцов // Компьютерные системы и сети: материалы 54-й научной конференции аспирантов, магистрантов и студентов, Минск, 23 – 27 апреля 2018 г. / Белорусский государственный университет информатики и радиоэлектроники. – Минск, 2018. – С. 54 – 55.

\clearpage