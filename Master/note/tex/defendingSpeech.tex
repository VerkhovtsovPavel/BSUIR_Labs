\small\selectfont
{\large\bfseries Речь}

\textbf{Введение}
\bigskip

Применение информационных систем в коммерческих компании стало необходимым условием выживания на рынке, а для социальных и государственных организаций и частных некоммерческих проектов использование информационных систем является единственным способом предоставить уровень сервиса соответствующий современным стандартам. 

При глобальном применении информационных технологий встает вопрос об организации и разграничении доступа к информации обрабатываемой системой, так как часть информации может представлять коммерческую тайну или содержать персональные данные.

На фоне глобального применения информационных технологий вопросы подбора квалифицированного и исполнительного персонала встают как никогда остро, так как все чаще возникают ситуации когда человек является самым медленным и ненадежным элементов производственной цепочки, а ошибки могут обернуться для компании не только потерями прибыли, но репутации.

Графология и почерковедение – это дисциплины, утверждающии о наличии связи между почерком и индивидуальными особенностями личности. Оперирую такими признаками как наклон, интервал между строками, форма полей эксперты могут установить психологическую устойчивость, склонность к агрессии, коммуникабельность и другие качества интересные работодателю.

Биометрическая аутентификация -- это процедура сбора биометрического параметра с последующей возможностью предъявления пользователем своего уникального биометрического параметра и процесс сравнения его со всей базой имеющихся данных, с целью установления личности или причастности к группе лиц.

Микрография -- это нарушение письма выражающееся в нечеткости контуров и уменьшение букв. Наблюдается при паркинсоническом синдроме или депрессии, иногда при кататонических состояниях.

\textbf{Сегментация}
\bigskip

Перед началом сегментации изображения необходимо выполнить его предварительную обработку (бинаризация, удаление шумов);
Алгоритм сегментации реализованный в проекте является параллельным(\emph{Алгоритм сегментации}), что позволило значительно уменьшите время сегментации. 

Модель акторов - модель параллельных вычислений, основанная на взаимодействии изолированных примитивов, взаимодействующих по средствам получения и отправки сообщений. 

Благодаря отсутствую прямое взаимодействия акторов между собой, т.к. вся коммуникация осуществляется при помощи передачи сообщений, становится возможным полностью избежать блокировок потоков исполнения. В совокупности с использованием неизменяемых структур данных, механизм сообщений делает всю работу системы акторов априорно асинхронной и неблокирующей. Позволяет добиться прироста производительности сопоставимого с количеством ядер процессоров в среде исполнения, чего невозможно добиться при использовании классической модели параллельности на основе потоков и блокировках при доступе к общему изменяемому состоянию.

\textbf{Выделение признаков}
\bigskip

Следующим этапом работы программ является выделения признаков почерка.

Программное средство оперирует следующими признаками:
\begin{itemize}
  \item длительность написания;  
  \item количество линий;
  \item длина по горизонтали;
  \item длина по вертикали;
  \item площадь;
  \item общая длина;
  \item максимальное ускорение;
  \item минимальное ускорение;
  \item длительность написание по вертикали;
  \item длительность написание по горизонтали;
  \item наклон символов;
  \item наклон строк;
  \item интервал между символами;
  \item интервал между словами;
  \item интервал между строками;
  \item частота текста.
\end{itemize}

Что позволяет получить разностороннюю информацию для анализа.
Так же стоит отметить что для ускорения обработки образца, начало выделение признака начинается как-только готовы все исходные данные(\emph{Модуль выделения признаков почерка}), например определение интервала между строками начинается сразу после сегментации изображения на строки.

Результаты представлены на \emph{Результаты работы - 1.1}

\textbf{Параметры личности}
\bigskip

Следующий этапом работы является определение характеристик личности.

Всего определяется 3 характеристики
\begin{itemize}
  \item темперамент;
  \item лидерские качества;
  \item работоспособность.
\end{itemize}

Всего 16 классов образцов.

Для классификации использоваться классификатор на основе метода опорных векторов.

Обучающая выборка представляет собой примерно 1500 образцов почерка от 500 авторов.
Распределения образцов выборки по классам представлено на (\emph{Результаты работы - 1.2})

Все классы представлены в выборке примерно в разных пропорциях.

\textbf{Аутентификация}
\bigskip

Достойный диапозон работы (\emph{Результаты работы - 2.1})

\textbf{Основные научные результаты диссертации}
\bigskip

1. В ходе работы был установлен исчерпывающий набор признаков почерка, позволяющий производить независимый анализ в каждой из трех областей.

2. В ходе работы были проанализированны особенности распределения каждого признака почерка, как при их написании одним авторам, так и разными авторами, что позволило расчитать коэффициенты признаков почерка.

3. В ходе работы был апробирован ряд решений в области классификации признаков, в частности, исследовалось оптимальное количество классов, набор признаков, созависимость признаков в наборе. Также по ряду признаков проводился анализ оптимального подхода к формированию входных данных классификатора.

\textbf{Элементы научной новизны}
\bigskip

1. Использование категории вместо непрерывного значения для ряда признаков почерка позволило повысить качество классификации, и как следствие, увеличить точность определения психологических характеристик.

2. Совместное использование среднеквадратичного отклонения от скелета образца почерка для выявления тремора и набора признаков почерка для выявления проявлений микрографии позволило повысить точность определения неврологических отклонений.

3. Решение на основе радиально-базисной функции Гаусса и набора коэффициентов, расчитанных на основе распределений признаков почерка для одного и разных авторов, для задач биометрической аутентификации было аппробированно и показало свою состоятельность.

\textbf{Рекомендации по практическому использованию результатов}
\bigskip

1. Полученные результаты формируют теоретическую и практическую базу для разработки программного обеспечения анализа почерка на основе траектории линий в психологии, медицине и информационной безопасности.

2. Полученные результаты могут быть использованы для дальнейшего развития существующих систем.

3. Результаты работы могут использоваться отделами кадров на предприятиях различной формы собственности для оценки психологических характеристик претендентов на должность.

4. Результаты работы могут применяться пользователя (частными лицами) для самостоятельной диагностики неврологических отклонений и принятии решения о посещении специалиста невролога.