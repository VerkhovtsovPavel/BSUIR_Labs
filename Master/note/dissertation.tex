\input{tex/preamble}
\begin{document}

% Не пытаемся впихивать по максимуму - не получаем вылазящих за правый край слов
\sloppy

% Титульник
\begin{titlepage}
  \begin{center}
    Министерство образования Республики Беларусь\\[1em]
    Учреждение образования\\
    БЕЛОРУССКИЙ ГОСУДАРСТВЕННЫЙ УНИВЕРСИТЕТ \\
    ИНФОРМАТИКИ И РАДИОЭЛЕКТРОНИКИ\\[1em]

    \begin{flushleft}
        Факультет компьютерных систем и сетей\\
        Кафедра программного обеспечения информационных технологий
    \end{flushleft}

    \begin{flushright}
      \begin{minipage}{0.4\textwidth}
        \textit{К защите допустить:}\\[0.8 em]
        Заведующая кафедрой ПОИТ\\[0.45 em]
        \underline{\hspace*{2.8 cm}} Н.\,В.~Лапицкая
      \end{minipage}\\[2.2 em]
    \end{flushright}

    {ПОЯСНИТЕЛЬНАЯ ЗАПИСКА}\\
    {к дипломному проекту}\\
    {на тему}\\[1em]
    \textbf{ПРОГРАММНОЕ СРЕДСТВО ОПРЕДЕЛЕНИЯ ПСИХОЛОГИЧЕСКИХ ХАРАКТЕРИСТИК ПО АНАЛИЗУ ОБРАЗЦОВ ПОЧЕРКА}\\[1em]


    {БГУИР ДП 1-40 01 01 03 021 ПЗ}\\[2em]

    \begin{tabular}{ p{0.65\textwidth}p{0.25\textwidth} }
      Студент & П.\,А.~Верховцов \\
      Руководитель & А.\,В.~Хмелева \\
      Консультанты: &\\
      \hspace*{3 ex}\emph{от кафедры ПОИТ} & А.\,В.~Хмелева \\
      \hspace*{3 ex}\emph{по экономической части} & Т.\,А.~Рыковская \\
      Нормоконтролёр & С.\,В.~Болтак\\
      & \\
      Рецензент &
    \end{tabular}

    \vfill
    {\normalsize Минск 2017}
  \end{center}
\end{titlepage}
 %

% Содержание
\clearpage \setcounter{page}{2}
% Зачем: Содержание пишется полужирным шрифтом, по центру всеми заглавными буквами
% Почему: Пункт 2.2.7 Требований по оформлению пояснительной записки.
\renewcommand \contentsname {\centerline{\bfseries\large{\MakeUppercase{содержание}}}}

% Зачем: Не захламлять основной файл
% Примечание: \small\selectfont злостный хак, чтобы уменьшить размер шрифта в ToC
{
\normalsize\selectfont
\tableofcontents
\newpage
}

\todo[inline,color=green]{Заменить везде фразу проект на работы}
\todo[inline,color=green]{Найти `'jpeg`', `'изображение`', 'пиксель', 'яркость' \ldots и удалить либо заменить на форматы используемые в магистерской}
\todo[inline,color=green]{Провести авто-проверку орфографии}
\todo[inline,color=green]{Исключить части про сегментацию}
\todo[inline,color=green]{Исключить части про Tesseract}
\todo[inline,color=green]{Добавить новые литературные источники из new-litsources.txt} %

% Определения и сокращения
\sectioncentered*{Определения и сокращения}

В представленной пояснительной записке используются следующие определения и сокращения:

Аутентификация -- процедура проверки подлинности идентификатора, предъявленного сущностью для получения доступа к ресурсу.

ПО -- программное обеспечение

ПС – программное средство

ОС -- операционная система

ООП -- объектно-ориентированное программирование

ФП -- функциональное программирование

БД -- база данных

СУБД -- Система управления базами данных

API -- Application Programming Interface (интерфейс прикладного программирования, интерфейс программирования приложений)

JSON -- JavaScript Object Notation (текстовый формат хранения и передачи данных, основанный на языке программирования JavaScript)

JWT -- JSON Web Token (открытый стандарт создания и верификации маркеров доступа)

HOCON -- Human-Optimized Config Object Notation (формат конфигурационных файлов ориентированный на человеко-читаемость, основанный на JSON)

\clearpage %

% Общая характеристика работы
\sectioncentered*{Общая характеристика работы}
\addcontentsline{toc}{section}{Общая характеристика работы}
\label{sec:general_overview}

\newcommand{\totpages}{\number\numexpr\getpagerefnumber{LastPage}}

\textbf{Цель и задачи исследования}
\bigskip

Целью данной работы состоит в исследовании подходов анализа психологического и медицинского состояния человека и информационной безопасности на основе образцов почерка и создании средства автоматизации подходов.

Объектом исследования диссертационной работы является образцы почерка.

Предметом исследования является автоматизация методов анализа образцов почерка на основе траектории линий.

Задачами исследования являются:
\begin{enumerate}
  \item изучить научно-методическую и справочную литературу по вопросу графологии и графологических методов;
  \item провести сравнительный анализ существующие подходу к автоматизации выделения признаков почерка;
  \item обосновать практическую пользу применения исследуемого \mbox{предмета};
  \item определить требования в программному средству;
  \item разработать программное средство автоматизации биометрической аутентификации по образцу почерка;
  \item разработать программное средство автоматизации определения неврологических отклонений по образцу почерка;
  \item разработать программное средство автоматизации определения психологических характеристик личности по образцу почерка.
\end{enumerate}
Объектом исследования диссертационной работы является образцы почерка.

Предметом исследования является автоматизация методов анализа образцов почерка на основе траектории линий.

Основной гипотезой, положенной в основу диссертационной работы, является возможность использования компьютеров общего назначения для задач ввода, обработки и анализа образцов почерка, для проведения биометрической аутентификации, определения неврологических отклонений и определения психологических характеристик личности. 

\bigskip
\textbf{Личный вклад соискателя}
\bigskip

Результаты, приведенные в диссертации, получены  соискателем лично. Вклад научного руководителя А. В. Хмелевой, заключается в формулировке целей и задач исследования.

\bigskip
\textbf{Опубликованность результатов диссертации}
\bigskip

По теме диссертации опубликовано 3 печатных работ, из них 1 статья в рецензируемом издании, 2 работы в сборниках трудов и материалов международных конференций.

\bigskip
\textbf{Структура и объем диссертации}
\bigskip

Диссертация состоит из общей характеристики работы, введения, пяти глав, заключения, списка использованных источников, списка публикаций автора и приложения. В первой главе представлен обзор предметной области, выявлены основные существующие подходы, методы и алгоритмы в рамках тематики исследования, а так же выявлены проблемы и недоставки существующих программных средств. Производится выбор и обоснование алгоритмов, средств разработки, языка программирования, набора прикладных библиотек. Вторая глава посвящена формированию требований к будущему программному средству, на основание результатов полученных в прошлой главе. В третьей главе производится описание архитектуры разрабатываемого программного средства, описывается структура модулей программного средства, определяется список признаков почерка и особенности реализации алгоритмов выбранных в первой главе для анализа неврологических отклонений, определения психологических характеристик и биометрической аутентификации. В четвертой главе описана практическая реализация программного средства анализа почерка на основе траектории линий в психологии, медицине и информационной безопасности. В пятой главе представлены результаты исследования.
Общий объем работы составляет \totpages~страниц, из которых основного текста - 54 страницы, \totfig{}~рисунков, на 8 страниц, \tottab{}~таблиц на 2 страницы, список использованных источников из \totref{}~наименования на 7 страниц и 1 приложение на 6 страниц. %

% Введение
\sectioncentered*{Введение}
\addcontentsline{toc}{section}{Введение}
\label{sec:intro}

В современном мире наблюдается повсеместное ускорение производство, улучшение качесва товаров и услуг засчет внедрения информационных систем учета и анализа различной степени сложности.

Применение информационных систем в коммерческих компании стало необходимым условием выживания на рынке, а для социальных и государственных организаций и частных некоммерческих проектов использование информационных систем является единственным способом предоставить уровень сервиса соответствующий современным стандартам. 

При глобальном примениние информационых технологий встает вопрос об огранизации и разграничении доступа к информации обработываемой системой, так как часть информации может представлять коммерческую тайну или содержать персональные данные.

На фоне глобальном примениние информационых технологий вопросы подбора квалифицированного и исполнительного персонала встают как никогда остро, так как все чаще возникают ситуации когда человек является самым медленным и ненадежным элементов производственной цепочки, а ошибки могут обернуться для компании не только потерями прибыли, но репутации.

Графология -- это дисциплина, утверждающая о наличии связи между почерком и индивидуальными особенностями личности. Оперирую такими признаками как наклон, интервал между строками, форма полей графологи могут установить психологическую устойчивость, склонность к агрессии, коммуникабельность и другие качества интересные работодателю.

Биометрическая аутентификация -- это процедура сбора биометрического параметра с последующией возможностью предъявления пользователем своего уникального биометрического параметра и процесс сравнения его со всей базой имеющихся данных, с целью установления личности или причастности к группе лиц.

Микрография -- это нарушение письма выражающееся в нечеткости контуров и уменьшение букв. Наблюдается при паркинсоническом синдроме или депрессии, иногда при кататонических состояниях.

Целью данной работы является разработка программного средства анализа образцов почерка на основе анализа траектории линии.
Объектом исследования диссертационной работы является набор характеристик почерка достаточный для проведения графологического анализа, биометрической аутентификации и определения неврологических заболеваний.
Предметом исследования является автоматизация методов графологии, биометрической аутентификации и определения неврологических заболеваний.

Задачами исследования являются:
\begin{enumerate}
  \item изучить научно-методическую и справочную литературу по вопросу графологии и графологических методов;
  \item провести сравнительный анализ существующие подходу к автоматизации выделения признаков почерка;
  \item обосновать практическую пользу применения исследуемого \mbox{предмета};
  \item определить требования в программному средству;
  \item разработать программное средство автоматизации биометрической аутентификации по образцу почерка;
  \item разработать программное средство автоматизации определения неврологических отклонений по образцу почерка;
  \item разработать программное средство автоматизации определения психологических характеристик личности по образцу почерка.
\end{enumerate}

В рамках данной работы рассматриваются возможность автоматизации методов графологии, биометрической аутентификации и анализа неврологических заболевания на основе образцов почерка для определения психологических характеристик человека, вероятных неврологических заболеваний, а так же проведение аутентификации.
Разрабатываемое программное средство может показать разнообразные способы использования набора параметров почерка в целях определения характеристик личности, биометрической аутентификации и определения неврологических заболеваний. %

% Глава 1 Обзор предметной области
\section{Обзор предметной области}
\label{sec:domain:intro}

В данном разделе будет произведён обзор программных средств аналогичных разрабатываемому в рамках дипломного проекта - определение психологических параметров человека по образцу почерка, а так же литературных источников. Проанализированы преимущества и недостатки различных подходов к выделению и классификации признаков рукописного текста.

\subsection{Графология}
\label{sub:domain:grafologic}
\emph{Графология} - это учение, постулирующее наличие устойчивой связи месту почерком и индивидуальными особенностями личности.

Идея использования почерка для выявления психологических параметров личности впервые была предложена в 1622 в книге итальянского профессора Камилло Бальдо <<Как узнать природу и качества человека, взглянув на букву, которую он написал>> ~\cite{kamillo_grafology}. Первым кто систематизировал знания стал Фландрэна аббат Мишон в 1872 году. Он проанализировал большое количество работ по графологии и образцов почерка и в своей книге <<Система графологии>> предложил \emph{метод Мишона}, он основывался на анализе штрихов, букв, слов, свободных движений, строк и пр.~\cite{mishon_grafology}

Начиная с середины 20 века графология начала рассматриваться как псевдонаучное учение~\cite{graphology_wiki}. По результатам исследования профессиональным графологам не удалось достоверно оценить трудовые способности человека. В среднем профессиональные графологи давали такую же по степени достоверности оценку, как и люди «с улицы»~\cite{neter_shakhar_psevdograph}~\cite{king_koehler_psevdograph}. В десятках исследований было показано отсутствие связи особенностей почерка с трудовыми способностями человека.

Тем не менее графология широко используется в современной практике отбора кадров~\cite{graphology_psyfactor}.

Основные признаки почерка, которые анализирует графологическая экспертиза:
\begin{enumerate}
  \item размер букв (очень маленькие, маленькие, средние, крупные);
  \item наклон букв (левый наклон, легкий наклон влево, правый наклон, резкий наклон вправо);
  \item направление почерка: (строчки ползут вверх, строчки прямые,  строчки ползут вниз);
  \item размашистость и сила нажима: (легкая, средняя, сильная, очень сильная);
  \item характер написания слов (склонность к соединению букв и слов, склонность к отдалению букв друг от друга, смешанный стиль);
  \item общая оценка (почерк старательный, почерк неровный, почерк небрежный, почерк неразборчивый).
\end{enumerate}

Перечисленные параметры почерка являются устойчивыми, но все же присутствует естественные отклонения параметров (длина, ширина, толщина, угол) от средних значений. Вариация становится наиболее заметной при изменение психологического состояния человека, например при страхе, беспокойстве, алкогольном опьянении.

\subsection{Анализ аналогов}
\label{sub:domain:analogs}

\subsubsection{ScriptAlyzeR}
\label{sub:domain:analogs:neuro_script} 

Программное средство <<ScriptAlyzeR>> является частью семейства программных средств для работы с рукописным текстов компании <<NeuroScript>> и представляет собой декстопное приложение для операционных систем Windows~\cite{analogs_scriptAlyzer}.

Основными возможностями ПС являются:
\begin{itemize}
  \item отслеживание положения, давления, ориентации с частотой 100-200 Гц;
	\item поддержка отслеживания руки, стилуса и мыши;
	\item измеряйте координацию одновременно две рук;
	\item отображение результатов в реальном времени;
	\item изменение толщины линии. Визуальная и звуковая обратная связь;
	\item искажение визуальная обратная связь. Поворот, перекос и отражение на мониторе компьютера в режиме реального времени;
	\item моделирование. Генерация рукописных цифровых данных с шумом и известными характеристиками штрихов;
	\item проверка непротиворечивости.
	\item анализ результатов. Статистика результатов с визуализацией;
	\item многостраничная записи. Разделение текст на слова и штрихи;
	\item внешние приложения. Полная интеграция с вашими собственными модулями с использованием сценариев MATLAB® или скомпилированных программ;
	\item оптически сканированные изображения.;
\end{itemize}

Основными недостатками ПС являются:
\begin{itemize}
  \item Поддержка только ОС семейства Windows (XP, 7, 8)
  \item Платное использование (799+ US\$)
\end{itemize}

Согласно утверждениям разработчиков ПС может быть использовано для оценки моторных функций, диагностики неврологических отклонений, а так же тестирования на состояние алкогольного опьянения.

\begin{figure}[ht]
    \centering
    \label{fig:domain:analogs:neuro_script}
    \includegraphics[width=0.7\textwidth]{figures/neuroscript.png}
    \caption{Программное средство <<ScriptAlyzeR>>}
\end{figure}

\subsubsection{Graphology}
\label{sub:domain:analogs:graphology} 

Программное средство <<Graphology>> является приложение для операционной системы Android разработанным компанией <<LH Apps>>~\cite{analogs_graphology}.

Программное средство <<Graphology>> предназначена для анализа почерка и определения характеристик личности. Алгоритм работы программы основан на обширных исследованиях и был создан при консультации профессиональных экспертов графологии.

Основными возможностями ПС являются:
\begin{itemize}

  \item Поддержка ОС Android;
  \item Многофакторная оценка параметров личности (почерк, подпись, рисунки);
  \item Выполнение анализа без доступа в интернет.
\end{itemize}

Основными недостатками ПС являются:
\begin{itemize}
  \item Поддержка только ОС Android;
  \item Поддержка только английского языка;
  \item Для ввода образцов почерка используется экран смартфона, что приводит к искажению в написании символов при низком разрешении и без использование стилуса.
\end{itemize}

\begin{figure}[ht]
    \centering
    \label{fig:domain:analogs:graphology}
    \includegraphics[width=0.7\textwidth]{figures/graphology_analog.jpeg}
    \caption{Программное средство <<Graphology>>}
\end{figure}

\subsubsection{Signature Analysis}
\label{sub:domain:analogs:signature_analysis} 

Программое средство <<Signature Analysis>> является приложение для операционной системы Android разработанным компанией <<Beyond Consultancy Services>>~\cite{analogs_signature_analysis}.

Программное средство <<Signature Analysis>> предназначена определения характеристик личности по образцу подписи. В разработке участвовал графолог с многолетним опытом, выступающей в качестве консультанта многих крупных компаний.

Основными возможностями ПС являются:
\begin{itemize}
  \item Поддержка ОС Android;
  \item Широкий спектр анализируемых параметров подписи (скорость, давление, длины, направления);
\end{itemize}

Основными недостатками ПС являются:
\begin{itemize}
  \item Поддержка только ОС Android;
  \item Платный анализ каждой подписи (0,83 US\$);
  \item Для работы необходимо интернет соединение;
  \item Для ввода образцов почерка используется экран смартфона, что приводит к искажению в написании символов при низком разрешении и без использование стилуса.
\end{itemize}

\begin{figure}[ht]{}
    \centering
    \label{fig:domain:analogs:signature_analysis}
    \includegraphics[height=0.5\textheight]{figures/analog_signature_analysis.png}
    \caption{Программное средство <<Signature Analysis>>}
\end{figure}

\subsubsection{My Graphology}
\label{sub:domain:analogs:my_graphology}

Программное средство <<My Graphology>> является приложение для операционной системы Android разработанным компанией <<PENS>>~\cite{analogs_my_graphology}.

Основными возможностями ПС являются:
\begin{itemize}
  \item Поддержка ОС Android;
  \item Использовать для ввода экран или фотографию почерка;
  \item Выполнение анализа без доступа в интернет.
\end{itemize}

Основными недостатками ПС являются:
\begin{itemize}
  \item Поддержка только ОС Android;
  \item В разработке не участвовали эксперты графологи;
  \item Поддержка только испанского языка интерфейса.
\end{itemize}

\begin{figure}[ht]
    \centering
    \label{fig:domain:analogs:my_graphology}
    \includegraphics[width=0.55\textwidth]{figures/analog_my_graphology.jpeg}
    \caption{Программное средство <<My Graphology>>}
\end{figure}

\subsubsection{GRAPHOLOGY signature analysis}
\label{sub:domain:analogs:graphology_sign_analysis}

Программное средство <<GRAPHOLOGY signature analysis>> является приложение для операционной системы Android разработанным компанией <<DokThor>>~\cite{analogs_graphology_sign_analysis}. Программное средство <<GRAPHOLOGY signature analysis>> предназначена определения характеристик личности по образцу подписи.

Основными возможностями ПС являются:
\begin{itemize}
  \item Поддержка ОС Android;
  \item Выполнение анализа без доступа в интернет;
  \item Предоставление характеристик по личности по 5 основным критериям.
\end{itemize}

Основными недостатками ПС являются:
\begin{itemize}
  \item Поддержка только ОС Android;
  \item В разработке не участвовали эксперты графологи;
  \item Механизм ввода подписи неочевиден;
  \item Для ввода образцов почерка используется экран смартфона, что приводит к искажению в написании символов при низком разрешении и без использование стилуса.
\end{itemize}

\begin{figure}[ht]
    \centering
    \label{fig:domain:analog:graphology_sign_analysis}
    \includegraphics[height=0.5\textheight]{figures/analog_graphology_sign_analysis.png}
    \caption{<<GRAPHOLOGY signature analysis>>}
\end{figure}

\subsubsection{Graphology Lite}
\label{sub:domain:analogs:graphology_lite}

Программное средство <<Graphology Lite>> является приложение для операционной системы Android разработанным компанией <<Hyperborea>>~\cite{analogs_graphology_sign_analysis}.

Основными возможностями ПС являются:
\begin{itemize}
  \item Поддержка ОС Android;
  \item Выполнение анализа без доступа в интернет;
  \item Использовать для ввода экран или фотографию почерка.
\end{itemize}

Основными недостатками ПС являются:
\begin{itemize}
  \item Поддержка только ОС Android;
  \item В разработке не участвовали эксперты графологи;
  \item Бесплатная версия позволяет произвести анализ только одного образца. Платная версия стоит 1.05 US\$.
\end{itemize}

\begin{figure}[ht]
    \centering
    \label{fig:domain:analogs:graphology_lite}
    \includegraphics[height=0.5\textheight]{figures/analog_graphology_lite.png}
    \caption{Программое средство <<Graphology Lite>>}
\end{figure}

\subsection{Анализ литературных источников}
\label{sub:domain:literary_sources}

Публикации и научные статьи на темы схожие с темой данной работы можно условно разделить по следующим признакам:
\begin{itemize}
  \item метод сегментации изображения;
  \item метод классификации признаков изображения.
\end{itemize}



\subsection{Постановка задачи}
В результате выполнения дипломного проекта должно быть разработано программное средство определения психологических параметров личности по образцу подчетка, реализующее процедуры выбеления признаков почерка из изображения и их классификацию, а так же механизм авторизации для обеспечения секретности данных. К разрабатываемому программному средству предъявляются следующие требования:
\begin{itemize}
\item разрабатываемое ПО должно работать на операционных системах Linux, MacOS и Windows;
\item программное средство должно быть выполнено в виде клиент"=серверного приложения;
\item программное средство должно поддерживать русской язык интерфейса;
\item программное средство должно поддерживать работу как в режиме выделения признаков рукописного текста, так и в режиме их классификации;
\item программное средство должно предусматривать механизм регистрации, аутентификации и авторизации пользователей.
\end{itemize}
 

% Глава 2 Функциональные требования
\section{Анализ требований к программному средству}
\label{sec:freq}
\subsection{Описание функциональности ПС}
Основываясь на требованиях изложенных в разделе \ref{sec:domain:requirements} и диаграмме вариантов использования, рисунок \ref{fig:freg:usecase}, разрабатываемое ПС должно выполнять следующие функции:

\begin{itemize}
	\item регистрация пользователя;
	\item авторизация пользователя;
	\item просмотр сохраненных образцов почерка;
	\item удаление сохраненных образцов почерка;
	\item добавление нового образца почерка;
	\item выделение признаков образца почерка;
	\item определение параметров личности;
	\item идентификация пользователя по образцу почерка;
	\item определение неврологических отклонений.
\end{itemize}
\todo[inline]{Переделать диаграмму}
\begin{figure}[ht]
\centering
    \includegraphics[scale=0.4]{figures/use_case.png}  
    \caption{Диаграмма вариантов использования}
  \label{fig:freg:usecase}
\end{figure}

Перечисленные функции позволят обеспечить полноценное функционирование программного средства, так как они покрываю все базовые операции хранения данных, а так же включают специализированные операции выделения признаков почерка и из классификации. Так же стоит отметить, что для полноценной промышленного использования программного средства необходимо соблюдение следующих требований:
\begin{itemize}
  \item независимая работа режимов выделения признаков рукописного текста их анализа;
  \item контроль сложности пароля;
  \item пароль не должен хранится в базе данных в открытов виде;
  \item асинхронное взаимодействие всех компонентов;
  \item параметры текста представляют собой вещественные значения с точностью до четырех знаков после запятой;
  \item использование JWT-маркеры для аутентификации и авторизации пользователей сервисами.
\end{itemize}

\subsection{Спецификация функциональных требований}
На основании функций программного средства разработана спецификация требований. 
\subsubsection{Регистрация пользователя}
\label{sec:freq:reg}
\begin{itemize}
	\item возможность регистрации должна быть доступна из пользовательского веб-интерфейса;
	\item пароль при регистрации должен проверяться на сложность(должен содержать не менее восьми символов в верхнем и нижнем регистрах и \mbox{цифры);}
	\item авторизационные данные пользователя должны передаваться по защищенному соединению.
\end{itemize}

\subsubsection{Авторизация пользователя}
\label{sec:freq:auth}
\begin{itemize}
	\item авторизационные данные пользователя должны передаваться по защищенному соединению;
	\item пароль пользователя не должен передаваться в явном виде (вычисления хеша в браузере);
 	\item сообщение об ошибке при вводе неверное логина или пароля не должно сообщать что именно введено неправильно.
\end{itemize}

\subsubsection{Просмотр сохраненных образцов почерка}
\label{sec:freq:show}
\begin{itemize}
	\item список ранее загруженных образцов доступен на главной странице пользователя;
	\item образец и дополнительная информация(признаки почерка, результаты анализа) должны загружаться только после выбора образца из списка;
	\item пользователь имеет возможность запустить процесс анализ образца почерка;
	\item пользователь имеет возможность запустить процесс анализа образца почерка, если процесс выделения признаков почерка еще не был выполнен, то он будет запущен и по окончанию начнется процесс анализа.
\end{itemize}

\subsubsection{Удаление сохраненных образцов почерка}
\label{sec:freq:delete}
\begin{itemize}
	\item возможность удаления сохраненных образцов почерка должна быть доступна из пользовательского веб"=интерфейса страница просмотра образца;
	\item при удалении должно запрашиваться подтверждение действия пользователя с сообщением о последствиях;
	\item реальное удаление образца происходит через неделю после подтверждения удаления пользователем (возможность восстановить файл при ошибочном удалении).
\end{itemize}

\subsubsection{Добавление нового образца почерка}
\label{sec:freq:add}
\begin{itemize}
	\item возможность добавить новый образец доступна на главной странице пользователя;
	\item возможно очистить поле ввода до сохранения образца;
	\item после добавления нового образца пользователь переходит на страницу просмотра образца.
\end{itemize}

\subsubsection{Выделение признаков образца почерка}
\label{sec:freq:extract_features}
\begin{itemize}
	\item возможность выделение признаков образца почерка должна быть доступна из пользовательского веб-интерфейса страница просмотра образца;
	\item выделенные признаки образца почерка передаются по сети в виде JSON-объекта;
	\item выделенные признаки образца почерка хранится в таблице БД;
	\item до завершения выделения признаков на странице отображается индикатор обработки в виде вращающегося круга.
\end{itemize}

\subsubsection{Определение параметров личности}
\label{sec:freq:psiho_analysis}
\begin{itemize}
	\item возможность определение параметров личности должна быть доступна из пользовательского веб-интерфейса страницы просмотра образца;
	\item параметры личности передаются по сети в виде JSON-объекта;
	\item параметры личности хранится в таблице БД;
	\item JSON-объект описывающий параметры личности содержит текстовое описание и метку класса;
	\item до завершения определения параметров личности на странице отображается индикатор отработки.
\end{itemize}

\subsubsection{Идентификация пользователя по образцу почерка}
\todo[inline]{Вычитать}
\label{sec:freq:identification}
\begin{itemize}
	\item необходимо добавить минимум 5 образцов почерка для 
	\item возможность идентификации должна быть доступна из пользовательского веб-интерфейса главной страницы;
	\item результаты идентификации передаются по сети в виде JSON-объекта;
	\item JSON-объект описывающий результаты идентификации содержит текстовое описание и флаг успешности идентификации;
	\item до завершения идентификации на странице отображается индикатор отработки.
\end{itemize}

\subsubsection{Определение неврологических отклонений}
\todo[inline]{Вычитать}
\label{sec:freq:neuro_analysis}
\begin{itemize}
	\item возможность определение неврологических отклонений должна быть доступна из пользовательского веб-интерфейса страницы просмотра образца;
	\item результаты неврологического анализа передаются по сети в виде JSON-объекта;
	\item JSON-объект описывающий результаты неврологического анализа содержит текстовое описание результатов и вещественное значение вероятности отклонений;
	\item до завершения определение неврологических отклонений на странице отображается индикатор отработки в виде вращающегося круга.
\end{itemize}

\subsubsection{Пользовательский интерфейс программного средства}
\begin{itemize}
	\item пользовательский интерфейс представляет собой Web-страницу;
	\item пользовательский интерфейс поддерживает русский и английский \mbox{языки;}
	\item возможность смены языка доступна пользователю на любой \mbox{странице;}
	\item поддержка Google Chrome и Firefox последних версий;
\end{itemize}

Поддержка операционных систем Windows, Linux, MacOS обеспечивается использованием <<тонкого клиента>> и ограничена только поддержкой данными системами версий браузеров.
Разрабатываемое программное средство не должно налагать ограничений на количество обрабатываемых и хранимых образцов почерка.

% Глава 3 Проектирование архитектуры программного средства
\section*{ГЛАВА 3}
\section*{ПРОЕКТИРОВАНИЕ ПРОГРАММНОГО СРЕДСТВА}
\addcontentsline{toc}{section}{ГЛАВА 3}
\addcontentsline{toc}{section}{ПРОЕКТИРОВАНИЕ ПРОГРАММНОГО СРЕДСТВА}
\setcounter{section}{3}
\setcounter{subsection}{0}
\bigskip

Исходя из требований и современных стандартов разработки, программное средство должно обладать следующими свойствами:
\begin{itemize}
    \item модульность;
    \item простота сопровождения;
    \item параллельная обработка данных;
    \item безопасность хранения конфиденциальных и личных данных.
\end{itemize}

Для соответствия выше описанным свойствам было принято решение разрабатывать программное средство с использованием монолитной архитектуры. Так как использование микросервисной архитектуры упростило бы горизонтальное масштабирование и уменьшило бы связность модулей, однако данный подход требует дополнительный усилий по развертыванию, мониторингу и поддерживанию работоспособности программного средства что делает его использование нецелесообразным. 

Разработанное программное средство состоит из следующих модулей:
\begin{itemize}
    \item модуль выделения признаков почерка;
    \item модуль определения характеристик личности;
    \item модуль биометрической аутентификации;
    \item модуль определения неврологических отклонений;
    \item модуль контроля доступа;
    \item модуль доступа к базе данных.
\end{itemize}

Вышеописанные модули опираются на следующие группы классов, разработанные в рамках диссертационной работы:
\begin{itemize}
    \item набор классов для сегментации образца на строки, слова, символы и выделения признаков. Написан на языке Scala;
    \item набор классов машинного обучения (для классификации признаков текста). Написан на языке программирования Scala и содержит реализацию алгоритма основанного на методе опорных векторов (Support Vector Machine), подробнее рассмотрен в главе~\ref{sec:architecture:personal_parameters};
    \item набор классов для контроля доступа. Написана на языке Scala. Содержит классы для регистрации, авторизации и управлением сессией пользователя. Основан на стандарте JSON Web Token (JWT);
    \item набор классов для организации доступа и хранения авторизационных данным пользователя и коллекции обработанных образцов.
\end{itemize}

Далее приведено подробное описание структуры и назначения каждого модуля.
\subsection{Модуль определения неврологических отклонений}
\label{sec:architecture:neuro_ill}
Основываясь на анализе литературных источников в разделе~\ref{sub:domain:literary_sources} и спецификации требований (раздел~\ref{sec:freq:neuro_analysis}) было принято решение для определения неврологических отклонений по признаком рукописного текста использовать размер символов почерка и среднеквадратичное отклонение от скелета символа.

Так как в результирующем наборе параметров, описанный в разделе~\ref{sec:architecture:feature_extraction}, не присутствует размер символов напрямую и для улучшения качества анализа используется набор признаков состоящий из
следующий признаков:
\begin{itemize}
  \item площадь;
  \item интервал между символами;
  \item интервал между словами;
  \item интервал между строками;
  \item частота текста.
\end{itemize}

На основание вышеперечисленных признаков строится заключение об общем размере элементов рукописного текста.
Для проведения скелетизации используется метод уточнения областей. Стоит отметить что из-за формата хранения данных в виде трех отдельных массивов, как и в случае со сегментацией, необходимо преобразовать в единую матрицы координат точек.

При расчете среднеквадратичного отклонения учитываются три смежные точки исходного образца:   
\begin{equation}
  \label{eq:architecture:sko}
  S = \sqrt{\frac{1}{3 n} \sum\limits_{i=1}^{n}\sum\limits_{k=-1}^{1} d(q_i,a_{i+k})^2},
\end{equation}
\begin{explanation}
где & $ S $ & величина среднеквадратичного отклонения; \\
    & $ n $ & количество точек образца; \\
    & $ d(q,a) $ & евклидово расcтояние между точками $q$ и $a$; \\
    & $ q_i $ & точка скелета образца; \\
    & $ a_{i+k} $ & точка исходного образца.
\end{explanation}

На основании величины среднеквадратичного отклонения и интегральной метрики размера элементов почерка может быть рассчитанна вероятность наличия неврологических отклонений.

\subsection{Модуль биометрической аутентификации}
\label{sec:architecture:bioauth}
Основываясь на анализе литературных источников в разделе~\ref{sub:domain:literary_sources}, спецификации требований (раздел~\ref{sec:freq:bio_identification}) и факта что признаки почерка подчиняются нормальному распределению было принято решение использования радиально-базисную функцию Гаусса, представленную в главе~\ref{eq:architecture:gaussian_core} и описываемую формулой~(\ref{eq:architecture:gaussian_core}).

На этапе построения эталона используется 5 образцов почерка, ранее сохраненных в базу данных, на основе выделенных признаков которых и рассчитываются распределения для каждого из признаков.

На этапе аутентификация из представленного образца выделяются признаки почерка и проводится оценка сходства по формуле:

\begin{equation}
  \label{eq:architecture:major_bio_auth}
  R(S) = \frac{1}{n} \sum\limits_{i=1}^{n} w_i \cdot \exp(-\frac{\left|\left| s_i - x_i^{'} \right|\right|}{2\sigma_{i}^2}),
\end{equation}
\begin{explanation}
где & $ R(S) $ & величина сходства образцов;\\
    & $ n $ & количество сравниваемых празнаков почерка;\\
    & $ w_i $ & вес $i$-го признака почерка (коэффициент от 0 до 1);\\
    & $ s_i $ & значение $i$-го признака почерка предъявляемого образца;\\
    & $ x_i^{'}$ & среднее значение $i$-го признака почерка эталонных образцов;\\
    & $ \sigma_{i} $ & дисперсия значения $i$-го признака почерка эталонных образцов. 
\end{explanation}

Далее полученное значение $R(S)$ сравнивается с пороговым значением аутентификации, например 0.85, и выносится решение об успешности процедуры аутентификации.

\subsection{Модуль определения характеристик личности}
\label{sec:architecture:personal_parameters}
Основываясь на анализе литературных источников в разделе~\ref{sub:domain:literary_sources} и спецификации требований, раздел~\ref{sec:freq:psiho_analysis}, было принято решение для определения характеристик личности по признаком рукописного текста использовать классификатор на основе метода опорных векторов, широко используемого метода машинного обучения~\cite{manning_ir}, который при своей относительной простоте реализации, позволяет добиться очень неплохих результатов классификации.

Метод опорных векторов (\emph{SVM, support vector machine}) – семейство схожих алгоритмов обучения с учителем, использующихся для задач классификации и регрессионного анализа. Особым свойством метода опорных векторов является непрерывное уменьшение эмпирической ошибки классификации и увеличение зазора, поэтому метод также известен как метод классификатора с максимальным зазором~\cite{mitchell_ml, wiki_SVM}.

Параметры текста представляют собой непрерывные величины подчиняющиеся нормальному распределению (рисунок~\ref{fig:architecture:normal_pd}). Исходя их этого свойства использование полиномиальных однородных и неоднородных ядер не позволит достичь хороший результатов классификации, для достижения хороших результатов распознавания следует использовать в качестве ядер радиально-базисная функцию Гаусса~\cite{wiki_gauss, orr}:

\begin{equation}
  \label{eq:architecture:gaussian_core}
  k(x) = \exp(-\frac{\left|\left| x - x^{'} \right|\right|}{2\sigma_{}^2}),
\end{equation}
\begin{explanation}
где & $x^{'}$ & среднее значение параметра, рассчитанное для объектов, принадлежащих
классу $C$; \\
    & $ \sigma_{}^2 $ & дисперсия значения параметра объектов из класс $C$.
\end{explanation}

Дисперсия значения параметра представлена формулой:
\begin{equation}
  \label{eq:architecture:dispersion}
  \sigma_{}^2 = \frac{1}{n - 1} \sum\limits_{x \in C} (x_i - \overline{x_{}}^2).
\end{equation}

\begin{figure}[!h]
    \centering
    \includegraphics[width=0.85\textwidth]{figures/gauss.png}
    \caption{График функции плотности вероятности для нормального распределения}
    \label{fig:architecture:normal_pd}
\end{figure}

\subsection{Модуль выделения признаков почерка}
\label{sec:architecture:feature_extraction}
На основании требование, сформированных в разделе~\ref{sec:freq:extract_features}, основным функциями данного модуля являются:
\begin{itemize}
  \item подготовка образца к обработке;
  \item сегментация образца;
  \item выделения признаков почерка.
\end{itemize}

Учитывая цифровой формат ввода образца, процесс подготовки сводится к построению единой матрицы координат точек из двух массивов координат точек и массива флагов разрыва.

Следующий этапам является сегментация образца, а первой стадией сегментации является сегментация строк.
Задача выделения строк сводиться к нахождению верхних и нижних граней строк текста на исходном образце. Алгоритм сегментации строк основывается на том, что средняя плотность точек в межстрочных промежутках существенно ниже средней плотности точек в текстовых строк~\cite{cv_text_image_segmentator}.

Первым этапом необходимо для всех строк образца находим их значения плотности точек:
\begin{equation}
  \label{eq:architecture:line_medium_brigth}
  d_j = d_j(M) = \frac{1}{n}\cdot\sum\limits_{i=1}^{n} m_{ij}.
\end{equation}

Затем необходимо определить среднюю плотность точек всего образца:
\begin{equation}
  \label{eq:architecture:medium_brigth}
  d(M) = \frac{1}{l}\cdot\sum\limits_{j=1}^{l} d_j(M).
\end{equation}

Средняя плотность точек в межстрочных интервалах невелика (близка к нулю). Поэтому плотность точек верхней границы строки можно выразить из плотности точек всего образца:
\begin{equation}
  \label{eq:architecture:line_up_interval_medium_brigth}
  d^{t} = k_{t} \cdot d(M),
\end{equation}
\begin{explanation}
где & $ k_{t} $ & коэффициент от 0 до 1.
\end{explanation}

Аналогично плотность нижней границы может быть выражена через плотности точек всего образца:
\begin{equation}
  \label{eq:architecture:line_down_interval_medium_brigth}
  d^{b} = k_{b} \cdot d(B),
\end{equation}
\begin{explanation}
где & $ k_{b} $ & коэффициент от 0 до 1.
\end{explanation}

Работа алгоритма заключается в последовательном просмотре массива средних значений $ (d_1,...,d_m) $ и выявлении множества пар индексов $ (d^t_i,d^b_i) $ строк соответствующих ниже приведенным условиям и следовательно являющимися верхней $ d^t_i $ и нижней $ d^b_i $ границам строк.

Условия верхней границы текстовой строки:
\begin{itemize}
  \item плотность точек текущей строки превышает границу $ d^{t} $;
  \item плотность точек двух предыдущих строк ниже этой границы;
  \item плотность точек трех последующих строк выше границы $ d^{b} $.
\end{itemize}

Следовательно, должно выполняться логическое условие:
\begin{equation}
  \label{eq:architecture:logic_up_interval}
  (d_{i-2} < d^{t}) \wedge (d_{i-1} < d^{t}) \wedge (d_i > d^{b}) \wedge (d_{i+1} > d^{b}) \wedge (d_{i+2} > d^{b}) \wedge (d_{i+3} > d^{b}).
\end{equation}

Условия нижней границы текстовой строки:
\begin{itemize}     
  \item было зафиксировано начало области;
  \item плотность точек текущей строки превышает границу $ d^{t} $;
  \item плотность точек последующей строки ниже границы $ d^{b} $.
\end{itemize}
     
Или:

\begin{itemize}
   \item было зафиксировано начало области;
   \item плотность точек трех последующих строк ниже границы $ d^{b} $.
\end{itemize}

Следовательно, должно выполняться логическое условие:
\begin{equation}
  \label{eq:architecture:logic_down_interval}
  ((d_{i+1} < d^{b}) \wedge (d_{i+2} < d^{b}) \wedge (d_{i+3} < d^{b}) \vee ((d_i > d^{t}) \wedge (d_{i+1} < d^{b}))).
\end{equation}

Результатом работы алгоритма является множество пар индексов верхних и нижних границ строк. На основе этих данных можно рассчитать высоты строки (разность между индексами). Недостатком данного алгоритма является <<срезание>> символов, которые имеют высоту выше средней.

Для устранения этого недостатка можно использовать следующий прием расширения найденной границы. Необходимо определить строку с минимальной высотой $ H_{min} $, а затем границы всех строк на величину $ 0.3 \cdot  H_{min} $. Данный шаг не приведет к слиянию строк, т.к. межстрочные интервалы текста, как правило, больше чем высота строки.
 
Алгоритмы сегментации слов и символом сходи с алгоритмом сегментации строк. Основными отличиями являются необходимость построения карты плотности столбцов, а не строк, а так же наличие дополнительных этапов постобработки, направленных на удаление ложных границ после сегментации слов и символов. Исходными данными следующего алгоритма являются результаты работы предыдущего, так на вход алгоритма сегментации слов подается результат сегментации строк.

Для сегментации образцов можно использовать готовый алгоритм реализованный в сторонней системы распознавания рукописного текста, например Tesseract или ABBYY FineReader. Однако данный подход требует дополнительных усилий по предварительной обработке образца почерка и лишает разработчика возможности оптимизировать алгоритм под конкретную задачу. Помимо часть подобных средств распространяется на платной основе, что увеличит стоимость разработки и сопровождения.

Следующей функцией данного модуля является выделение из образца следующих признаков текста:
\begin{itemize}
  \item длительность написания;
  \item количество линий;
  \item длина по горизонтали;
  \item длина по вертикали;
  \item площадь;
  \item общая длина;
  \item максимальное ускорение;
  \item минимальное ускорение;
  \item длительность написание по вертикали;
  \item длительность написание по горизонтали;
  \item наклон символов;
  \item наклон строк;
  \item интервал между символами;
  \item интервал между словами;
  \item интервал между строками;
  \item частота текста.
\end{itemize}

Для определения угла наклона символа необходимо определить координаты верхней и нижней точек символа и используя арктангенс вычислить угол:

\begin{equation}
  \label{eq:architecture:symbol_angle}
  \Theta = \tan^{-1}{\frac{y_2 - y_1}{x_2 - x_1}},
\end{equation}
\begin{explanation}
где & $\Theta$ & угол наклона символа; \\
    & $ (x_1, y_1) $ & координаты нижней точеки символа; \\
    & $ (x_2, y_2) $ & координаты верхней точеки символа.
\end{explanation}    

На рисунке~\ref{fig:architecture:symbol_angle} приведены примеры символов с выделенными верхними и нижними точками, координаты $ (x_2, y_2) и (x_1, y_1) $ соответственно и условно обозначенным угол $ \Theta $. Представлены образцы трех типов наклона символов левосторонний, правосторонний и прямой.

Сегментация рукописного текста является нетривиальной задачей ввиду непостоянства таких параметров как пробелы между символами, в плодь до их отсутствия, словами и строками. Задача становится еще сложнее учитывая то, что данные признаки необходимо сохранить и использовать в дальнейшей работе.

\begin{figure}[ht]
    \centering
    \includegraphics[width=0.85\textwidth]{figures/char_angle.png}
    \caption{Пример расчета угла наклона символа}
    \label{fig:architecture:symbol_angle}
\end{figure}

Поскольку процесс сегментации образцов и выделение признаков являются относительно долгими операциями, сегментация вместе с выделением признаков занимает от 0.5 до 3 секунды, в зависимости от особенностей образца, обеспечение параллельной обработки выходит на передний план. К счастью выбранные признаки почерка независимы и могут рассчитываться параллельно. Так же нет необходимость ожидать полной сегментации сегментации для начала выделения признаков, например интервал между строками можно вычислить сразу после разбиения образца на строки. Это позволит значительно ускорить работу данного модуля.

\subsection{Модуль доступа к данным}
Данный модуль отвечает за организацию работы с базой данных и предоставление другим модулям удобного интерфейса.
На основании требований, сформированных в разделах~\ref{sec:freq:show},~\ref{sec:freq:delete},~\ref{sec:freq:add}, основными функциями данного модуля являются:
\begin{itemize}
  \item добавление авторизационных данных пользователя в базу при регистрации (включая проверку дублирования имен пользователей);
  \item проверка наличия пользователя в базе и соответствие хеша пароля;
  \item добавление нового образца в базу;
  \item обновление информации о параметрах образца;
  \item удаление образца из базы.
\end{itemize}

Так как список параметров образца постоянный и значения параметров обновляются крайне редко, в данном случае рационально использование классической реляционной базы данных, не смотря на параллельное добавление параметров на этапе выделения, в тоже время на базу данных не налагается каких-либо существенных ограничений по быстродействию и скорости обработки запросов. Основываясь на выше перечисленном в качестве СУБД была выбрана PosgreSQL, описанный в главе~\ref{sec:techs:postgresql}.
Так же плюсом этого решения можно отнести возможность хранение в базе JSON-объектов, что исключает дополнительное преобразование данных перед отправкой их другим модулям и на клиент, а так же отсутствие платы за использование, что позволит снизить стоимость разработки и эксплуатации.

\subsection{Модуль контроля доступа}
Поскольку разрабатываемое приложение может содержать персональные данные пользователя, вплоть до имени и фамилии в графе комментариев к образцу, задача организации безопасного доступа и хранению подобных данных стоит довольно остро.

Данный модуль отвечает за регистрацию и авторизацию пользователей. 
В данной работе для предоставления защищенного доступа к модулям будет использоваться открытый стандарт JSON Web Token (RFC 7519). JWT-маркер  содержит в зашифрованном виде всю минимально необходимую информацию для аутентификации и авторизации. При этом не требуется хранить в сессии данных о пользователе, так как маркер самодостаточный. Данный факт упрощает организацию системы сессий и хорошо подход для выбранной архитектуры.

На рисунке~\ref{fig:architecture:jwt_diagram} представлен алгоритм создания, подписи и проверки JWT-маркера. В данном примере выдачу и проверку маркера осуществляет один и тот же сервер, однако, как было описано выше, это не является обязательным условием и представлено лишь для упрощения диаграммы.

\begin{figure}[ht]
    \centering
    \includegraphics[width=0.7\textwidth]{figures/jwt_diagram.png}
    \caption{Диаграмма генерации и верификации JWT-маркеров}
    \label{fig:architecture:jwt_diagram}
\end{figure}

На основании требований, сформированных в разделах~\ref{sec:freq:reg} и~\ref{sec:freq:auth}, основными функциями данного модуля являются:
\begin{itemize}
  \item регистрации новых пользователей;
  \item авторизация пользователей;
  \item генерация JWT"=маркеров.
\end{itemize}

Использование контроля доступа на основе JWT-маркеров позволяет снизить нагрузку на модуль контроля доступа, благодаря возможности проверки подлинности маркера на стороне других модулей.

\subsection{Краткие выводы}
В данном разделе была описана архитектура программного средства. Составлен список модулей и определены их взаимосвязи.  Выявлен исчерпывающий набор признаков почерка необходимый для проведения всех заявленных типов анализа.

Разработанное программное средство состоит из следующих модулей:
\begin{itemize}
    \item модуль биометрической аутентификации;
    \item модуль определения неврологических отклонений;
    \item модуль выделения признаков почерка;
    \item модуль определения характеристик личности;
    \item модуль доступа к базе данных.
    \item модуль контроля доступа;
\end{itemize}

Основываясь на вышеизложенном можно приступать к реализации программного средства. %

% Глава 4 Разработка программного средства
\section{Разработка программного средства}
В данном раздела описывается процесс разработки программного \mbox{средства.}

\subsection{Обучение классификатора}
Правильная классификация черт личности является одной из задач данного работы, результат зависит от предподготовки данных, процедуры оценки результата и финального качества обучения.

\begin{figure}[h]
    \centering
    \includegraphics[width=0.7\textwidth]{figures/SVM_flow.png}
    \caption{Алгоритм обучения}
    \label{fig:develoipment:svm_flow}
\end{figure}

Общая схема подготовки, обучения и использования классификатора представлена на рисунке~\ref{fig:develoipment:svm_flow}.

После обучения классификатора можно переходить к классификации образов. 

\subsubsection{Определение параметров личности}
В листинге~\ref{listing:development:classification} представлен метод классификации нового образца почерка на основе обученной модели. Всего в программном средстве различается 16 типов личности определяющих характеристики.
\lstinputlisting[
    style=commonstyle,
    caption=Метод классификации личности по параметрам почерка,
    label=listing:development:classification
]{src/evaluate_single_instance.scala}

Для обучения классификатора будет использоваться обучающая выборка <<IAM Handwriting Database>>~\cite{IAM_handwriting_database} состоящая из 1539 образцов текста, написанных 500 авторами, с заранее выделенными параметрами почерка либо на основании выделенных параметров можно рассчитать параметры используемые в данной работе. Образцы почерка представлены изображениями в формате png, а параметры XML документом, пример параметров представлен в листинге~\ref{listing:development:sample_set}.
\lstinputlisting[
    style=commonstyle,
    caption=Пример XML-документа описывающего параметры почерка,
    label=listing:development:sample_set
]{src/sample_set.xml}

Образцы представлены в выборке в виде png-изображений и для их конвертации в наборы координат используются алгоритмы линейной бинаризации изображений с последующей фильтрацией для удаления шумов и скелетизацией для сокращения объема обрабатываемых данных. Так как ряд динамических параметров, например скорость, не могут быть восстановлены или рассчитаны на основании других параметров они заменяются параметрами по умолчанию. 

Данный объем и содержание обучающей выборки позволяет добиться хорошего качества классификации после обучения благодаря репрезентативности, в частности максимальная разность в количестве образцов разных классов составляет 13\%.

\subsection{Иерархия классов}
\begin{figure}[!ht]
    \centering
    \includegraphics[width=1\textwidth]{figures/classes-fdp.png}
    \caption{Иерархия классов}
    \label{fig:develoipment:class_fdp}
\end{figure}

Общая диаграмма иерархии классов представлена на рисунке~\ref{fig:develoipment:class_fdp}. Все классы отвечающие за обработку запросов и данных, в частности \emph{SampleService}, \emph{ProficeService}, \emph{AuthService} используют контекст исполнения запросов системы акторов библиотеки Akka, описанную в классе \emph{Boot}, благодаря механизму неявных параметров языка Scala. Позволяет обеспечить асинхронную обработку данных и запросов на всех уровнях приложения.

Классы \emph{ItemNotFound}, \emph{Forbidden}, \emph{Unauthorized} являются наследниками класса \emph{Response} и представляют собой обертки для HTTP ответов на различные ошибочные ситуации на стороне клиента.

Классы \emph{Boot}, \emph{App} и \emph{Config} позволяют реализовать механизм управления зависимостями называемый \emph{Cake pattern}. При данном подходе все зависимости модуля, интерфейсы которые он использует в работе, перечисляются в специальной секции. Достоинствами данного подходя является полная поддержка средствами языка и проверка зависимостей на этапе компиляции.

Приведенная диаграмма является обобщенной диаграммой всех модулей и представленные в ней элементы и связи справедливы для всех сервисов разработанного программного средства. Различия заключается лишь в функциях классов сервиса.

Общая структура разделения кода на классы в каждом модуле позволит упростить понимания взаимодействия классов внутри всех модулей после изучения кода хотя бы одного из них.

\subsection{Маршрутизация запросов}
Библиотека Akka-Http, описанная в пункте~\ref{sec:techs:akka_http}, имеет очень выразительный внутренний язык описания разбора, обработки и маршрутизации запросов к сервису. Модуль доступа к данным является связующий звеном между модулями приложения и базой данных. Поскольку одной из задач программного средства является предотвращение не авторизированного доступа к данным. Каждый обработчик запросов содержит секцию отвечающую за проверку прав доступа.
\lstinputlisting[
    style=commonstyle, 
    caption=Маршрутизация запросов к серверу в модуле доступа к данным,
    label=listing:development:db_rout
]{src/routing.scala}
Конструкция \emph{authenticate} описанная в классе \emph{SecurityDirectives} отвечает за проверку прав доступа, подробнее будет рассмотрена в разделе~\ref{sec:development:access_control}, а различные функции с префиксом \emph{path} отвечает за маршрутизацию и разбор параметров запросов. 

Использование высоко-интегрированных компонентов, таких как \emph{jwt-scala} позволяет снизить риски некорректной работы при обновлении версий одного из компонентов и ускорить разработку и сопровождения благодаря использованию схожих концепций и стиля разработки.

\subsection{Хранение и передача данных}
Для хранения учетных и пользовательских данных используется СУБД PostgreSQL, схема данных представлена на рисунке~\ref{fig:develoipment:data_base}.

Условно все данные передаваемые между компонентами можно разделить на образцы почерка, листинг~\ref{listing:development:json:sample}, результаты анализа~\ref{listing:development:json:result} и описание пользовательских данных, листинг~\ref{listing:development:json:user}. Для передачи данных между компонентами используется формат JSON с целью, по возможность, исключения или упрощает преобразование информации при общения сервисов между собой и с клиентом. 

\lstinputlisting[
    style=commonstyle,
    caption=Пример JSON-документа описывающего образец текста,
    label=listing:development:json:sample
]{src/sample_item.json}

\lstinputlisting[
    style=commonstyle,
    caption=Пример JSON-документа описывающего пользователя,
    label=listing:development:json:user
]{src/user_item.json}

\lstinputlisting[
    style=commonstyle,
    caption=Пример JSON-документа результат анализа на неврологические отклонения,
    label=listing:development:json:result
]{src/sample_result.json}

\begin{figure}[!ht]
    \centering
    \includegraphics[width=0.7\textwidth]{figures/dataDiagram.png}
    \caption{Схема данных}
    \label{fig:develoipment:data_base}
\end{figure}

\subsection{Контроль доступа}
\label{sec:development:access_control}
Для контроля доступа используется механизм основанный на использовании JWT-маркера. Основой данного подхода является использование асинхронной криптографии для подписи и проверки JWT-маркера и доверее информации хранящейся в нем. Для подписи используется алгоритм RSA.

В разрабатываемом программное средстве для обеспечение создания, подписи и проверки JWT-маркеров используется модуль \emph{jwt-scala}. Модуль авторизации после после сравнения имени пользователя и пароля с авторизационных данными хранящимися в базе, создает новый JWT-маркер содержащий уникальный идентификатор пользователя, алгоритм подписи, срок действия, тип маркера, а так же информацию о том является ли он администратором. Далее модуль авторизации подписывает созданный маркер своим секретным ключом и добавляет подпись к маркеру. Так же есть возможность передавать отрытый ключ для его проверки в теле маркера.
Финальное содержание JWT-маркера представлено в листинге~\ref{listing:development:jwt_token}. Далее весь маркер переводится в кодировку Base64, части маркера разделяются точкой, и прикрепляется к ответу клиента.
\lstinputlisting[
    style=commonstyle,
    caption=Пример JWT-маркера,
    label=listing:development:jwt_token
]{src/jwt_token.json}

Клиент получив JWT-маркер использует его для подтверждения прав доступа при запросам к сервисам входящим в состав программного средства прикрепляя его к запросам.

Каждый из сервисов приложения имеет открытый ключ, являющийся парным к закрытому ключу сервиса авторизации, для проверки подлинности подписи маркера. После чего данные содержащиеся в теле маркера могут быть использованы в работе, так как их подлинность доказана. Такой подход позволяет снизить нагрузку на сервис авторизации благодаря возможности сервисов самостоятельно осуществлять проверку подлинности маркера.

\subsection{Сборка и развертывание}
Для получения последнюю версию исходного кода программного средства используйте команду <<git clone>> в корневом каталоге ПС. Разрешение зависимостей, сборка и развертывание программного средства выполняется с помощью автоматической системы сборки sbt~\cite{sbt}, поэтому перед началом сборки необходимо установить данный инструмент. Программное средство в своей работе использует ряд библиотек. При помощи команды <<sbt>> в корневом каталоге ПС будет произведены загрузка всех необходимых библиотек включая компилятор языка Scala, а так же компиляция модуля программного средства.
Для развертывания модуля программного средства необходимо указать параметры развертывания в файле \emph{application.conf}, подробнее об этом рассказывается в главе~\ref{sec:manpage:admin_man}, и выполнить команду <<sbt run>>. Процесс сборки и развертывания практически полностью выполняется автоматически и не должен вызвать проблем.

Помимо этого сборка и развертывание приложения на тестовом окружении осуществляется автоматически при внесении изменений в базу исходного кода командой <<git push>>.

В данном разделе была рассмотрена общая структура обучение классификатора на основе метода опорных векторов, диаграмма иерархии классов разработанного программного средства, формата хранения внутрих данных приложения, таких как учетные записи пользователей и образцы почерка, а так же внешние данные, обучающую выборку. На данном этапе разработка и отладка программное средства закончена и можно переходи к его тестированию.

\subsection{Тестирование программного средства}
\label{sec:testing}
Разработанное программное средство представляет собой набор программных модулей и сервер базы данных. Тестирование программного средства представляет собой проверку работоспособности самих разработанных модулей независимо от сервера базы данных. 
Перед подготовкой программного средства к тестированию необходимо выполнить следующие действия:
\begin{enumerate}
  \item развернуть все модули программного средства;
  \item сконфигурировать доступ к базе данных;
  \item сконфигурировать адреса сервисов выделения признаков и определения психологических характеристик в модуле управления образцами почерка.
\end{enumerate}

Для оценки правильности работы программного средства было проведено тестирование.

\subsection{Методика использования программного средства}
\label{sec:manpage:admin_man}
Программное средство определения параметров личности по анализу образцов почерка работает как веб-приложение. Ниже приведено описание вариантов использования и конфигурации модулей приложения.

Модули разрабатываемого программного средства разворачиваются как независимые компоненты, что позволяет добиться большой гибкости.

\subsubsection{Документация Swagger}
Каждый модуль имеет самостоятельную документацию реализованную с помощью инструмента Swagger.
Данная документации содержит следующую информацию:
\begin{itemize}
    \item типы доступных запросов;
    \item коды возможных ответов;
    \item типы и обязательность параметров;
    \item параметры ответа.
\end{itemize}

Просмотреть документацию можно с помощью запроса \mbox{\emph{URL"=модуля/docs}}

\subsubsection{Конфигурация модулей}
Конфигурация модулей осуществятся с помощью файла \emph{application.conf} расположенного в папке \emph{resources} в каталоге с файлами модуля. Для описания конфигурации используется формат HOCON позволяющий представить конфигурацию в легко-читаемом форме. 

\lstinputlisting[
    style=commonstyle,
    caption=Пример файла конфигурации модуля,
    label=listing:manpage:admin_man:app_conf
]{src/application.conf}

Варьируя параметры \emph{host} и \emph{port} можно добиться развертывания модуля на произвольном адресе.

\subsubsection{Конфигурация подключения к БД}
Помимо параметров развертывания и адресов других модулей файл конфигурации модуля доступа к данным содержит блок отвечающий за информацию о базе данных.

\lstinputlisting[
    style=commonstyle,
    caption=Пример блока конфигурации доступа к базе данных,
    label=listing:manpage:admin_man:db_conf
]{src/postgresql.conf}

Наиболее важными параметрами являются \emph{jdbc-url}, \emph{username} и \emph{password}. С помощью параметра \emph{jdbc-url} указывается адрес сервера баз данных и имя базы, а параметра \emph{username} и \emph{password} отвечают за данные для авторизации. Так же при низкой скорость интернет соединения может возникнуть необходимость увеличить значение параметра \emph{connection-timeout}. %

% Глава 5 Анализ полученных результатов
\section{Анализ полученных результатов}
\label{sec:summary}

В данном диссертационной работы были проанализированы существующие методы и подходы в областях автоматизации графологического анализа, диагностики неврологических отклонений и биометрической аутентификации на основе анализа образцов почерка. А так же современные системы проектирования, разработки и сопровождения программных средств и систем. Современные прикладные библиотеки и фреймворки хорошо зарекомендовавшие себя в сфере разработки программного обеспечения и только набирающие популярность.

В ходе работы был установлен исчерпывающий набор признаков почерка позволяющий производить независимый анализ в каждой из трех областей. Не мало важно так жен было избежать дублирования и исключить из списка параметры в высокой степенью зависимости.
Результирующий список признаков имеет следующий вид:
\begin{itemize}
  \item длительность написания;  
  \item количество линий;
  \item длинна по горизонтали;
  \item длинна по вертикали;
  \item площадь;
  \item общая длинна;
  \item максимальное ускорение;
  \item минимальное ускорение;
  \item длительность написание по вертикали;
  \item длительность написание по горизонтали;
  \item наклон символов;
  \item наклон строк;
  \item интервал между символами;
  \item интервал между словами;
  \item интервал между строками;
  \item частота текста.
\end{itemize}

Важно отметить некоторые части разработанного программного средства, в частности модуль определения неврологических отклонений, требует вычисления дополнительных параметров таких как среднеквадратичное отклонение от скелета образца, тем не менее, так как параметр является частным и требуются достаточно существенные вычислительные и временные затраты на его нахождение в результирующий набор он не входит.

В ходе работы был апробированн ряд решений в области классификации признаков, таких как количество классов, типы параметров, в частности решался вопрос использования признака максимальное ускорение как непрерывной величины либо разбиение его на категории в процессе классификации признаков почерка. Были опробованы различные решения в области построения скелета образца  

В ходе работы были проанализированные особенности распределения каждого признака почерка, как при их написании одним человеком, так и разными людьми, что легло в основу коэффициентов слагаемых при биометрической аутентификации, а так же выборе типов входов, непрерывное значение, категория при определении характеристик личности. 

Стоит отметить достаточно большой процент ошибок второго рода при использовании метода биометрической аутентификации, это может быть связано с неполной проработкой модели коэффициентов или излишне высоким порогом аутентификации. Однако уменьшение вышеописанного порога неизбежно приведет в увеличению ошибок первого рода, что недопустимо учитывая специфику области информационной безопасности.

\newcommand{\tableHead}{\hline Тип признака & Название признака & Отклонение от обучающей выборки \\ \hline}

Разброс рассчитанных признаков почерка, представлен в таблице~\ref{table:summary:feauture_error}.
\begin{longtable}[l]{| >{\raggedright}m{0.27\textwidth}
                     | >{\raggedright}m{0.32\textwidth}
                     | >{\centering\arraybackslash}m{0.33\textwidth}|}
  \caption{Исходные данные}
  \label{table:summary:feauture_error} \\
  \endfirsthead
  \caption*{Продолжение таблицы \ref{table:summary:feauture_error}}\\
   \tableHead
  \endhead
    \tableHead

    Наклон & & 10\% \\ \hline
    & наклон символов & 11\% \\ \hline
    & наклон строк & 9\% \\ \hline
    Интервал & & 7\% \\ \hline
    & интервал между символами & 10\% \\ \hline
    & интервал между словами & 7\% \\ \hline
    & интервал между строками & 4\% \\ \hline
    Длинна & & 3\% \\ \hline
    & длинна по горизонтали & 3\% \\ \hline
    & длинна по вертикали & 3\% \\ \hline
    & общая длинна & 3\% \\ \hline
    Прочие & & 5\% \\ \hline
    & количество линий & 4\% \\ \hline
    & частота текста & 3\% \\ \hline
    & площадь & 6\% \\ \hline
\end{longtable}
Как видно ошибка вычисление признаков не превышает 11\% что позволяет судить о качестве полученных данных и доверии к прогнозам построенных на их основе. 

Как было описано выше так был про анализированн разброс признаков в разных образцах почерка одного человека, результаты представлены в таблице~\ref{table:summary:personal_dispersion}.   

\begin{longtable}[l]{| >{\raggedright}m{0.27\textwidth}
                     | >{\raggedright}m{0.32\textwidth}
                     | >{\centering\arraybackslash}m{0.33\textwidth}|}
  \caption{Исходные данные}
  \label{table:summary:personal_dispersion} \\
  \endfirsthead
  \caption*{Продолжение таблицы \ref{table:summary:personal_dispersion}}\\
  \hline Тип признака & Название признака & Отклонение \\ \hline
  \endhead
  \hline Тип признака & Название признака & Отклонение \\ \hline

    Наклон & & 8\% \\ \hline
    & наклон символов & 10\% \\ \hline
    & наклон строк & 6\% \\ \hline
    Интервал & & 11\% \\ \hline
    & интервал между символами & 15\% \\ \hline
    & интервал между словами & 9\% \\ \hline
    & интервал между строками & 8\% \\ \hline
    Длинна & & 7\% \\ \hline
    & длинна по горизонтали & 9\% \\ \hline
    & длинна по вертикали & 7\% \\ \hline
    & общая длинна & 8\% \\ \hline
    Время & & \\ \hline
    & длительность написания & 25\% \\ \hline
    & максимальное ускорение & 15\% \\ \hline
    & минимальное ускорение & 3\% \\ \hline
    & длительность написание по вертикали & 17\%\\ \hline
    & длительность написание по горизонтали & 17\%\\ \hline
    Прочие & & 5\% \\ \hline
    & количество линий & 1\% \\ \hline
    & частота текста & 5\% \\ \hline
    & площадь & 9\% \\ \hline
\end{longtable}
Основываясь на данных из таблицы были рассчитаны коэффициенты для биометрической аутентификации. Стоит отметить что не смотря на низкое значение отклонения такие параметры как количество линий и минимальное ускорение не могут иметь большой вес, так так имеют так же и низкое значение отклонения среди разных людей. Использование большого веса подобных коэффициентов приведет к возрастанию количества ошибок второго рода.

Выбор в качестве архитектуры приложения монолитной архитектуры доказал свою целесообразность, задержки между вызовами метод модулей минимальны, скорость обработки и передачи образцов почерка между модулями приложения достигается путем сведения к минимуму передачи данные по сети Интернет.

Как итог стоит отметить что тема работы полностью раскрыта, задачи исследования выполнены, а  поставленная цель достигнута. %

% Заключение
\sectioncentered*{Заключение}
\addcontentsline{toc}{section}{Заключение}
\label{sec:outro}

В данном диссертационной работе были проанализированы существующие аналоги разрабатываемого программного средства и изучены литературные источники, на основании анализа были составлены и специфицированны требования к разрабатываемому ПС. Тема работы полностью раскрыта, поставленная цель достигнута, а задачи исследования выполнены.

В качестве метода сегментации текста используется статический метод так он дает достойный результат и обладает хорошими временными характеристиками, для задач определения неврологических отклонений и биометрической аутентификации использовались статистические методы, в качестве метода классификации используется метод опорных векторов из-за быстрого времени обучения и распознавания и наличия достаточной обучающей выборки для его использования, а в качестве алгоритма построения скелета образца метод уточнения областей, так как учитываю структуру и особенности входных образцов почерка он дает оптимальный результат скелетизации за минимальное время при сравнении с другими алгоритмами.

На основании требований к программному средству в качестве основного языка программирования был выбран Scala, как язык обеспечивающий высокую производительность вместе с выразительностью и компактностью кода. Фреймворки Akka и Akka-Http были выбраны из-за использования неблокирующих операций и модели акторов что обеспечивает общий подход к разработке и упрощает сопровождение.

На основании спецификации требований и используемых технологий была разработана обобщенная архитектура будущего программного средства состоящая из моделя выделения признаков почерка, модуля определения характеристик личности, модуль биометрической аутентификации, модуль определения неврологических отклонений, модуля контроля доступа и модуля доступа к базе данных.

После разработки было осуществлено тестирование программного и разработана инструкции по реконфигурации и администрированию.

В рамках диссертационной работы было спроектировано, создано и протестировано программное средство анализа почерка на основе траектории линий в психологии, медицине и информационной безопасности, а так же проанализированны полученные результаты. 

Полученные результаты могут лечь в основу новых научных работ или быть расширены путем добавления ролевой модели доступа, дополнительных признаков почерка или реализацией мобильного клиента. %

% Список использованных источников
% Зачем: Изменение надписи для списка литературы
% Почему: Пункт 2.8.1 Требований по оформлению пояснительной записки.
\renewcommand{\bibsection}{\sectioncentered*{Список использованных источников}}
\phantomsection\pagebreak% исправляет нумерацию в документе и исправляет гиперссылки в pdf
\addcontentsline{toc}{section}{Список использованных источников}

% Зачем: Печать списка литературы. База данных литературы - файл bib/database.bib
\bibliography{bib/database}

\bigskip
\textbf{Список публикаций соискателя}
\bigskip

1-А. Верховцов, П. А. Диагностика неврологических заболеваний на основе образцов почерка / П. А. Верховцов // Компьютерные системы и сети: материалы 54-й научной конференции аспирантов, магистрантов и студентов, Минск, 23 – 27 апреля 2018 г. / Белорусский государственный университет информатики и радиоэлектроники. – Минск, 2018. – С. 54 – 55.

2-А. Верховцов, П.А. Методы и подходы автоматизации графологического анализа / П.А. Верховцов // Актуальные направления научных исследований XXI века: теория и практика. Т. 5. № 10 (36). / Воронежский государственный лесотехнический университет им. Г.Ф. Морозова. – Воронеж, 2017. – С. 99-102.

3-А. Верховцов, П.А. Аутентификация на основе рукописной подписи / П.А. Верховцов // Актуальные направления научных исследований XXI века: теория и практика / Воронежский государственный лесотехнический университет им. Г.Ф. Морозова. – Воронеж, 2018. %

% Приложение А Исходный код программного средства
\sectioncentered*{ПРИЛОЖЕНИЕ А}
\label{sec:sources}
\addcontentsline{toc}{section}{Приложение А Исходный код программного средства}
\begin{center}
\textbf{Исходный код программного средства}
\end{center}


\lstinputlisting[style=commonstyle]{src/fulllisting/AuthRoute.scala}
\lstinputlisting[style=commonstyle]{src/fulllisting/Boot.scala}
\lstinputlisting[style=commonstyle]{src/fulllisting/SampleTable.scala}
\lstinputlisting[style=commonstyle]{src/fulllisting/register.component.html}
\lstinputlisting[style=commonstyle]{src/fulllisting/login.component.ts}
%TODO Add sources
 %

\end{document}