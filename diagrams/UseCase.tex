\documentclass[a4paper,11pt,russian]{article}

\usepackage{./tikz-uml}
\usepackage{verbatim}
\usepackage[T2A]{fontenc}
\usepackage[utf8]{inputenc}
\usepackage[russian]{babel}
\usepackage{listings}

\textwidth 18.5cm
\textheight 25.5cm
\hoffset=-2.9cm
\voffset=-2.9cm

\tikzumlset{fill usecase=white}
\tikzumlset{fill note=white}

\lstdefinelanguage{tikzuml}{language=[LaTeX]TeX, classoffset=0, morekeywords={umlbasiccomponent, umlnote, umlusecase, umlactor, umlassoc,  umlinclude}, sensitive=true, morecomment=[l]{\%}}

\begin{document}
	\begin{tikzpicture}
		\umlactor[y=-12]{user}
		\umlnote[y=-14]{user}{Пользователь}

		\umlusecase[x=5,y=-15,width=2cm]{Авторизация}
		\umlusecase[x=-5,y=-15,width=2cm]{Регистрация}
		\umlusecase[x=-5,width=2cm]{Добавление нового образца почерка}
		\umlusecase[x=-5,y=-5,width=2cm]{Просмотр сохраненных образцов почерка}
		\umlusecase[x=-5,y=-10,width=2cm]{Удаление сохраненных образцов почерка}
		\umlusecase[x=3,width=2cm]{Выделение признаков почерка}
		\umlusecase[x=-0.5, y=-5,width=2.3cm]{Определение психологических характеристик личности}
		\umlusecase[x=8,y=-5,width=2.3cm]{Биометрическая аутентификации пользователя}
		\umlusecase[x=4,y=-5,width=2cm]{Определение неврологических отклонений}

		\umlassoc{user}{usecase-1}
		\umlassoc{user}{usecase-2}
		\umlassoc{user}{usecase-3}
		\umlassoc{user}{usecase-4}
		\umlassoc{user}{usecase-5}
		\umlassoc{user}{usecase-6}
		\umlassoc{user}{usecase-7}
		\umlassoc{user}{usecase-8}
		\umlassoc{user}{usecase-9}

		\umlextend{usecase-4}{usecase-3}
		\umlextend{usecase-5}{usecase-4}
		\umlextend{usecase-1}{usecase-2}

		\umlinclude{usecase-7}{usecase-6}
		\umlinclude{usecase-8}{usecase-6}
		\umlinclude{usecase-9}{usecase-6}
	\end{tikzpicture}
\end{document}
