\documentclass{minimal}
\usepackage{tikz}
\usepackage{verbatim}
\usepackage[T2A]{fontenc}
\usepackage[utf8]{inputenc}

\usetikzlibrary{calc,trees,positioning,arrows,chains,shapes.geometric,decorations.pathreplacing,decorations.pathmorphing,shapes,matrix,shapes.symbols}

\tikzset{
>=stealth',
  punktchain/.style={
    rectangle, 
    draw=black, very thick,
    text width=10em, 
    minimum height=3em, 
    text centered, 
    on chain},
  every join/.style={-, thick,shorten >=1pt},
  decoration={brace},
  tuborg/.style={decorate, thick},
  tubnode/.style={midway, right=2pt},
}

\begin{document}
\begin{tikzpicture}
  [node distance=.8cm,
  start chain=going below]
     \node[punktchain, join] (intro) {Формирование обучающей выборки};
     \node[punktchain, join] (probf) {Выделение признаков};
     \node[punktchain, join] (investeringer) {Нормализация признаков};
     \node[punktchain, join] (init) {Инициализация обучающего алгоритма};
     \begin{scope}[start branch=venstre,
        every join/.style={->, thick, shorten <=1pt}, ]
        \node[punktchain, on chain=going left, join=by {<-}]
            (risiko) {Начальные параметры};
    \end{scope}
    \node[punktchain, join] (disk) {Обучение модели};
    \node[punktchain, join] (disk) {Оценка эффективности};
    \begin{scope}[start branch=venstre,
       every join/.style={->, thick, shorten <=1pt}, ]
       \node[punktchain, on chain=going left, join=by {<-}]
            (risiko) {Критерии эффективности};
    \end{scope}
    \node[punktchain, join] (disk) {Регулировка модели};
    \node[punktchain, join] (makro) {Оптимизация модели};
    \node[punktchain, join] (use) {Распознование новых образов};


  \draw[tuborg] let \p1=(intro.north), \p2=(investeringer.south) in
    ($(2, \y1)$) -- ($(2, \y2)$) node[tubnode] {Подготовка};
    
   \draw[tuborg] let \p1=(init.north), \p2=(makro.south) in
    ($(2, \y1)$) -- ($(2, \y2)$) node[tubnode] {Обучение};
    
   \draw[tuborg] let \p1=(use.north), \p2=(use.south) in
    ($(2, \y1)$) -- ($(2, \y2)$) node[tubnode] {Использование};
  \end{tikzpicture}
\end{document}