\sectioncentered*{Заключение}
\addcontentsline{toc}{section}{ЗАКЛЮЧЕНИЕ}
\label{sec:outro}

В данном диссертационной работе были проанализированы существующие аналоги разрабатываемого программного средства и изучены литературные источники, на основании анализа были составлены и специфицированны требования к разрабатываемому ПС. Тема работы полностью раскрыта, поставленная цель достигнута, а задачи исследования выполнены.

В качестве метода сегментации текста используется статический метод так он дает достойный результат и обладает хорошими временными характеристиками, для задач определения неврологических отклонений и биометрической аутентификации использовались статистические методы, в качестве метода классификации используется метод опорных векторов из-за быстрого времени обучения и распознавания и наличия достаточной обучающей выборки для его использования, а в качестве алгоритма построения скелета образца метод уточнения областей, так как учитываю структуру и особенности входных образцов почерка он дает оптимальный результат скелетизации за минимальное время при сравнении с другими алгоритмами.

На основании требований к программному средству в качестве основного языка программирования был выбран Scala, как язык обеспечивающий высокую производительность вместе с выразительностью и компактностью кода. Фреймворки Akka и Akka-Http были выбраны из-за использования неблокирующих операций и модели акторов что обеспечивает общий подход к разработке и упрощает сопровождение.

На основании спецификации требований и используемых технологий была разработана обобщенная архитектура будущего программного средства состоящая из моделя выделения признаков почерка, модуля определения характеристик личности, модуль биометрической аутентификации, модуль определения неврологических отклонений, модуля контроля доступа и модуля доступа к базе данных.

После разработки было осуществлено тестирование программного и разработана инструкции по реконфигурации и администрированию.

В рамках диссертационной работы было спроектировано, создано и протестировано программное средство анализа почерка на основе траектории линий в психологии, медицине и информационной безопасности, а так же проанализированны полученные результаты. 

Полученные результаты могут лечь в основу новых научных работ или быть расширены путем добавления ролевой модели доступа, дополнительных признаков почерка или реализацией мобильного клиента.

\bigskip
\bigskip
\textbf{Основные научные результаты диссертации}
\bigskip

1. В ходе работы был установлен исчерпывающий набор признаков почерка позволяющий производить независимый анализ в каждой из трех областей.

2. В ходе работы были проанализированные особенности распределения каждого признака почерка, как при их написании одним человеком, так и разными людьми что позволило повысить качество процедуры биометрической аутентификации на основании образца почерка.

3. В ходе работы был апробирован ряд решений в области классификации признаков, в частности исследовалось оптимальное количество классов, набор признаков, созависимость признаков в наборе. Так же по ряду признаков проводился анализ оптимального подхода к формированию входных данных классификатора, было принято решение использовать категорию признака вместо непрерывного значения. Вышеописанное позволило повысить качество классификации и как следствие увеличить точность определения психологических характеристик.

\bigskip
\textbf{Рекомендации по практическому использованию результатов}
\bigskip

1. Полученные результаты формируют теоретическую и практическую базу для разработки программного обеспечения анализа почерка на основе траектории линий в психологии, медицине и информационной безопасности и могут быть использованы для дальнейшего развития существующих систем.

2. Результаты работы могут использоваться отделами кадров на предприятиях для оценки психологических характеристик кандидатов.

3. Результаты работы могут использоваться пользователя для самостоятельной диагностики неврологических отклонений и принятии решения о посещении специалиста невролога.