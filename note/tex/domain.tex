\section*{ГЛАВА 1}
\section*{ОБЗОР ПРЕДМЕТНОЙ ОБЛАСТИ}
\addcontentsline{toc}{section}{ГЛАВА 1}
\addcontentsline{toc}{section}{ОБЗОР ПРЕДМЕТНОЙ ОБЛАСТИ}
\label{sec:domain:intro}
\setcounter{section}{1}
\setcounter{subsection}{0}
\bigskip

В данном разделе будет произведён обзор программных средств аналогичных разрабатываемому в рамках работы, а так же литературных источников. Проанализированы преимущества и недостатки различных подходов к выделению, обработке и классификации признаков рукописного текста.

\subsection{Биометрическая аутентификация}
\label{sub:domain:bioauthentication}
\emph{Биометрическая аутентификации} -- это распознавание объекта основано на сравнении физиологических или психологических особенностей этого объекта с его характеристиками, хранящимися в
базе данных системы.

Биометрические технологии можно разделить на две большие категории - физиологические и психологические (поведенческие). В первом случае анализируются такие признаки, как черты лица, структура глаза, параметры пальцев, ладонь, форма руки. Психологические характеристики - это голос человека, особенности его подписи, динамические параметры письма и особенности ввода текста с клавиатуры~\cite{bioauth_dinamic_moves}.
Логически биометрическую систему можно разделить на два модуля: модуль регистрации и модуль аутентификации.

На этапе регистрации биометрические датчики сканируют необходимые физиологические или поведенческие характеристики человека и создают их цифровое представление. Специальный модуль обрабатывает это представление с целью выделить характерные особенности и сгенерировать эталон. На этапе аутентификации биометрический датчик снимает характеристики человека, которого нужно аутентифицировать, и преобразует эти характеристики в тот же цифровой формат, в котором хранится эталон. Полученный набор характеристик сравнивается с хранимым, чтобы определить, соответствуют ли они друг другу.

\subsubsection{Методы биометрической аутентификации}
Методы биометрической аутентификации можно разделить на две \mbox{группы~\cite{wiki_sign_verification}.}

Статические методы -- основываются на физиологической (статической) характеристике человека, то есть уникальной характеристике неотъемлемых от него~\cite{prohorov}.
Примерами статических методов являются:
\begin{itemize}
  \item по отпечатку пальца (уникальность рисунка папиллярных узоров на пальцах);
  \item по форме ладони (трехмерный образ кисти);
  \item по форме лица (выделяются контуры бровей, глаз, носа, губ и вычисляется расстояние между ними);
  \item по сетчатке глаза;
  \item по радужной оболочке глаза;
  \item по ДНК.
\end{itemize}

Динамические методы -- основываются на поведенческой (динамической) характеристике человека, то есть построены на особенностях, характерных для подсознательных движений в процессе воспроизведения какого-либо действия~\cite{habr_bio_auth}.
Примерами динамических методов являются:
\begin{itemize}
  \item по рукописному почерку (подпись);
  \item по клавиатурному почерку (набор кодового слова);
  \item по голосу (частотных и статистических характеристик звука).
\end{itemize}

Сравнивать описанные выше биометрические методы по показаниям ошибок первого рода очень сложно, так как они сильно разнятся из-за сильной зависимости от оборудования на котором они реализованы~\cite{lorette_plamondon}.

По показателям ошибок второго рода общая сортировка методов биометрической аутентификации выглядит так (по возрастанию ошибки):
\begin{enumerate}
  \item ДНК;
  \item радужная оболочка глаза, сетчатка глаза;
  \item отпечаток пальца, термография лица, форма ладони;
  \item форма лица, расположение вен на кисти руки и ладони;
  \item подпись;
  \item клавиатурный почерк;
  \item голос.
\end{enumerate}

При современном уровне развития технологий, с каждым днем растет количество пользовательских данных доступных через глобальную сеть, электронную почту, файловые хранилища, различные облачные сервисы. В связи с этим вопрос надежности процедуры аутентификации стоит, как никогда остро~\cite{vorona}. Однако увеличивающееся число аутентификаций, которое должен проходить пользователь ежедневно, приводит к новым проблемам. Упрощение процедуры аутентификации для конечного пользователя, при этом сохранение ее надежности, является перспективным направлением исследований~[1-А.].

\subsection{Микрография}
\emph{Микрография} - это приобретенное расстройство, которое характеризуется аномально малым, стесненным почерком или прогрессией до постепенному уменьшению почерка.~\cite[c.~221]{larner} Он обычно ассоциируется с нейродегенеративными нарушениями базальных ганглий, такими как болезнь Паркинсона, но также приписывается подкорковым очагам поражения. О'Салливан и Шмитц описывают его как ненормально маленький почерк, который трудно читать~\cite[с.~177]{metman_kompoliti}. Микрография также наблюдается у пациентов с болезнью Вильсона, обсессивно-компульсивным расстройством, метаморфопсией или с изолированными очаговыми поражениями среднего мозга или базальных ганглий~\cite{mavrogiorgou_mergl_tigges_husseini_schroter_juckel_zaudig_hegerl}.

Неврологические заболевания такие как болезнь Паркинсона, синдром дефицита внимания и гиперактивности могут быть диагностированы на начальных стадиях на основе анализа признаков почерка либо динамики изменения параметром~\cite{wiki_micrographia}. Данные подход позволяет удешевить и ускорить первичную диагностику неврологические отклонений~\cite{mavrogiorgou_mergl_tigges_husseini_schroter_juckel_zaudig_hegerl}.
При анализе почерка пациентов страдающих неврологическими отклонениями было выяснено что у 7 - 95\% появляются в различной степени микрография и нечеткость контуров символов вызванная тремором~\cite{mckhight}.

Оба критерия хорошо поддаются автоматическому выделению и оценке. Так для текущего состояния и динамики развития микрографии достаточно несколько образцов почерка на листах одного или схожего размера, например A4, однако наличие линейной сетки на листе позволит получить более точные данные~\cite{nervous_system_disorders}.
Для анализа второго фактора в качестве меры может быть принято отклонение элемента рукописного символа от его скелета. Сумма отклонений деленная на общее число выделенных элементов принимается за текущий показатель~[2-А.].
Существуют так же и другие параметры опущенные в данной статье ввиду большей сложности расчета и автоматизации, а так же меньшей корреляциею с неврологическими отклонениями, например сила нажатия и количество ошибок~\cite{ziliotto_cersosimo_micheli}.

Учитывая индивидуальные особенности почерка отдельного человека и погрешности при оцифровке, сбор образца не в цифровом виде неизбежно приводит к существенного разбросу параметров от человеку к человеку что делает невозможным выявление небольших отклонений без анализ динамики.
Анализируя выше изложенный набор факторов можно повысить эффективность более дорогостоящих и долгих исследований и анализов.
Описанный подход является эффективным, однако он может выявлять неврологические отклонения на ранних стадиях только при анализу динамики изменения параметров, что требует накопления информации об отдельном пациенте.

\subsection{Графология}
\label{sub:domain:grafologic}
\emph{Графология} -- это учение, постулирующее наличие устойчивой связи месту почерком и индивидуальными особенностями личности.

Идея использования почерка для выявления психологических характеристик личности впервые была предложена в 1622 в книге итальянского профессора Камилло Бальдо <<Как узнать природу и качества человека, взглянув на букву, которую он написал>>~\cite{kamillo_grafology}. Первым кто систематизировал знания стал Фландрэна аббат Мишон в 1872 году. Он проанализировал большое количество работ по графологии и образцов почерка и в своей книге <<Система графологии>> предложил \emph{метод Мишона}, он основывался на анализе штрихов, букв, слов, свободных движений, строк и пр.~\cite{mishon_grafology}.

Начиная с середины 20-го века графология начала рассматриваться как псевдонаучное учение~\cite{wiki_graphology}. По результатам исследования профессиональным графологам не удалось достоверно оценить трудовые способности человека~\cite{what_your_handwriting}. В среднем профессиональные графологи давали такую же по степени достоверности оценку, как и люди «с улицы»~\cite{neter_shakhar_psevdograph, king_koehler_psevdograph}. В десятках исследований было показано отсутствие связи особенностей почерка с трудовыми способностями человека.

Тем не менее графология широко используется в современной практике отбора кадров~\cite{graphology_psyfactor}.

Основные признаки почерка, которые анализирует графологическая экспертиза:
\begin{enumerate}
  \item размер букв (очень маленькие, маленькие, средние, крупные);
  \item наклон букв (левый наклон, легкий наклон влево, правый наклон, резкий наклон вправо);
  \item направление почерка: (строчки ползут вверх, строчки прямые, строчки ползут вниз);
  \item размашистость и сила нажима: (легкая, средняя, сильная, очень сильная);
  \item характер написания слов (склонность к соединению букв и слов, склонность к отдалению букв друг от друга, смешанный стиль);
  \item общая оценка (почерк старательный, почерк неровный, почерк небрежный, почерк неразборчивый).
\end{enumerate}

Перечисленные признаки почерка являются устойчивыми, но все же присутствует естественные отклонения параметров (длина, ширина, толщина, угол) от средних значений. Вариация становится наиболее заметной при изменение психологического состояния человека, например при страхе, беспокойстве, алкогольном опьянении~[3-А.].

\subsection{Анализ аналогов}
\label{sub:domain:analogs}
\subsubsection{MovAlyzeR}
\label{sub:domain:analogs:movalyzer}
Программное средство <<MovAlyzeR>> является частью семейства программных средств для работы с рукописным текстов компании <<NeuroScript>> и представляет собой приложение для операционных систем Windows~\cite{analogs_movalyzer}. Пользовательский интерфейс приложения представлен на рисунке~\ref{fig:domain:analogs:movalyzer}.

\begin{figure}[ht]
    \centering
    \includegraphics[width=0.7\textwidth]{figures/movalyzer.png}
    \caption{Программное средство <<Movalyzer>>}
    \label{fig:domain:analogs:movalyzer}
\end{figure}

Основными возможностями ПС являются:
\begin{itemize}
  \item отслеживание положения, давления, ориентации с частотой \mbox{100-200 Гц;}
  \item поддержка отслеживания руки, стилуса и мыши;
  \item измерение координации обеих рук одновременно;
  \item отображение результатов в реальном времени;
  \item изменение толщины линии, визуальная и звуковая обратная связь;
  \item искажение визуальная обратная связь, поворот, перекос и отражение на мониторе компьютера в режиме реального времени;
  \item моделирование, генерация рукописных цифровых данных с шумом и известными характеристиками штрихов;
  \item проверка непротиворечивости;
  \item анализ результатов, статистика результатов с визуализацией;
  \item многостраничная запись, разделение текст на слова и штрихи;
  \item внешние приложения, полная интеграция с вашими собственными модулями с использованием сценариев MATLAB или скомпилированных программ;
  \item оптически сканированные изображения.
\end{itemize}

Основными недостатками ПС являются:
\begin{itemize}
  \item поддержка только ОС семейства Windows (XP, 7, 8);
  \item платное использование (799+ \$).
\end{itemize}

ПС может быть использовано для оценки моторных функций и диагностики неврологических отклонений.

В настоящий момент существует достаточное количество исследований, часть из них будет рассмотрена в разделе~\ref{sub:domain:literary_sources}, доказывающих связь отклонений к почерке с развитием неврологических отклонений, однако на рынке представлено относительно небольшое количество аналогов ввиду дороговизны разработки ПО в сфере здравоохранения и ряда этических вопросов связанных со сбором и обработкой историй болезни.

\subsubsection{Web-Key}
\label{sub:domain:analogs:biokey}
Семейство программных и технических средств компании <<Bio-Key>>~\cite{analogs_biokey} предназначено для внедрения аутентификации на основе отпечатков пальце в корпоративные системы. Далее будет рассмотрено программное средство <<Web-Key>> как наиболее близкое по функционалу к рассматриваемому в работе.

Основными возможностями и достоинствами ПС являются:
\begin{itemize}
  \item высочайшие результаты независимой проверки и проверки NIST по скорости и точности аутентификации на основе отпечатков пальцев;
  \item возможность использования как однофакторной биометрической аутентификации, так и двухфакторной совместно с паролями или другими способами аутентификации для гибкости в развертывании;
  \item поддержка Plug-and-Play (включая Windows Biometric Framework) для интеграции со сканерами отпечатков пальцев от всех основных производителей;
  \item пошаговая аутентификация для доступа к определенным модулям и данным или для завершения определенных конфиденциальных транзакций;
  \item механизм векторной сегментации (VST) позволяющий производить сопоставление «один ко многим» с группами пользователей практически любого размера;
  \item сервер аутентификации может быть развернут в общедоступном или частном облаке для дополнительной масштабируемости и снижения требований к ИТ-ресурсам компании;
  \item шифрование на основе PKI защищает все данные, передаваемые между компонентами ПС;
  \item многоуровневое тройное шифрование самих моделей отпечатков пальцев для предотвращения мошеннического захвата данных;
\end{itemize}

Основными недостатками ПС являются:
\begin{itemize}
  \item необходимость использования специального оборудования;
  \item дорогостоящая лицензия.
\end{itemize}

\subsubsection{BioID Web-Service}
\label{sub:domain:analogs:bioId}
Семейство программных средств компании <<BioID>>~\cite{analogs_bioid} предназначено для аутентификации частных и корпоративных клиентов на основе формы лица. Далее будет рассмотрено программное средство <<BioID Web-Service>> как наиболее близкое по функционалу к рассматриваемому в работе.

Основными возможностями и достоинствами ПС являются:
\begin{itemize}
  \item хорошо спроектированный SOAP и RESTful Web API;
  \item облачное решение с дата-центрами по всему миру;
  \item высокая масштабируемость с балансировщиком нагрузки;
  \item передача данных защищена шифрованием TLS / SSL;
  \item контроль доступа с помощью клиентских сертификатов X.509 (SOAP) или маркера приложения (REST);
  \item изолированный экземпляр сервиса для каждого клиента;
  \item запатентованная антиспуфинговая технология обнаружения фотографий и видео-записей для предотвращения мошенничества и нелегального доступа.
\end{itemize}

Основными недостатками ПС являются:
\begin{itemize}
  \item необходимость использования специального оборудования;
  \item дорогостоящая лицензия.
\end{itemize}

\subsubsection{Aware WebEnroll}
\label{sub:domain:analogs:aware}
Семейство программных средств компании <<Aware>>~\cite{analogs_aware} предназначено для аутентификации корпоративных клиентов на основе широкого набора методов, таких как отпечаток пальца, форма лица, сетчатка глаза, голос и другие. Далее будет рассмотрено программное средство <<WebEnroll> как наиболее близкое по функционалу к рассматриваемому в работе.

Основными возможностями и достоинствами ПС являются:
\begin{itemize}
    \item аудит операций в системе с поиском, сортировкой и фильтрацией;
    \item создание пользователей и групп;
    \item поддержка ролевой модели доступа;
    \item предотвращение повторного использования паролей;
    \item принудительная смена пароля через указанное количество дней;
    \item блокировка аккаунта после указанного количества неудачных входов в систему в течение указанного интервала времени;
    \item блокировка аккаунта через указанное количество дней бездействия.
\end{itemize}

Основными недостатками ПС являются:
\begin{itemize}
  \item необходимость использования специального оборудования;
  \item дорогостоящая лицензия.
\end{itemize}

\subsubsection{Bio-IDiom}
\label{sub:domain:analogs:nec}
Семейство программных средств в области биометрической аутентификации компании <<NEC>>~\cite{analogs_nec} предназначено для повышения уровня безопасности и уменьшения риска несанкционированного доступа к ресурсам корпоративных клиентов на основе широкого набора методов, таких как отпечаток ладони, рисунок капилляров, сетчатка глаза, голос и другие. Далее будет рассмотрено программное средство <<Bio-IDiom> как наиболее близкое по функционалу к рассматриваемому в работе.

Основными возможностями и достоинствами ПС являются:
\begin{itemize}
  \item широкий спектр поддерживаемых способов аутентификации;
  \item возможность использования многофакторной аутентификации.
\end{itemize}

Основными недостатками ПС являются:
\begin{itemize}
  \item необходимость использования специального оборудования;
  \item дорогостоящая лицензия.
\end{itemize}

На текущий момент рынок биометрической аутентификации имеет следующую структуру: голос - 11\%, лицо - 15\%, радужная оболочка глаза 34\%, отпечатки пальцев - 34\%, геометрия руки - 25\%, подпись - 3\%~\cite{vorona}.
На рынке представлен широкий набор программных и аппаратных средств биометрической аутентификации, как для частных, так и для корпоративных клиентов и высокой степенью точности и отказоустойчивости, однако все они отличаются высокой стоимостью лицензии и зачастую требуют использования дорогостоящего оборудования~\cite{wacom_lcd}.

\subsubsection{ScriptAlyzeR}
\label{sub:domain:analogs:neuro_script}
Программное средство <<ScriptAlyzeR>> является частью семейства программных средств для работы с рукописным текстов компании <<NeuroScript>> и представляет собой приложение для операционных систем Windows~\cite{analogs_scriptAlyzer}. Пользовательский интерфейс приложения представлен на рисунке~\ref{fig:domain:analogs:neuro_script}. Программное средство использует одно функциональное ядро с ранее рассмотренным <<MovAlyzeR>> (пункт ~\ref{sub:domain:analogs:movalyzer}), однако имеет отличную область применения.

\begin{figure}[ht]
    \centering
    \includegraphics[width=0.7\textwidth]{figures/neuroscript.png}
    \caption{Программное средство <<ScriptAlyzeR>>}
    \label{fig:domain:analogs:neuro_script}
\end{figure}

Основными возможностями ПС являются:
\begin{itemize}
  \item отслеживание положения, давления, ориентации с частотой \mbox{100-200 Гц;}
	\item поддержка отслеживания руки, стилуса и мыши;
	\item измерение координации обеих рук одновременно;
	\item отображение результатов в реальном времени;
	\item изменение толщины линии, визуальная и звуковая обратная связь;
	\item искажение визуальная обратная связь, поворот, перекос и отражение на мониторе компьютера в режиме реального времени;
	\item моделирование, генерация рукописных цифровых данных с шумом и известными характеристиками штрихов;
	\item проверка непротиворечивости;
	\item анализ результатов, статистика результатов с визуализацией;
	\item многостраничная запись, разделение текст на слова и штрихи;
	\item внешние приложения, полная интеграция с вашими собственными модулями с использованием сценариев MATLAB или скомпилированных программ;
	\item оптически сканированные изображения.
\end{itemize}



Основными недостатками ПС являются:
\begin{itemize}
  \item поддержка только ОС семейства Windows (XP, 7, 8);
  \item платное использование (799+ \$).
\end{itemize}

Согласно утверждениям разработчиков ПС может быть использовано для оценки моторных функций, а так же тестирования на состояние алкогольного опьянения.

\subsubsection{Graphology}
\label{sub:domain:analogs:graphology}
Программное средство <<Graphology>> является приложением для операционной системы Android, разработанным компанией <<LH Apps>>~\cite{analogs_graphology}.

Программное средство <<Graphology>> предназначено для анализа почерка и определения характеристик личности. Алгоритм работы программы основан на обширных исследованиях и был создан при консультации профессиональных экспертов графологии. Пользовательский интерфейс приложения представлен на рисунке~\ref{fig:domain:analogs:graphology}.

\begin{figure}[ht]
    \centering
    \includegraphics[width=0.7\textwidth]{figures/graphology_analog.jpeg}
    \caption{Программное средство <<Graphology>>}
    \label{fig:domain:analogs:graphology}
\end{figure}

Основными возможностями ПС являются:
\begin{itemize}
  \item поддержка ОС Android;
  \item многофакторная оценка характеристик личности (почерк, подпись);
  \item выполнение анализа без доступа в интернет.
\end{itemize}

Основными недостатками ПС являются:
\begin{itemize}
  \item поддержка только ОС Android;
  \item поддержка только английского языка;
  \item для ввода образцов почерка используется экран смартфона, что приводит к искажению в написании символов при низком разрешении и без использование стилуса.
\end{itemize}

\subsubsection{Signature Analysis}
\label{sub:domain:analogs:signature_analysis}

Программное средство <<Signature Analysis>> является приложением для операционной системы Android, разработанным компанией <<Beyond Consultancy Services>>~\cite{analogs_signature_analysis}.

Программное средство <<Signature Analysis>> предназначено для определения характеристик личности по образцу подписи. В разработке участвовал графолог с многолетним опытом, выступающей в качестве консультанта многих крупных компаний.  Пользовательский интерфейс приложения представлен на рисунке~\ref{fig:domain:analogs:signature_analysis}.

\begin{figure}[ht]
    \centering
    \includegraphics[height=0.4\textheight]{figures/analog_signature_analysis.png}
    \caption{Программное средство <<Signature Analysis>>}
    \label{fig:domain:analogs:signature_analysis}
\end{figure}

Основными возможностями ПС являются:
\begin{itemize}
  \item поддержка ОС Android;
  \item широкий спектр анализируемых параметров подписи (скорость, давление, длины, направления);
\end{itemize}

Основными недостатками ПС являются:
\begin{itemize}
  \item поддержка только ОС Android;
  \item платный анализ каждой подписи (0,83 \$);
  \item для работы необходимо интернет соединение;
  \item для ввода образцов почерка используется экран смартфона, что приводит к искажению в написании символов при низком разрешении и без использование стилуса.
\end{itemize}

\subsubsection{My Graphology}
\label{sub:domain:analogs:my_graphology}
Программное средство <<My Graphology>> является приложением для операционной системы Android, разработанным компанией <<PENS>>~\cite{analogs_my_graphology}. Пользовательский интерфейс приложения представлен на рисунке~\ref{fig:domain:analogs:my_graphology}.

\begin{figure}[ht]
    \centering
    \includegraphics[width=0.4\textheight]{figures/analog_my_graphology.jpeg}
    \caption{Программное средство <<My Graphology>>}
    \label{fig:domain:analogs:my_graphology}
\end{figure}

Основными возможностями ПС являются:
\begin{itemize}
  \item поддержка ОС Android;
  \item использование для ввода экрана или фотографии почерка;
  \item выполнение анализа без доступа в интернет.
\end{itemize}

Основными недостатками ПС являются:
\begin{itemize}
  \item поддержка только ОС Android;
  \item в разработке не участвовали эксперты графологи;
  \item поддержка только испанского языка интерфейса.
\end{itemize}

\subsubsection{GRAPHOLOGY signature analysis}
\label{sub:domain:analogs:graphology_sign_analysis}
Программное средство <<GRAPHOLOGY signature analysis>> является приложением для операционной системы Android, разработанным компанией <<DokThor>>~\cite{analogs_graphology_sign_analysis}. Программное средство <<GRAPHOLOGY signature analysis>> предназначено для определения характеристик личности по образцу подписи. Пользовательский интерфейс приложения представлен на рисунке~\ref{fig:domain:analog:graphology_sign_analysis}.

Основными возможностями ПС являются:
\begin{itemize}
  \item поддержка ОС Android;
  \item выполнение анализа без доступа в интернет;
  \item предоставление характеристик личности по 5 основным критериям.
\end{itemize}

Основными недостатками ПС являются:
\begin{itemize}
  \item поддержка только ОС Android;
  \item в разработке не участвовали эксперты графологи;
  \item механизм ввода подписи неочевиден;
  \item для ввода образцов почерка используется экран смартфона, что приводит к искажению в написании символов при низком разрешении и без использования стилуса.
\end{itemize}

\begin{figure}[ht]
    \centering
    \includegraphics[height=0.5\textheight]{figures/analog_graphology_sign_analysis.png}
    \caption{<<GRAPHOLOGY signature analysis>>}
    \label{fig:domain:analog:graphology_sign_analysis}
\end{figure}

\subsubsection{Graphology Lite}
\label{sub:domain:analogs:graphology_lite}

Программное средство <<Graphology Lite>> является приложением для операционной системы Android, разработанным компанией <<Hyperborea>>~\cite{analogs_graphology_lite}. Пользовательский интерфейс приложения представлен на рисунке~\ref{fig:domain:analogs:graphology_lite}.

Основными возможностями ПС являются:
\begin{itemize}
  \item поддержка ОС Android;
  \item выполнение анализа без доступа в интернет;
  \item использование для ввода экрана или фотографии почерка.
\end{itemize}

Основными недостатками ПС являются:
\begin{itemize}
  \item поддержка только ОС Android;
  \item в разработке не участвовали эксперты графологи;
  \item бесплатная версия позволяет произвести анализ только одного образца (платная версия стоит 1.05 \$).
\end{itemize}

\begin{figure}[ht]
    \centering
    \includegraphics[height=0.5\textheight]{figures/analog_graphology_lite.png}
    \caption{Программное средство <<Graphology Lite>>}
    \label{fig:domain:analogs:graphology_lite}
\end{figure}

В результате анализа было выявлено, что текущие аналоги, описанные в данном разделе, не обладают следующими возможностями, необходимыми для эффективного практического использования:
\begin{itemize}
  \item поддержка ОС Linux и MacOS;
  \item поддержка русского языка интерфейса;
  \item взимание платы за использование;
  \item поддержка механизма контроля доступа.
\end{itemize}

\subsection{Анализ литературных источников}
\label{sub:domain:literary_sources}

\subsubsection{Аутентификация на основе рукописной подписи}
В анализируемых литературных источниках описаны два основных подхода к обработке рукописных образцов:
\begin{enumerate}
  \item Анализ самой подписи (подход сводится к сопоставлению двух изображений)~\cite{hahgai_yamanaka_hamamoto, hastie_kishon, gurrala}.
  \item Анализ характеристик подписи, средней силы нажима, скорости, ускорения, т.е. свертка, в которую входит информация по подписи, временными и статистическими характеристиками её написания~\cite{bryxomickii, hao_chan}.
\end{enumerate}

Первый подход отличается высокой скоростью работы, однако подвержен большему числу ошибок в связи с естественными отличиями в написании подписи из-за человеческой физиологии.
Второй подход обладает высокой точностью~\cite{nelson_kishon}, но требует существенных вычислительных затрат как на этапе предобработки образцов почерка и формирования эталонного образца, так и на этапе сопоставления образца, предъявляемого системе с эталонным.
Набор разработанных на сегодняшний день способов представления образцов почерка достаточно широк~\cite{ivanov_korparate_network}:
\begin{itemize}
  \item частотно-временное оконное преобразование Фурье~\cite{vorona, nalwa, koliadin};
  \item вейвлет-преобразование с радиальным базисом~\cite{leus, anisimova};
  \item локальные экстремумы~\cite{nalwa, zhu_yang_zhu};
  \item временные и статистические характеристики (скорость, нажим)~\cite{bryxomickii, hao_chan, ruchai, doroshenko_koctychenko, lognicov};
  \item эллипсы инерции~\cite{nalwa}.
\end{itemize}

Все вышеперечисленные методы можно условно разделить на группы по трактовке подпись, так, например, способ вейвлет-преобразование рассматривает подпись как цифровой сигнал, локальные экстремумы как функцию, а эллипсы инерции как физический объект. Методы, относящиеся к одной группе, имеют схожие достоинства и недостатки.
Основная сложность заключается в формировании списка признаков, которые, во-первых, будут уникально характеризовать любого пользователя, т.е. отрицать существование двух пользователей с одинаковыми значениями параметров, в-третьих значения признака не должно завесить от времени.

\subsubsection{Графология}
Публикации и научные статьи на темы схожие с темой данной работы можно условно разделить по следующим признакам:
\begin{itemize}
  \item метод сегментации;
  \item метод классификации признаков.
\end{itemize}

Метод сегментации является неотъемлемой частью алгоритмов обработки рукописного текста, будь это распознавание или анализ. От качества сегментации напрямую зависит качество работы всего алгоритма поэтому выбор метода сегментации является важным этапом.

В рассмотренной литературе предлагаются следующие алгоритмы сегментации рукописного текста:
\begin{itemize}
  \item преобразования Хафа~\cite{louloudis_gatos_pratikakis_halatsis};
  \item нечеткие интервалы~\cite{louloudis_gatos_pratikakis_halatsis};
  \item нечеткий и адаптивный рекурсивный метод наименьших \mbox{квадратов~\cite{louloudis_gatos_pratikakis_halatsis};}
  \item статистические методы (средний интервал между строками)~\cite{gomathi_umadevi_mohanavel};
  \item метод Лоулодиса-Гатоса-Пратикакиса-Халатсиса (LGPH)~\cite{louloudis_gatos_pratikakis_halatsis};
  \item проецирование контуров~\cite{louloudis_gatos_pratikakis_halatsis}.
\end{itemize}

Преобразования Хафа являются мощным инструментом компьютерного зрения, позволяющим извлекать элементы из образца. Данный метод позволяет достичь высокой точности сегментации текста, однако требует больших вычислительных затрат в связи с объемом вычислений и использованием тригонометрических функций при вычислении.

Метод рекурсивных нечетких и адаптивных наименьших квадратов, так же как и преобразования Хафа, широко используется для выделения областей образца благодаря высокому качеству работы и устойчивости к шумам и искажениям, однако вычислительная стоимость данных методов относительно велика.

Метод Лоулодиса-Гатоса-Пратикакиса-Халатсиса показывает очень хорошие результаты сегментации и по результатам исследований превосходит по качеству и времени работы все приведенные выше алгоритмы, однако он пока находится в состоянии исследования и не имеет реальных примеров реализации и использования в практических проектах.

Методы проецирования контуров и нечетких интервалов хорошо подходят для распознавания печатного текста, но дают плохие результаты при распознавании рукописного из-за динамически изменяющихся интервалов между символами,строками и словами.

Статические методы основаны на разбиении образца в зависимости от распределения средней плотности частей образца, качества работы данных методов ниже чем у преобразований Хафа или рекурсивных методов на основе наименьших квадратов, но время работы намного меньше благодаря меньшему количеству вычислений и быстрым операциям сложению и делению.

В разрабатываемом программном средстве не требуется высокая точность распознавания, как например для распознавания текста, в то время как программное средство может работать с большим количеством образцов одновременно и быстродействие важно. Исходя из этого в качестве алгоритма сегментации будет использоваться статический метод.

Не менее важным является алгоритм классификации, так как именно он будет устанавливать соответствие между параметрами текста и психологическими характеристиками, например силой нажима и степенью наклона символа.

В рассмотренной литературе предлагаются следующие алгоритмы классификации признаков рукописного текста:
\begin{itemize}
  \item нейронные сети~\cite{champa_ananda_kumar_ann, grewal_prashar, gabrani_solomon_dviwe,puri_lakhwani, dang_kumar, kathait_singh};
  \item мешок особенностей~\cite{rothacker_bag_of_features};
  \item метод опорных векторов~\cite{slideshare_khandelwal_garg, gabrani_solomon_dviwe, prasad_singh_sapre};
  \item метод основанный на правилах~\cite{champa_ananda_kumar_rule_base}.
\end{itemize}

Метод основанный на правилах заключается в последовательной проверке соответствия признаков почерка набору правил <<если"=то>>. Как пример можно привести правило <<Если строки наклонены влево и сила нажима слабая, то человек пессимист не склонный к выражению эмоций>>. Данный подход основан только на описании графологических метод, обладает высокой скоростью и не требует обучения. Однако требуется определение границ значений анализируемых параметров, а так же количество правил экспоненциально растет с количеством параметров и их возможных значений, так для двух параметров с тремя значениями каждого понадобиться 9 правил~\cite{champa_ananda_kumar_rule_base}, а для пяти параметров уже 243.

Метод основанный на <<мешке особенностей>> состоит в составлении <<словаря слов>> на основе большой базы образцов. Данный словарь будет содержать фрагмент образца описанный каким-либо дескриптором, например SIFT, и частоту появления этого фрагмента. По сути данный метод рассматривает задачу определения психологических параметров как задачу категоризации. Основными недостатками данного метода является необходимость в сборе и ручной обработке, определения характеристик личности экспертами графологами, огромной базы образцов, т.к. все образцы почерка достаточно похожи и сложно выделить отличительные признаки.

Использование нейронных сетей является хорошим решением благодаря способности сети обобщать данные, а использование обратного распространения ошибок позволяет добиться очень хорошего качества распознавания. Однако выбор оптимальной структуры сети и функции активации нейронов является нетривиальной задачей, так же время обучения и распознавания относительно велико.

Метод опорных векторов основан на нахождении границы классов максимально удаленной от их экземпляров. К его достоинствам относиться хорошее качество распознавания, высокая скорость обучения и классификации. Однако необходим большой объем обучающей выборки, такой чтобы каждый из возможных наборов признаков встречался хотя бы раз. Так же вопрос выбора оптимального типа ядер схож с выбором функции активации для нейронных сетей.

Каждая из описанных задач имеет свои требования к детализации и сложности модели.

Основываясь на проведенном анализе и объеме обучающей выборки, ($\sim$ 1500 образцов), было принято решение использовать метод опорных векторов в качестве классификатора. Точность классификации нейронных сетей и метода опорных векторов примерно равны, однако скорость обучения и распознавания у опорных векторов выше, что так же повлияло на решение.

\subsubsection{Выявление неврологических отклонений по образцу почерка}
Основными проявлениями микрографии являются уменьшение размера символов и нечеткие контуры слов и символов связанные с тремором. Основываясь на этом при анализе образцов почерка необходимо оценить размеры и плотность символов, а так же среднеквадратичное отклонение точек символа от его скелета.

В анализируемых литературных источниках описаны следующие подходы к скелетизации:
\begin{itemize}
  \item волновой метод~\cite{klybkov};
  \item метод уточнения областей~\cite{gonsales_wygs};
  \item метод скелетизации по шаблону~\cite{pfalz_rosenfeld}.
\end{itemize}

Волновой метод заключается в анализе пути прохождения сферической волны по образу. На каждом этапе анализируется смещение центра масс точек, образующих новый шаг волны, относительно его предыдущих положений. После завершения построения скелета с помощью сферической волны, полученный результат оптимизируется и анализируется, отыскиваются особые точки фигуры.

Метод уточнения областей заключается в анализе окрестности каждой из закрашенных точек. Основываясь на состояниях окрестных точек вычисляются два параметра $ A(P1) $ - число переходов от белой точки к черной в цепочке и $ B(P1) $ - общее количество всех черных точек в окрестности. Далее с учетом состояний точек четырехсвязной области принимается решение о изменении состояния точки на белое, иначе состояние точки не меняется.

Метод скелетизации по шаблону заключается в поочередном просмотре все точек образца и анализе окрестности только закрашенных точек:   
\begin{equation}
  \label{eq:architecture:pattern_skelet}
  K = \sum\limits_{i=-1}^{1}\sum\limits_{j=-1}^{1} f(x+j,y_i) \cdot h(i,j),
\end{equation}
\begin{explanation}
где & $ K $ & величина оценки окрестности; \\
    & $ (x,y) $ & координаты исследуемой точки; \\
    & $ f(x,y) $ & состояние точки; \\
    & $ h(i,j) $ & маска, представленная в виде 
$ \begin{bmatrix}
  128 & 64 & 32 \\
  16  & 0  & 8  \\
  4   & 2  & 1
\end{bmatrix} $. 
\end{explanation}

В зависимости от полученного значения $ K $ над группой точек выполняется одна из заранее определенных операций(шаблонов), например удаление центрального элемента. Благодаря использованию степеней двойки в качестве элементов элементов маски, достигается возможность кодирования всех возможных состояний группы точек.  

Анализируя достоинства и недостатки вышеописанных методов, а так же специфику выполняемой работы оптимальным алгоритмом построение скелета образца почерка является метод уточнения областей. Исходный формат ввода позволяет свести в минимуму шумы и избыточную толщину линий, в связи с этим подходу используемые в другим алгоритмах не принесут существенного улучшения результата скелетизации, однако потребуют дополнительных вычислительных ресурсов и увеличат время исполнения.   

\subsection{Обоснование выбора языка и сред разработки}
\label{sec:techs:intro}
Выбор технологий является важным предварительным этапом разработки сложных информационных систем. Платформа и язык программирования, на котором будет реализована система, заслуживает большого внимания, так как множество исследований показали, что выбор языка программирования значительно влияет на производительность труда программистов и качество создаваемого ими кода~\cite[c.~59]{mcconnell_2005}.

На выбор технологий повлияли следующие факторы:
\begin{itemize}
  \item программное средство должно быть выполнено в виде клиент"=серверного приложения;
  \item разрабатываемое ПО должно работать на операционных системах Linux, MacOS и Windows;
  \item разработчик имеет опыт работы с объектно"=ориентированными и функциональными языками программирования.
\end{itemize}

\subsubsection{Язык программирования Scala}
\label{sub:techs:scala}
Scala – мультипарадигмальный, компилируемый, строго типизированный язык программирования, спроектированный кратким и безопасным для простого и быстрого создания компонентного программного обеспечения, сочетающий возможности функционального и объектно-ориентированного программирования~\cite{wiki_scala}.

Scala поддерживает объектно-ориентированную и функциональную парадигмы программирования, но доминирующей является объектно-ориентированная. Язык был выпущен для общего пользования на платформе JVM и .NET, так же создан LLVM-компилятор (Scala Native) и транслятор в JavaScript (ScalaJS).

Отличительные особенности языка Scala:
\begin{itemize}
  \item лаконичность. Код на Scala в средней вдвое короче кода на Java;
  \item открытый исходный код. Код стандартной библиотеки опубликован к открытом доступе на портале GitHub и любой желающий, при наличии желания и способностей, может стать участником проекта;
  \item высокий уровень абстракции. В стандартной библиотеке реализованы большинство типичных операции над строками и коллекциями, в частности итерация по элементам коллекции инкапсулирована в методах map, filter, flatMap. Так же выражение for трансформируется в вызов выше описанных методов, что позволяет использовать в нем пользовательские контейнеры и типы данных;
  \item платформонезависимость. Код языка компилируется в JVM байт-код и может исполняться на любой платформе поддерживающей JVM;
  \item строгая система типов. Позволяет выявить многое ошибки еще на стадии компиляции;
  \item объектно-ориентированность. Язык поддерживает основные концепции объектно-ориентированного программирование (наследование, инкапсуляция, полиморфизм);
  \item функциональность. Язык поддерживает основные концепции функционального программирование (функции высших порядков, сопоставление с образцов, <<ленивые>> вычисления);
  \item расширяемость. В языке присутствуют механизм (неявные преобразования) позволяющий расширять функционал стандартных классов и сторонних библиотек;
  \item использование Java-кода. Возможно использовать не только библиотеки написанные на Java, но и классы написанные на Java в Scala-проекте, но отсутствует полная обратная совместимость;
  \item широкий набор библиотеки. Стандартная библиотека Scala содержит классы для работы с вводом-выводом, регулярными выражениями, параллельной обработки, работы со строками, коллекции. Так же существует большое количество сторонних библиотек.
\end{itemize}

\paragraph{Объектно-ориентированное программирование}
Объектно-ориентированная парадигма программирования играет в языке важную роль. Стандартная библиотека языка реализована в виде набора классов и примесей, а так же модульность обеспечивается классами и пакетами~\cite{horsman_scala}.

Особенности Scala с точки зрения объектно"=ориентированной \mbox{парадигмы:}
\begin{itemize}
  \item статическая сильная полная типизация с автоматическим \mbox{выведением типов;}
  \item наследование, в том числе использование примесей (Traits);
  \item полиморфизм;
  \item инкапсуляция;
  \item конструкторы, деструкторы;
  \item все математические операторы являются методами;
  \item гибкое управление доступом к полям и методам;
  \item метапрограммирование;
  \item объекты компаньоны (используются для инкапсуляции статических полей и методов).
\end{itemize}

\paragraph{Функциональное программирование}
Особенности Scala с точки зрения функциональной парадигмы:
\begin{itemize}
  \item функции высших порядков;
  \item функции объект первого класса;
  \item оптимизация хвостовой рекурсии;
  \item сопоставление с образцов;
  \item поддержка неизменяемых структур данных;
  \item функциональные комбинаторы и композиции;
  \item частичное применение функции;
  \item <<ленивые>> вычисления;
  \item структурное переиспользование неизменяемых коллекций.
\end{itemize}

Основываясь на выше перечисленных факторах было принято решение использовать в качестве основного языка программирования Scala как современный, активно набирающий популярность язык, поддерживающий функциональную и объектно"=ориентированную парадигмы программирования. Так же благодаря интеграции с Java, части кода требующего быстродействия могут быть реализованы на этом языке.

\subsubsection{Фреймворк Akka}
\label{sec:techs:akka}

\emph{Akka} – набор прикладных библиотек (фреймворк) предоставляющий высокоуровневый интерфейс для разработки, развертывания и отладки систем акторов.

Основными достоинствами фреймворка Akka являются:
\begin{itemize}
  \item простота интерфейса, высокая степень абстракции;
  \item устойчивость к отказам, благодаря механизму супервизор;
  \item масштабируемость. Простата добавления акторов в систему и развертывания компонентов на другой машине;
  \item высокая скорость работы и степень параллелизма, благодаря использованию модели акторов;
  \item минимально количество блокирующих операций, отсутствие общего изменяемого состояния.
\end{itemize}

\paragraph{Модель акторов}
\label{sec:techs:akka:actor_model}
\emph{Модель акторов} -- это модель параллельных вычислений, основанная на взаимодействии изолированных примитивов, взаимодействующих по средствам получения и отправки сообщений. Впервые была предложена в 1973 году~\cite{hewitt_bishop_steiger_actor_model}.

Основным понятием данной модели является «актор». Согласно модели актор лишен состояния и информации о структуре системы (количество братьев и родителей).

На рисунке~\ref{fig:techs:akka:actor_hierar} представлена иерархия акторов, в данном случае <<top-1>> является родителем для акторов <<child-1>> и <<child-2>>. Актор <<top-1>> выполняет функции супервизора и в случае ошибки в дочерних акторах получает соответствущее сообщение и может принимать решение для его исправлению, например перезапустить актор. Так же в случае необходимости можно делегировать принятие решения родительскому актору, в данном примере <<user>>.

\begin{figure}[ht]
    \centering
    \includegraphics[width=0.5\textwidth]{figures/actors_hier.png}
    \caption{Иерархия акторов}
    \label{fig:techs:akka:actor_hierar}
\end{figure}

Как видно на рисунке~\ref{fig:techs:akka:actor_model:comulication}, отсутствует прямое взаимодействия акторов между собой, вся коммуникация осуществляется при помощи передачи сообщений. В совокупности с использованием неизменяемых структур данных, механизм сообщений делает всю работу системы акторов априорно асинхронной и неблокирующей.

\begin{figure}[ht]
    \centering
    \includegraphics[width=0.7\textwidth]{figures/actor_model.png}
    \caption{Пример взаимодействия акторов}
    \label{fig:techs:akka:actor_model:comulication}
\end{figure}

При получении сообщения актор может:
\begin{itemize}
  \item отправить конечное число сообщений другим акторам;
  \item создать конечное число новых акторов;
  \item выбрать поведение, которое будет использоваться при обработке следующего полученного сообщения.
\end{itemize}

Данный подход позволяет теоретически полностью избежать блокировок благодаря отсутствию прямых вызовов методов актора и даже механизма ожидания ответа, позволяет добиться прироста производительности сопоставимого с количеством ядер процессоров в среде исполнения, чего невозможно добиться при использовании классической модели параллельности на основе потоков и блокировках при доступе к общему изменяемому состоянию.

Перечисленные достоинства, в особенности устойчивой к отказам, являются крайне важными при реализации любой современной архитектуры приложения. Исходя из выше перечисленного, для обеспечения параллельной обработки данных будет использоваться набор прикладных библиотек (фреймворк) Akka.

\subsubsection{Фреймворк Akka-Http}
\label{sec:techs:akka_http}

\emph{Akka-Http} -- это набор прикладных библиотек (фреймворк), предназначенный для реализации веб-приложений, написанный на Scala. Фреймворк базируется на описанном выше фреймворке Akka, раздел~\ref{sec:techs:akka}, и реализует асинхронное распределение запросов пользователя на иерархию акторов. Является наследником фреймворка Spray и копирует его основные концепции и компоненты.

Основными достоинствами фреймворка Akka-Http являются:
\begin{itemize}
  \item полная асинхронность, отсутствие блокировок (весь интерфейс полностью асинхронный);
  \item высокая производительность (используются специальные низкоуровневые компоненты);
  \item модульность (весь интерфейс полностью асинхронный);
  \item легковесность (включаются только необходимые модули).
\end{itemize}

Данный набор библиотек достаточно молодой и еще не получил широкого распространения в коммерческой разработке, что может сказаться на стабильности работы и производительности, однако разработчики быстро устраняют дефекты и выпускают новые версии фреймворка раз в месяц. Использование данного фреймворка позволит упростить и ускорить разработку, отладку и последующее сопровождения благодаря использования одной концепции с предыдущим фреймворком.
Альтернативой является использование таких фреймворков как Lift и Play, однако они не имеют встроенной поддержки модели акторов, а так же требует дополнительных компонентов и веб-серверов, например Tomcat или Jetty, для развертывания.

Исходя из выше перечисленного, для реализации серверной части приложения будет использоваться набор прикладных библиотек Akka-Http, так как несмотря на возможные недоработки он хорошо встраивается в экосистему данной работы.

\subsubsection{Библиотека TensorFlow}
\label{sec:techs:tensor_flow}

\emph{TensorFlow} -- это открытая программная библиотека для машинного обучения, разработанная компанией Google для решения задач построения и тренировки нейронной сети с целью автоматического нахождения и классификации образов, достигая качества человеческого восприятия Применяется как для исследований, так и для разработки собственных продуктов Google. Основное API для работы с библиотекой реализовано для Python, также существуют реализации для C++, Haskell, Java, Go и Swift.

Является продолжением закрытого проекта DistBelief. Изначально TensorFlow была разработана командой Google Brain для внутреннего использования в Google, в 2015 году система была переведена в свободный доступ с открытой лицензией Apache 2.0.

\subsubsection{СУБД PostgreSQL}
\label{sec:techs:postgresql}
\emph{PostgreSQL} -- это объектно-реляционная система управления базами данных (ORDBMS) с акцентом на расширяемость, соответствие отраслевым стандартам и обратную совместимость. PostgreSQL показывает хорошие результаты начиная от небольших приложений на одной машине и заканчивая большими Web-приложениями и приложениями для хранения и обработки данных со многими одновременными пользователями~\cite{postgres}.

PostgreSQL создана на основе некоммерческой СУБД Postgres, разработанной как open-source проект в Калифорнийском университете в Беркли. Название расшифровывалось как «Post Ingres», и при создании Postgres были применены многие уже ранее сделанные наработки. В настоящий момент PostgreSQL разрабатывается и поддерживается PostgreSQL Global Development Group, группой, состоящей из множества компаний и отдельных участников. Это бесплатная программа с открытым исходным кодом, выпущенная на условиях лицензии PostgreSQL.

PostgreSQL является ACID-совместимым и транзакционным. Основные возможности PostgreSQL являются функции, триггеры, правила и представления, индексы, широкий набор типов данных (сетевые типы, XML-данные, массивы, JSON), пользовательские объекты, внешние ключи, поддержка функций и хранимые процедуры.

На MacOS Server PostgreSQL является базой данных по умолчанию, и также доступна для Microsoft Windows и Linux.

Выше перечисленные достоинства и особенности, а так же потенциальный профиль нагрузку на базу данных свойственный приложениям схожим с проектируемый делают СУБД PostgreSQL оптимальным вариантом для использования в данной работе.

\subsection{Краткие выводы}
\label{sec:domain:summary}
В результате выполнения диссертационной работы должно быть разработано программное средство реализующее процедуры выделения признаков почерка из образца и на основе выделенных признаков определения психологических характеристик личности и их классификацию, определение неврологических отклонений, проведение биометрической аутентификации, а так же реализовывать механизм авторизации для обеспечения секретности данных.

Разрабатываемое программное средство должно выполнять следующие функции:
\begin{itemize}
  \item регистрация пользователя;
  \item авторизация пользователя;
  \item просмотр сохраненных образцов почерка;
  \item удаление сохраненных образцов почерка;
  \item добавление нового образца почерка;
  \item выделение признаков образца почерка;
  \item определение психологических характеристик личности;
  \item определение неврологических отклонений;
  \item проведение биометрической аутентификации.
\end{itemize}

К разрабатываемому программному средству предъявляются следующие требования:
\begin{itemize}
  \item разрабатываемое ПО должно работать на операционных системах Linux, MacOS и Windows;
  \item модуль программного средства требует 512 MB оперативной \mbox{памяти};
  \item программное средство должно быть выполнено в виде клиент"=серверного приложения;
  \item программное средство должно поддерживать английский язык \mbox{интерфейса;}
  \item выходными данными являются документы в формате JSON описывающий результаты анализа образца почерка;
  \item входными данными являются документы в формате JSON описывающий координаты точек образца почерка.
\end{itemize}