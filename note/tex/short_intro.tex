\begin{center}
{\bfseries КРАТКОЕ ВВЕДЕНИЕ}
\end{center}
\label{sec:intro}

Применение информационных систем в коммерческих компаниях стало необходимым условием выживания на рынке, а для социальных и государственных организаций и частных некоммерческих проектов использование информационных систем является единственным способом предоставить уровень сервиса, соответствующий современным стандартам. 

При глобальном применении информационных технологий встает вопрос об организации и разграничении доступа к информации, обрабатываемой системой, так как часть информации может представлять коммерческую тайну и/или содержать персональные данные.

На фоне глобального применения информационных технологий вопросы подбора квалифицированного и исполнительного персонала встают как никогда остро, так как все чаще возникают ситуации когда человек является самым медленным и ненадежным элементов производственной цепочки, а ошибки могут обернуться для компании не только потерями прибыли, но и репутации.

В рамках данной работы рассматривается возможность автоматизации методов графологии, биометрической аутентификации и анализа неврологических отклонений на основе образцов почерка для определения психологических характеристик человека, вероятных неврологических отклонений, а также проведение аутентификации.

Разрабатываемое программное средство может продемонстрировать разнообразные способы использования набора признаков почерка в трех различных областях:
\begin{itemize}
	\item определения характеристик личности;
	\item биометрической аутентификации;
	\item определения неврологических отклонений.
\end{itemize}

Результаты работы формируют теоретическую и практическую базу для будущих разработок и развития существующих систем.
Использование разработанного программного средства отделами кадров для оценки психологических характеристик кандидатов позволит ускорить процесс принятия решения о найме и, тем самым, сократить издержки предприятия. Разработанное программное средство может помочь пользователям провести самостоятельную диагностику неврологических отклонений и принятии своевременного решения о посещении специалиста-невролога.
\clearpage