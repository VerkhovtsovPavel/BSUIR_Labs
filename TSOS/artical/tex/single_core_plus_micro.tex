\section{Одноядерные чипы ЦОС $ + $ микроконтроллер}

В другом классе процессоров ЦОС используется дополнительное микроконтроллерное ядро на чипе. Иногда это отдельное ядро, такое как процессор ARM. В других случаях ядро процессора содержит как функции ЦОС, так и MCU. Подобный подход имеет название Blackfin архитектура процессоров ЦОС от Analog Devices. 
Blackfin~\cite{blackfin} основывается на 10-ступенчатой RISC - MCU / ЦОС конвейере в смешанной с 16/32-битной архитектурой набора команд. Это обеспечивает комбинацию функции обработки сигналов и использования универсального микроконтроллера. Blackfin архитектура процессора полностью SIMD-совместимый (одна инструкции, множественные данные) и включает в себя инструкцию для ускоренной обработка видео и изображений~\cite{blackfin_data_sheet}. 
Такое сочетание обработки отличает Blackfin процессоры от своих братьев. Они призваны одинаково хорошо работать как в приложениях обработки сигналов, так и в области управления, во многих случаях устраняя требование для отдельных процессоров в системе. Blackfin процессоры работают с частотой до 756 МГц в одноядерной версии. 
Помимо встроенную поддержки 8-битовых данных, который является размером слова , общий для многих алгоритмов пиксельных обработки, так же Blackfin архитектура включает в себя инструкцию дополнительные инструкции , в частности, для повышения производительности видео-обработки. Например, команда «SUM ABSOLUTE DIFFERENCE» поддерживает алгоритмы оценки движения, используемые в алгоритмах сжатия видео, таких как MPEG2, MPEG4 и JPEG. 
Архитектура обрабатывает кодировку команд различной длинны. Очень часто используемые инструкции типа управления длинной 16-битные слова вперемешку с более математическими инструкциями обработки сигналов, закодированными как 32-битные значения. Процессор будет смешивать и связывать 16-битные инструкции управления с 32-битными инструкциями обработки сигналов в 64-битные группы, чтобы максимизировать использовать памяти. При кэшировании и извлечению инструкций  ядро автоматически полностью заполняет длину шины, поскольку она не имеет ограничений по выравниванию. 
Все Blackfin процессоры, такие как ADSP-BF523, содержат независимые контроллеры DMA, которые поддерживают автоматическую передачу данных с минимальными накладными расходами от ядра процессора. Передачи с использованием DMA могут происходить между внутренними памятью и любым из многих периферийных устройств, поддерживающих DMA. Передачи также могут происходить между периферийными устройствами и внешними устройствами, подключенными к интерфейсам внешней памяти, включая контроллер SDRAM и контроллер асинхронной памяти. 
Архитектура памяти включает в себя как блоки памяти L1, так и L2. L1-память подключается непосредственно к ядру процессора, работает с тактовой частотой системы и обеспечивает максимальную производительность системы для сегментов программы критичный по времени. Кроме того, память L1 может быть настроена как SRAM, кэш или их комбинация. 
Поддерживая как модели SRAM, так и модели кэширования, разработчики системы могут выделять критически важные наборы данных обработки для обработки сигналов в реальном времени, которые требуют высокой пропускной способности и низкой задержки в SRAM, сохраняя при этом задачи управления и оперативной системы (OS) в «мягкие» реальном времени кэш-память. Память L2 представляет собой больший по объему блок памяти для хранения, который предлагает несколько снижения производительности, но все еще быстрее, чем внешняя память. 
Каждый Blackfin процессор использует несколько методов энергосбережения, которые выборочно отключает питание функциональных блоков на основе принципа инструкция за инструкцией. Эти процессоры также поддерживают несколько режимов отключения для периодов, когда требуется мало ресурсов или нет активности CPU. 
В этом самодостаточной схеме динамического управления питанием, рабочая частота и напряжение могут быть независимо друг от друга , чтобы манипулировать удовлетворение требований к производительности алгоритма в настоящее время выполняется. Большинство процессоров Blackfin предлагают на кристалле ядра схемы регулирования напряжения данная возможность особенно хорошо подходят для портативных приложений , которые требуют долгого срок службы батареи. 
Blackfin процессоры поставляются с различными периферийными устройствами микроконтроллера, в том числе 10/100 Ethernet MAC, UARTs, SPI, CAN контроллер, таймеры с поддержкой импульсной модуляции, сторожевые таймеры, часы реального времени, и синхронных и асинхронных контроллер памяти.
