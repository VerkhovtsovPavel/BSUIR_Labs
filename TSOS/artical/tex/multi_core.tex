\section{Многоядерные процессоры ЦОС}

Хороший примером многоядерного ЦОС является Freescale MSC8144~\cite{freescale_data_sheet}, который основан на технологии компаний StarCore третьего поколения ядра ЦОС SC3400.
Чип включает четыре подсистемы ЦОС. Внутри каждой подсистемы находится ядро ЦОС SC3400, кэш-память 16-kbyte L1, кэш данных L1 объемом 32 Кбайт, блок управления памятью (MMU), расширенный программируемый контроллер прерываний (EPIC) и два 32-разрядных таймера общего назначения. Подсистема поддерживает отладочную и профилирующую поддержку и режимы работы с низким энергопотреблением Wait и Stop. Каждое ядро ЦОС работает на частоте до 1 ГГц, поэтому чип обеспечивает эквивалентную производительность одноядерного ЦОС 4 ГГц. 
MSC8144 также содержит подсистему технологии QUICC Engine, которая включает в себя два RISC-процессора, 48-килобайтную многомассивную RAM и 48-килобайтную оперативную память. Эта подсистема поддерживает три коммуникационных контроллера с одним асинхронным режимом передачи (ATM) и двумя интерфейсами Gigabit Ethernet. Он также может снимать задачи планирования с ядер ЦОС. 
Контроллер ATM поддерживает UTOPIA level II 8/16 бит на частоте 25/50 МГц в режиме UTOPIA / POS с поддержкой уровня адаптации для AAL0, AAL2 и AAL5. Два контроллера Ethernet поддерживают 10 / 100/1000-Мбит / с операций с помощью MII / RMII / SMII / RGMII / SGMII и протоколом SGMII с использованием интерфейса четырех-контактный сериализатору / десериализатор (SERDES) со скоростью передачи данных 1000 Мбит / с. 
Как и чипы ЦОС упоминалось ранее, в данном присутствуют QUICC подсистемы с памятью, интерфейсы ввода / вывода. Что касается памяти, чип содержит 128-килобайтный общий кэш-память L2, память М2 объемом 512 Кбайт для критических данных и временную буферизацию данных, загрузочное ПЗУ 96-килобайт и огромные 10 Мбайт 128-битной памяти M3. 
Контроллеры DDR и DMA также находятся на чипе. Контроллер DDR имеет до 200 МГц (400 МГц) и 16/32-битную шину данных. Он поддерживает до 1 Гбайт памяти DDR1 и DDR2 в одном или двух банков. Контроллер DMA имеет 16 двунаправленных каналов с 1024 дескрипторами буфера и программируемой конфигурацией приоритета, буфера и мультиплексирования. 
Система арбитража и коммутации на уровне чипа (CLASS) обеспечивает полный неблокирующий арбитраж между элементами обработки (и другими инициаторами) и такими элементами, как память M2, контроллер DDR SRAM, а также регистры управления конфигурацией устройства и состояния. 
MSC8144 поддерживает интерфейсы следующего поколения и устаревшие интерфейсы, такие как двойной Gigabit Ethernet, Serial RapidIO межсоединения, UTOPIA,, PCI, и временным разделением каналов (TDM).
Последовательный RapidIO 1x / 4x порт соответствует спецификации 1.2 торговой ассоциации RapidIO. Он поддерживает чтение, запись, сообщения и обращения к техническим средствам в режиме входящего соединения, а также сообщения в режиме исходящего. Интерфейс PCI соответствует спецификации PCI версии 2.2 на частоте 33 или 66 МГц с доступом ко всем адресным пространствам PCI.
Чип содержит до восьми независимых модулей TDM предоставляющих функции , такие как программируемый размер слова (2-, 4-, 8- или 16-бит), аппаратный преобразования A-закон / $\mu$-закону, со скоростью до 128 Мбит по всем каналам, с интерфейсом к E1 или T1 и способностью взаимодействовать с Н-MVIP / H.110 устройствами, TSI, и кодеками , такими как AC'97. 
Благодаря многоядерной архитектуре, наличию большого количества интерфейсов, MSC8144DSP хорошо подходит для приложений с большими объемами данных. Такие как услуги Triple-Play (голос, видео и данные), звонки по интернет - протоколу (VoIP) , медиа-шлюзы, видео-конференц оборудования, а также WCDMA и WiMAX базовые станции.