\sectioncentered*{Введение}
\addcontentsline{toc}{section}{Введение}
\label{sec:intro}

В некоторых случаях, когда аналоговые схемы не могут рассматриваться для решении задачи из-за экономических издержек или сложности, ЦОС все еще являются жизнеспособным выбором и во многих случаях выполняют эти задачи без особых усилий. Преимущество заключается в том, что ЦОС очень хороша и очень быстра при арифметических операциях, таких как сложение и умножение и в настоящее время создано огромное количество оптимизированных алгоритмов для решения сложных задач обработки сигналов, используя в основном эти два математические операции.

Современные специализированные чипы ЦОС - это нечто большее, чем просто процессор. В них также интегрированы подсистемы памяти, высокоскоростные интерфейсы ввода / вывод, и многие другие. Эти элементы включены с целью повышения общей производительности, снижения энергопотребления и ориентации на конкретные задачи обработки сигналов.

Чтобы лучше понять различные варианты чипов ЦОС и как различные части устройства работают вместе над для решение общей задачи, полезно рассмотреть несколько специализированных процессоров ЦОС представительных на рынке сегодня. Мы рассмотрим примеры одноядерных процессоров, одноядерных процессоров с использованием микроконтроллеров и многоядерных процессоров ЦОС.
