\section{Одноядерные чипы ЦОС}

Неверно, но чипы ЦОС могут имеють лишь одно ядро. Возьмем, к примеру, TMS320C6452 Texas Instruments~\cite{building_simple}. Являясь членом семейства высокопроизводительных чипов ЦОС основанных на архитектуре с фиксированной точкой TMS320C64x+, чип предназначен для интенсивной многоканальной телекоммуникационной инфраструктуры и медицинских систем визуализации. Ядро ЦОС является лишь частью чипа. Чип так же содержит память, порты ввод / вывод, а также другие функциональные блоки. 
C6452 имеет интегрированную в кристалл памяти, организованную в виде двухуровневой системы. Первый уровень памяти (L1) содержит память программы и память данные по 32 кбайт каждая. Эта память может быть сконфигурирована как отображение RAM, кэш или их комбинация. 
Когда L1 память сконфигурирован как кэш, память программа (L1P) представляет собой кэш с прямой адресацией, тогда как память данные L1 (L1D) представляют собой двунаправленный ассоциативный кэш. Память уровня 2 (L2) разделяется между памятью программы и пространством данных. Память L2 также может быть сконфигурирована как RAM, кэш или их комбинация. Дизайнеры могут использовать встроенную память для реализации различных функций в своих проектах~\cite{powering_today}. 
C6452 также включает два порта доступа к среде Ethernet с независимым интерфейсом Serial Gigabit Media (SGMII) и один гигабитный коммутатор. Коммутатор повышает эффективность многочиповых архитектур, автоматически контролируя поток данных, чтобы гарантировать, что только подходящие узлы коммутированы на решающий шлюз, данный подход может использоваться для распознавания голоса. Если чип ЦОС полностью посвящен речевой обработке, он может блокировать входной поток данных, что значительно повышает эффективность использования его пропускной способности. Кроме того, устройство поставляется с двумя телекоммуникационными портами последовательного интерфейса (TSIPs), обеспечивая беспрепятственное подключение к общим телекоммуникационным потокам последовательных данных. 
Другие порты входы / выходы на C6452 включают в себя интерфейс PCI 66 МГц или универсальный хост - порт интерфейса (UHPI); интерфейс внешней памяти с двойной скоростью передачи данных (DDR2); VLYNQ, проприетарный последовательный интерфейс связи, разработанный TI; 16-битный интерфейс внешней памяти (EMIFA); многоканальный универсальный аудио последовательный порт (McASP); и другие интерфейсы. Основываясь на этом списке нет сомнений что он найдет свое место в телекоммуникационных приложениях, а так же работать в других сегментах. 
В основе C6452 и нескольких других чипов ЦОС от Texas Instruments лежит мегамодуль C64x, состоящий из нескольких компонентов: процессора C64x+, контроллера памяти L1 и памяти данных, контроллера памяти L2, внутреннего DMA (IDMA), контроллера прерываний, мощности контроллер и контроллер внешней памяти. Модный модуль также поддерживает защиту памяти для L1P, L1D и L2-памяти. Он обеспечивает управление пропускной способностью для локальных ресурсов для мега-модуля. 
Процессор C64x+ на модуле является очень быстрым процессором
 ЦОС, который может работать со скоростью до 1,2 ГГц. Он использует восемь функциональных блоков, два регистрационных файла и два пути данных. Два из этих восьми функциональных блоков являются мультипликаторами или единицами M. Каждый M-блок выполняет четыре 16-битных умножения-накапливает (MAC) каждый такт. 
Таким образом, восемь 16-бит 16-разрядных MAC-адресов могут выполняться каждый цикл на ядре C64x+. При частоте 1,2 ГГц 9600 16-разрядных MMAC могут возникать каждую секунду. Более того, каждый умножитель на ядре C64x+ может вычислять один 32-разрядный MAC-адрес или четыре 8-битных MAC-адреса каждый такт. 
Новая функция процессора C64x+ имеет привлекательное имя SPLOOP. Этот небольшой буфер команд помогает создавать петли конвейерной обработки программного обеспечения, где параллельно выполняются несколько итераций цикла. Буфер SPLOOP уменьшает размер кода, связанного с конвейерной обработкой программного обеспечения.