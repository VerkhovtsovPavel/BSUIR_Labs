\def \byr{Br}

\section{Технико-экономическое обоснование разработки и использования ПС определения психологические характеристик личности по анализу образцов почерка}

% Категория сложности
\def \complexityGroup{3}

% Категория новизны
\def \noveltyGroup{Б}

% ---
Разрабатываемое в дипломе ПС позволяет определять психологические характеристики личности по образцу почерка, а так же выделять признаки почерка из образцов.

Программное средство относится к \complexityGroup-й группе сложности. Категория новизны – <<\noveltyGroup>>. 
При оценке экономической эффективности ПС будет рассчитана смета затрат на разработку, цена и прибыль.

Расчеты выполнены на основе методического пособий~\cite{palicyn_2006,gorovoi_2014}.

\subsection{Расчёт сметы затрат и цены программного продукта}

% Коэффициент сложности, ед.
\def \additionalComplexityFactor{0.12}
\FPeval{complexityFactor}{clip(1 + \additionalComplexityFactor)}

% Степень использования при разработке стандартных модулей, ед.
\def \stdModuleUsageFactor{0.5} 

% Коэффициент новизны, Кн
\def \noveltyFactor{0.9}

% Годовой эффективный фонд времени, дн.
\def \daysInYear{365}
\def \holidaysInYear{9}
\def \weekendDaysInYear{103}
\def \vacationDaysInYear{24}
\FPeval{workingDaysInYear}{,
    clip(\daysInYear -
         \holidaysInYear -
         \weekendDaysInYear -
         \vacationDaysInYear)
}

% Среднемесячная норма рабочего времени, Фр
\def \workingHoursInMonth{160}

% Продолжительность рабочего дня, ч.
\def \workingHoursInDay{8}

% Месячная тарифная ставка первого разряда, Br
% \def \tariffRate{298000} % Гос. предприятия
\def \tariffRate{31}

% Коэффициент премирования, ед
\def \bonusRate{1.5}

% Норматив дополнительной заработной платы, Br
\def \additionalSalaryRate{20}

% Норматив отчислений в ФСЗН и обязательное страхование, %
\def \socialProtectionRate{34.6}

% Норматив прочих затрат, %
\def \otherExpenseRate{20}

% Норматив накладных расходов, %
\def \overheadExpenseRate{100}

% Прогнозируемый уровень рентабельности, %
\def \profitability{35}

% Норматив НДС, %
\def \vatRate{20}

% Норматив налога на прибыль, %
\def \profitTaxRate{18}

% Норматив расхода материалов, %
\def \materialsRate{3}

% Норматив расхода машинного времени, ч.
\def \debugRate{15} % часов / 100 строк кода

% Цена одного часа машинного времени, Br
\def \machineHourCost{2.5}

% Норматив расходов на сопровождение и адаптацию ПО, %
\def \supportAndAdaptationRate{30}

% ---
На основании сметы затрат и анализа рынка ПО определяется плановая отпускаемая цена.
Для составления сметы затрат на создание ПО необходима предварительная оценка трудоемкости ПО и его объёма.
Расчет объёма программного продукта предполагает определение типа программного обеспечения, всестороннее техническое обоснование функций ПО и определение объёма каждой функций.
Согласно классификации типов программного обеспечения~\cite[с.~59,~приложение 1]{palicyn_2006}, разрабатываемое ПО можно классифицировать как ПО методo"=ориентированных расчетов.

Общий объём программного продукта определяется исходя из количества и объёма функций, реализованных в программе:
\begin{equation}
  \label{eq:economics:total_loc}
  V_{o} = \sum_{i = 1}^{n} V_{i} \text{\,,}
\end{equation}
\begin{explanation}
где & $ V_{i} $ & объём отдельной функции ПО, LoC; \\
    & $ n $ & общее число функций.
\end{explanation}

Исходные данные разрабатываемого проекта указаны в таблице~\ref{table:economics:initial_data}.
\clearpage
\begin{longtable}[l]{| >{\raggedright}m{0.62\textwidth}
                     | >{\centering}m{0.17\textwidth}
                     | >{\centering\arraybackslash}m{0.13\textwidth}|}
  \caption{Исходные данные}
  \label{table:economics:initial_data} \\
  \endfirsthead
  \caption*{Продолжение таблицы \ref{table:economics:initial_data}}\\
   \hline
    {\begin{center} Наименование \end{center} } & Условное обозначение & Значение
    \\ \hline
  \endhead

   \hline
    {\begin{center} Наименование \end{center} } & Условное обозначение & Значение
    \\ \hline

    Категория сложности
    & & \complexityGroup
    \\ \hline

    Коэффициент сложности, ед.
    & $ \text{К}_\text{с} $ & \num{\complexityFactor}
    \\ \hline

    Степень использования при разработке стандартных модулей, ед.
    & $ \text{К}_\text{т} $ & \num{\stdModuleUsageFactor}
    \\ \hline

    Коэффициент новизны, ед.
    & $ \text{К}_\text{н} $ & \num{\noveltyFactor}
    \\ \hline

    Годовой эффективный фонд времени, дн.
    & $ \text{Ф}_\text{эф} $ & \num{\workingDaysInYear}
    \\ \hline

    Продолжительность рабочего дня, ч.
    & $ \text{Т}_\text{ч} $ & \num{\workingHoursInDay}
    \\ \hline

    Месячная тарифная ставка первого разряда, действующая на предприятии, \byr{}
    & $ \text{Т}_{\text{м}_{1}}$ & \num{\tariffRate}
    \\ \hline

    Коэффициент премирования, ед.
    & $ \text{К} $ & \num{\bonusRate}
    \\ \hline

    Норматив дополнительной заработной платы, ед.
    & $ \text{Н}_\text{д} $ & \num{\additionalSalaryRate}
    \\ \hline

    Норматив отчислений в ФСЗН и обязательное страхование, $\%$
    & $ \text{Н}_\text{сз} $ & \num{\socialProtectionRate}
    \\ \hline

    Норматив прочих затрат, $\%$
    & $ \text{Н}_\text{пз} $ & \num{\otherExpenseRate}
    \\ \hline

    Норматив накладных расходов, $\%$
    & $ \text{Н}_\text{рн} $ & \num{\overheadExpenseRate}
    \\ \hline

    Прогнозируемый уровень рентабельности,
    $\%$ & $ \text{У}_\text{рп} $ & \num{\profitability}
    \\ \hline

    Ставка НДС, $\%$
    & $ \text{Н}_\text{дс} $ & \num{\vatRate}
    \\ \hline

    Ставка налога на прибыль, $\%$
    & $ \text{Н}_\text{п} $ & \num{\profitTaxRate}
    \\ \hline

    Норматив расхода материалов, $\%$
    & $ \text{Н}_\text{мз} $ & \num{\materialsRate}
    \\ \hline

    Норматив расхода машинного времени, ч.
    & $ \text{Н}_\text{мв} $ & \num{\debugRate}
    \\ \hline

    Цена одного часа машинного времени, \byr{}
    & $ \text{Н}_\text{мв} $ & \num{\machineHourCost}
    \\ \hline

    Норматив расходов на сопровождение и адаптацию ПО, $\%$
    & $ \text{Н}_\text{рса} $ & \num{\supportAndAdaptationRate}
    \\ \hline
\end{longtable}


На стадии технико-экономического обоснования проекта рассчитать точный объём функций невозможно.
Вместо вычисления точного объёма функций применяются приблизительные оценки на основе данных по аналогичным проектам или по нормативам~\cite[с.~61,~приложение 2]{palicyn_2006}, которые приняты в организации.

\def \totalLOC{13940} % TODO
\def \totalLOCCorrected{6300} % TODO

\afterpage{
\begin{longtable}{| >{\centering}m{0.12\textwidth}
                  | >{\raggedright}m{0.40\textwidth}
                  | >{\centering}m{0.18\textwidth}
                  | >{\centering\arraybackslash}m{0.18\textwidth}|}
  \caption{Перечень и объём функций программного модуля}
  \label{table:econ:function_sizes}
  \endfirsthead
  \caption*{Продолжение таблицы \ref{table:econ:function_sizes}}\\
  \endhead

  \hline
    \multirow{2}{0.12\textwidth}[-0.5 em]{\centering \No{} функции}
    & \multirow{2}{0.40\textwidth}[-0.55 em]{\centering Наименование (содержание)}
    & \multicolumn{2}{c|}{\centering Объём функции, LoC} \tabularnewline

  \cline{3-4} &
       & { по каталогу ($ V_{i} $) }
       & { уточненный ($ V_{i}^{\text{у}} $) } \tabularnewline
  \hline

  101 & Организация ввода информации
  & \num{100} & \num{100}
  \\ \hline

  102 & Контроль, предварительная обработка и ввод информации
  & \num{520} & \num{370}
  \\ \hline

  111 & Управление вводом/выводом
  & \num{2700} & \num{1900}
  \\ \hline

  301 & Формирование последовательного файла
  & \num{340} & \num{170}
  \\ \hline

  305 & Обработка файлов
  & \num{750} & \num{570}
  \\ \hline

  506 & Обработка ошибочных и сбойных ситуаций
    & \num{430} & \num{400}
  \\ \hline

  507 & Обеспечение интерфейса между компонентами
  & \num{730} & \num{680}
  \\ \hline

  701 & Математическая статистика и прогнозирование
  & \num{8370} & \num{2210}
  \\ \hline

  Итог & &
 {\num{\totalLOC}} & {\num{\totalLOCCorrected}}
  \\ \hline

\end{longtable}
}

Перечень и объём функций программного модуля перечислен в таблице~\ref{table:econ:function_sizes}.
Общий уточненный объём ПО $ V_{\text{у}} = \SI{\totalLOCCorrected}{\text{LoC}} $.

\subsection{Расчёт трудоемкости разработки программного средства}

% Нормативная трудоёмкость, Тн
\def \normativeWorkload{144}

% Общая трудоёмкость, То
\FPeval{totalWorkload}{
    trunc(\normativeWorkload *
          \complexityFactor *
          \stdModuleUsageFactor *
          \noveltyFactor + 0.5, 0)
}

% Срок разработки проекта, Тр
\def \developmentMonths{3}
\FPeval{developmentYears}{round(\developmentMonths / 12, 2)}

% Численность исполнителей, Чр
\FPeval{requiredProgrammers}{trunc(\totalWorkload / (\developmentYears * \workingDaysInYear) + 1, 0) }

% Считаем, что работы каждому достаётся поровну.
\FPeval{workloadSenior}{trunc( \totalWorkload / \requiredProgrammers, 0 )}
\FPeval{workloadLead}{clip(\totalWorkload - \workloadSenior)}

% ---

На основании общего объема ПО, составляющего $ V_{\text{у}} = \SI{\totalLOCCorrected}{\text{LoC}} $, определяется нормативная трудоемкость (
$ \text{Т}_\text{н}$
) с учетом сложности ПО. Для ПО \complexityGroup-ой группы сложности, к которой относится разрабатываемый программный продукт, нормативная трудоемкость составит~
$ \text{Т}_\text{н} = \SI{\normativeWorkload}{\text{чел.} / \text{дн.}} $

Нормативная трудоемкость служит основой для оценки общей трудоемкости~
$ \text{Т}_\text{о} $
.
Используем формулу (\ref{eq:economics:total_workload}) для оценки общей трудоемкости для небольших проектов:
\begin{equation}
  \label{eq:economics:total_workload}
  \text{Т}_\text{о} = \text{Т}_\text{н} \cdot
                      \text{К}_\text{с} \cdot
                      \text{К}_\text{т} \cdot
                      \text{К}_\text{н} \text{\,,}
\end{equation}
\begin{explanation}
где & $ \text{К}_\text{с} $ & коэффициент, учитывающий сложность ПО; \\
    & $ \text{К}_\text{т} $ & поправочный коэффициент, учитывающий степень использования при разработке стандартных модулей; \\
    & $ \text{К}_\text{н} $ & коэффициент, учитывающий степень новизны ПО.
\end{explanation}

Дополнительные затраты труда на разработку ПО учитываются через коэффициент сложности, который вычисляется по формуле
\begin{equation}
\label{eq:economics:additional_complexity_factor}
  \text{К}_{\text{с}} = 1 + \sum_{i = 1}^n \text{К}_{i} \text{\,,}
\end{equation}
\begin{explanation}
где & $ \text{К}_{i} $ & коэффициент, соответствующий степени повышения сложности ПО за счет конкретной характеристики; \\
    & $ n $ & количество учитываемых характеристик.
\end{explanation}

Наличие двух характеристик сложности позволяет~\cite[c.~66, приложение~4, таблица~П.4.2]{palicyn_2006} вычислить коэффициент сложности
\begin{equation}
\label{eq:economics:complexity_factor}
  \text{К}_{\text{с}} = \num{1} + \num{\additionalComplexityFactor} = \num{\complexityFactor} \text{\,.}
\end{equation}

Разрабатываемое ПО использует стандартные компоненты. Согласно справочным данным~\cite[c.~68,~приложение~4, таблица~П.4.5]{palicyn_2006} коэффициент использования стандартных модулей
$ \text{К}_\text{т} = \num{\stdModuleUsageFactor} $.

Согласно справочным данным~\cite[c.~67, приложение~4, таблица~П.4.4]{palicyn_2006}, коэффициент новизны для разрабатываемого ПО
$ \text{К}_\text{н} = \num{\noveltyFactor} $.

Подставив приведенные выше коэффициенты для разрабатываемого ПО в формулу~(\ref{eq:economics:total_workload}) получим общую трудоемкость разработки
\begin{equation}
  \label{eq:economics:total_workload_calc}
  \text{Т}_\text{о} = \num{\normativeWorkload}
                      \cdot \num{\complexityFactor}
                      \cdot \num{\stdModuleUsageFactor}
                      \cdot \num{\noveltyFactor}
                    \approx \SI{\totalWorkload}{\text{чел.}/\text{дн.}}
\end{equation}

На основе общей трудоемкости и требуемых сроков реализации проекта вычисляется плановое количество исполнителей.
Численность исполнителей проекта рассчитывается по формуле:
\begin{equation}
  \label{eq:economics:num_of_programmers}
  \text{Ч}_\text{р} = \frac{\text{Т}_\text{о}}
                           {\text{Т}_\text{р}
                      \cdot \text{Ф}_\text{эф}} \text{\,,}
\end{equation}
\begin{explanation}
где & $ \text{Т}_\text{о} $ & общая трудоемкость разработки проекта, $ \text{чел.}/\text{дн.} $; \\
    & $ \text{Ф}_\text{эф} $ & эффективный фонд времени работы одного работника в течение года, дн.; \\
    & $ \text{Т}_\text{р} $ & срок разработки проекта, лет.
\end{explanation}

Учитывая срок разработки проекта
$ \text{Т}_\text{р} = \SI{\developmentMonths}{\text{мес.}} = \SI{\developmentYears}{\text{года}} $
, общую трудоемкость и фонд эффективного времени одного работника, вычисленные ранее, можем рассчитать численность исполнителей проекта
\begin{equation}
  \label{eq:econ:num_of_programmers_calc}
  \text{Ч}_\text{р} = \frac{\num{\totalWorkload}}
                           {\num{\developmentYears}
                      \cdot \num{\workingDaysInYear}}
                    \approx \SI{\requiredProgrammers}{\text{чел}}.
\end{equation}

Для выполнения запланированного проекта в указанные сроки необходимо привлечь \requiredProgrammers ~специалистов: инженер-программиста 1 категории и ведущего программист.

\subsection{Расчёт расходов и прогнозируемой цены ПО}

% Тариффные коэффициенты, Тк
\def \tariffFactorSenior{3.25} % == 14 разряд
\def \tariffFactorLead{3.48}   % == 15 разряд

% Часовая тарифная ставка исполнителей, Тч
\FPeval{hourlySalarySenior}{round(\tariffRate * \tariffFactorSenior / \workingHoursInMonth, 2)}
\FPeval{hourlySalaryLead}{round(\tariffRate * \tariffFactorLead / \workingHoursInMonth, 2)}

% Основная зарплата, Зо
\FPeval{baseSalary}{round(\workingHoursInDay * \bonusRate * (\hourlySalarySenior * \workloadSenior + \hourlySalaryLead * \workloadLead), 2)}

% Дополнительная зарплата, Зд
\FPeval{additionalSalary}{round(\baseSalary * \additionalSalaryRate / 100, 2)}

Информация о работниках перечислена в таблице~\ref{table:economics:programmers}.
\begin{table}[ht]
  \caption{Работники, занятые в проекте}
  \label{table:economics:programmers}
  \begin{tabular}{| >{\raggedright}m{0.4\textwidth}
                  | >{\centering}m{0.15\textwidth}
                  | >{\centering}m{0.18\textwidth}
                  | >{\centering\arraybackslash}m{0.15\textwidth}|}
   \hline
   \centering{Исполнители} & Разряд & Тарифный коэффициент & \mbox{Чел./дн.} занятости
   \\ \hline

   Программист \Rmnum{1}-категории & $ \num{14} $ & $ \num{\tariffFactorSenior} $ & $ \num{\workloadSenior} $
   \\ \hline
   Ведущий программист & $ \num{15} $ & $ \num{\tariffFactorLead} $ & $ \num{\workloadLead} $
   \\ \hline
  \end{tabular}
\end{table}

Месячная тарифная ставка одного работника вычисляется по формуле
\begin{equation}
  \label{eq:economics:monthly_salary}
  \text{Т}_\text{ч} =
    \frac {\text{Т}_{\text{м}_{1}} \cdot \text{Т}_{\text{к}} }
          {\text{Ф}_{\text{р}} }  \text{\,,}
\end{equation}
\begin{explanation}
где & $ \text{Т}_{\text{м}_{1}} $ & месячная тарифная ставка 1-го разряда, \byr; \\
    & $ \text{Т}_{\text{к}} $ & тарифный коэффициент, соответствующий установленному тарифному разряду; \\
    & $ \text{Ф}_{\text{р}} $ & среднемесячная норма рабочего времени, час.
\end{explanation}

Подставив данные из таблиц~\ref{table:economics:initial_data} и ~\ref{table:economics:programmers} в формулу~(\ref{eq:economics:monthly_salary}), приняв среднемесячную норму рабочего времени
$ \text{Ф}_{\text{р}} = \SI{\workingHoursInMonth}{\text{часов}} $
получаем
\begin{equation}
  \label{eq:economics:monthly_salary_senior}
  \text{Т}_{\text{ч}}^{\text{прогр. \Rmnum{1}-разр.}} =
      \frac{ \num{\tariffRate} \cdot \num{\tariffFactorSenior} }
           { \num{\workingHoursInMonth} }
    = \SI{\hourlySalarySenior}{\text{\byr}/\text{час;}}
\end{equation}
\begin{equation}
  \label{eq:economics:monthly_salary_lead}
  \text{Т}_{\text{ч}}^{\text{вед. прогр.}} =
      \frac{ \num{\tariffRate} \cdot \num{\tariffFactorLead} }
           { \num{\workingHoursInMonth} }
    = \SI{\hourlySalaryLead}{\text{\byr}/\text{час.}}
\end{equation}

Основная заработная плата исполнителей на конкретное ПО рассчитывается по формуле
\begin{equation}
  \label{eq:economics:base_salary}
  \text{З}_{\text{о}} = \sum^{n}_{i = 1}
                        \text{Т}_{\text{ч}}^{i} \cdot
                        \text{Т}_{\text{ч}} \cdot
                        \text{Ф}_{\text{п}} \cdot
                        \text{К}
                        \text{\,,}
\end{equation}
\begin{explanation}
где & $ \text{Т}_{\text{ч}}^{i} $ & часовая тарифная ставка \mbox{$ i $-го} исполнителя, \byr$/$час; \\
    & $ \text{Т}_{\text{ч}} $ & количество часов работы в день, час; \\
    & $ \text{Ф}_{\text{п}} $ & плановый фонд рабочего времени \mbox{$ i $-го} исполнителя, дн.; \\
    & $ \text{К} $ & коэффициент премирования.
\end{explanation}

Подставив ранее вычисленные значения и данные из таблицы~\ref{table:economics:programmers} в формулу~(\ref{eq:economics:base_salary}) и приняв коэффициент премирования
$ \text{К} = \num{\bonusRate} $
получим
\begin{equation}
  \label{eq:economics:base_salary_calc}
  \text{З}_{\text{о}} = (\hourlySalarySenior \cdot \num{\workloadSenior} +
                         \hourlySalaryLead \cdot \num{\workloadLead})
                        \cdot \num{\workingHoursInDay}
                        \cdot \num{\bonusRate}
                      = \SI{\baseSalary}{\text{\byr}} \text{\,.}
\end{equation}

Дополнительная заработная плата включает выплаты предусмотренные законодательством о труде и определяется по нормативу в процентах от основной заработной платы
\begin{equation}
  \label{eq:economics:additional_salary}
  \text{З}_{\text{д}} =
    \frac {\text{З}_{\text{о}} \cdot \text{Н}_{\text{д}}}
          {100\%} \text{\,,}
\end{equation}
\begin{explanation}
  где & $ \text{Н}_{\text{д}} $ & норматив дополнительной заработной платы, $ \% $.
\end{explanation}

Дополнительная заработная плата разработчиков составит:
\begin{equation}
  \label{eq:econ:additional_salary_calc}
  \text{З}_{\text{д}} =
    \frac{\num{\baseSalary} \cdot 20\%}
         {100\%} = \SI{\additionalSalary}{\text{\byr}} \text{\,.}
\end{equation}

\FPeval{totalSalary}{\baseSalary + \additionalSalary}

% Отчисления в фонд социальной защиты
\FPeval{socialProtectionMoney}{round(\totalSalary * \socialProtectionRate / 100, 2)}

% Материалы и комплектующие
\FPeval{materialsMoney}{round(\baseSalary * \materialsRate / 100, 2)}

% Машинное время
\FPeval{machineTimeMoney}{
    round(\machineHourCost *
          \totalLOCCorrected *
          \debugRate / 100, 2)
}

% Прочие прямые расходы
\FPeval{otherExpenseMoney}{ round( \baseSalary * \otherExpenseRate / 100, 2 )}

% Накладные расходы
\FPeval{overheadExpenseMoney}{ clip(round( \baseSalary * \overheadExpenseRate / 100, 2 ))}

% Общая сумма
\FPeval{\totalExpense}{
    round( \totalSalary +
           \additionalSalary +
           \socialProtectionMoney +
           \materialsMoney +
           \machineTimeMoney +
           \otherExpenseMoney +
           \overheadExpenseMoney, 0 )
}

% Сопровождение и адаптация ПО
\FPeval{supportAndAdaptationMoney} { round( \totalExpense * \supportAndAdaptationRate / 100, 2 )}

% Полная себестоимость ПО
\FPeval{totalSoftwareCost}{ clip(\totalExpense + \supportAndAdaptationMoney) }

% Прогнозируемая прибыль
\FPeval{profitMoney}{ round( \totalSoftwareCost * \profitability / 100, 2) }

% Прогнозируемая цена без налогов
\FPeval{estimatedPrice}{clip(\profitMoney + \totalSoftwareCost)}

% Налог на добавленную стоимость
\FPeval{vatMoney}{round(\estimatedPrice * \vatRate / 100, 2 ) }

%Налог на прибыль
\FPeval{profitTax}{round(\profitMoney * \profitTaxRate / 100, 2)}

% Прогнозируемая отпускная цена
\FPeval{sellingPrice}{ round( \estimatedPrice + \vatMoney, 2 ) }

% ---

Расчеты общей суммы расходов и прогнозируемой цены ПО, а также его себестоимости производятся в соответствии с установленными законодательством и регламентом предприятия нормативами. Результаты расчётов сведены в таблицу~\ref{table:economics:expenses_and_cost}.


\begin{longtable}{| >{\raggedright}m{0.24\textwidth}
                  | >{\centering}m{0.16\textwidth}
                  | >{\centering}m{0.35\textwidth}
                  | >{\centering\arraybackslash}m{0.15\textwidth}|}
  \caption{Расчет себестоимости и отпускной цены ПО}
  \label{table:economics:expenses_and_cost} \\
  \endfirsthead
  \caption*{Продолжение таблицы \ref{table:economics:expenses_and_cost}}\\
  \hline
  \centering{Наименование статей} & Норматив,\% & Методика расчета & Значение, руб. \\ 
  \hline
  \endhead
  \hline
  \centering{Наименование статей} & Норматив,\% & Методика расчета & Значение, руб. \\ 
  \hline


    Отчисления в фонд социальной защиты населения и обязательного страхования
    & $ \text{Н}_{\text{сз}} = \num{\socialProtectionRate} $
    & $ \text{З}_{\text{сз}} = (\text{З}_{\text{о}} + \text{З}_{\text{д}}) \cdot \text{Н}_{\text{сз}} / {\num{100}} $
    & \num{\socialProtectionMoney}
    \\ \hline

    Материалы и комплектующие
    & $ \text{Н}_{\text{мз}} = \num{\materialsRate} $
    & $\text{М} = { \text{З}_{\text{о}} \cdot \text{Н}_{\text{мз}} } / { \num{100} } $
    & \num{\materialsMoney}
    \\ 

    Машинное время
    &
    & $ \text{Р}_{\text{м}} = \text{Ц}_{\text{м}} \cdot \text{V}_{\text{о}} / \num{100} \cdot \text{Н}_{\text{мв}} $
    $ \text{Н}_{\text{мв}} = \num{\debugRate}{\text{ машино-часов}} $
    $ \text{Ц}_{\text{м}} = \SI{\machineHourCost}{\text{\byr}} $
    $ V_{\text{у}} = \SI{\totalLOCCorrected}{\text{LoC}} $
    & \num{\machineTimeMoney}
    \\ \hline

    Прочие прямые расходы
    & $ \text{Н}_{\text{пз}} = \num{\otherExpenseRate} $
    & $  \text{П}_{\text{з}} = { \text{З}_{\text{о}} \cdot \text{Н}_{\text{пз}} } / \num{100} $
    & \num{\otherExpenseMoney}
    \\ \hline

    Накладные расходы
    & $ \text{Н}_{\text{рн}} = \num{\overheadExpenseRate} $
    & $  \text{Р}_{\text{н}} = { \text{З}_{\text{о}} \cdot \text{Н}_{\text{рн}} } / \num{100} $
    & \num{\overheadExpenseMoney}
    \\ \hline

    Общая сумма расходов по смете
    &
    & $ \text{С}_{\text{р}} = \text{З}_{\text{о}} + \text{З}_{\text{д}} + \text{З}_{\text{сз}} + \text{М} + \text{Р}_{\text{м}} + \text{Р}_{\text{к}} + \text{П}_{\text{з}} + \text{Р}_{\text{н}} $
    & \num{\totalSoftwareCost}\\
    \hline

    Сопровождение и адаптация ПО
    & $ \text{Н}_{\text{рса}} = \num{\supportAndAdaptationRate} $
    & $  \text{Р}_{\text{са}} = {\text{С}_{\text{р}} \cdot \text{Н}_{\text{рса}} } / { \num{100} } $
    & \num{\supportAndAdaptationMoney}
    \\ \hline

    Полная себестоимость ПО
    &
    & $ \text{С}_{\text{п}} = \text{С}_{\text{р}} + \text{Р}_{\text{са}} $
    & \num{\totalSoftwareCost}
    \\ \hline

    Прогнозируемая прибыль

    & $ \text{У}_{\text{рп}} = \num{\profitability} $
    & $  \text{П}_{\text{с}} = { \text{С}_{\text{п}} \cdot \text{У}_{\text{рп}} } / \num{100} $
    & \num{\profitMoney}
    \\ \hline

    Прогнозируемая цена без налогов
    &
    & $ \text{Ц}_{\text{п}} = \text{С}_{\text{п}} + \text{П}_{\text{с}}$
    & \num{\estimatedPrice}
    \\ \hline

    Налог на добавленную стоимость
    & $ \text{Н}_{\text{дс}} = \num{\vatRate} $
    & $ \text{НДС}_{\text{}} = { (\text{Ц}_{\text{п}} + \text{О}_{\text{мр}}) \cdot \text{Н}_{\text{дс}} } / \num{100} $
    & \num{\vatMoney}
    \\ \hline

    Прогнозируемая отпускная цена
    &
    & $ \text{Ц}_{\text{о}} = \text{Ц}_{\text{п}} + \text{О}_{\text{мр}} + \text{НДС} $
    & \num{\sellingPrice}
    \\ \hline
\end{longtable}

По результатам расчетов получаем прогнозируемую отпускную цену программного средства $ \text{Ц}_{\text{о}} = \num{\sellingPrice}~\text{\byr}$ и прогнозируемую прибыль $ \text{П}_{\text{с}} = \num{\profitMoney}~\text{\byr}$.

Программное средство разрабатывалось для одного заказчика в связи с этим экономическим эффектом разработчика будет являться чистая прибыль от реализации.

Чистую прибыль от реализации проекта можно рассчитать по формуле
\begin{equation}
  \label{eq:economics:profit}
  \text{П}_\text{ч} =
    \text{П}_\text{c} \cdot
    \left(1 - \frac{ \text{Н}_\text{п} }{ \num{100\%} } \right) \text{\,,}
\end{equation}
\begin{explanation}
  где & $ \text{Н}_{\text{п}} $ & ставка налога на прибыль,~$\%$.
\end{explanation}

\FPeval{netProfit}{round(\profitMoney - \profitTax, 0)}

Подставив данные в формулу~(\ref{eq:economics:profit}) получаем чистую прибыль
\begin{equation}
  \label{eq:economics:net_profit}
  \text{П}_\text{ч} =
    \num{\profitMoney} \cdot \left( 1 - \frac{\num{\profitTaxRate\%}}{100\%} \right) = \SI{\netProfit}{\text{\byr}} \text{\,.}
\end{equation}

На реализацию программного средства будет затрачено $ \totalSoftwareCost~\text{\byr}$ а экономический эффект от реализации, в виде чистой прибыли, равен $ \profitMoney~\text{\byr}$. 

\def \tariffFactorJuniorGraphologyAnalytic{2.48}

\subsection{Расчёт экономической эффективности у заказчика}
Эффект от применения программного продукта заказчиком заключается в сокращении расходов на оплату труда экспертов"=графологов, так же с помощью разработанного программного средства можно сократить время на выполнение работ за счет уменьшение времени на выполнение рутинных процедур, например, предварительную обработку образцов почерка, что приведет к ускорению графологического анализа. Так как оценить экономический эффект от сокращения времени на проведения анализа технически сложно, далее будет рассматриваться и рассчитываться только эффект от сокращения расходов на оплату труда. В настоящий момент штат графологов компании составляет 6 человек. Их них 1 старший эксперт-графолог, 2 эксперта графолога и 3 младших эксперта-графолога. Подготовки и предварительного анализа образцов являются обязанностями младших экспертов-графологов.

Согласно отзывам заказчика после внедрения программного средства время подготовки и предварительного анализа образцов почерка сократилось с 30 до 10 минут.

Приняв количество обрабатываемых в месяц образцов за 1000 и время на обработку одного образца за 30 минут и подставив в формулу~(\ref{eq:economics:total_workload}) получим трудозатраты на подготовку и предварительный анализа образцов почерка до внедрения ПС.
\begin{equation}
  \label{eq:economics:total_workload_calc}
  \text{Т}_\text{о} = \frac{1000}{0.5 \cdot 160} \approx 3 {\text{чел.}/\text{мес.}}
\end{equation}
Рассчитанные данные совпадают с количеством младших экспертов"=графологов в штате.

Приняв количество обрабатываемых в месяц образцов за 1000 и время на обработку одного образца за 10 минут и подставив в формулу~(\ref{eq:economics:total_workload}) получим трудозатраты на подготовку и предварительный анализа образцов почерка после внедрения ПС.
\begin{equation}
  \label{eq:economics:total_workload_calc}
  \text{Т}_\text{о} = \frac{1000}{0.33 \cdot 160} \approx 1 {\text{чел.}/\text{мес.}}
\end{equation}

Благодаря переноса функции подготовки образцов почерка и ускорению предварительного анализа образцов почерка после внедрения разработанного программного средства могут быть высвобождены ресурсы в виде 2-х младших экспертов"=графологов.

% Часовая тарифная ставка исполнителей, Тч
\FPeval{hourlySalaryAnalytic}{round(\tariffRate * \tariffFactorJuniorGraphologyAnalytic / \workingHoursInMonth, 2)}

% Основная зарплата, Зо за год
\FPeval{baseSalaryAnalytic}{round(\workingHoursInDay * \bonusRate * (\hourlySalaryAnalytic * \workingDaysInYear * 2), 2)}

% Дополнительная зарплата, Зд
\FPeval{additionalSalaryAnalytic}{round(\baseSalaryAnalytic * \additionalSalaryRate / 100, 2)}

\begin{table}[ht]
  \caption{Работники, занятые на стороне заказчика}
  \label{table:economics:graph_analytic}
  \begin{tabular}{| >{\raggedright}m{0.4\textwidth}
                  | >{\centering}m{0.15\textwidth}
                  | >{\centering}m{0.18\textwidth}
                  | >{\centering\arraybackslash}m{0.15\textwidth}|}
   \hline
   \centering{Исполнители} & Разряд & Тарифный коэффициент & \mbox{Чел./дн.} занятости
   \\ \hline

   Младший эксперт"=графолог & $ \num{12} $ & $ \num{\tariffFactorJuniorGraphologyAnalytic} $ & $ \num{\workingDaysInYear} $
   \\ \hline
  \end{tabular}
\end{table}

Подставив данные из таблицы~\ref{table:economics:graph_analytic} в формулу~(\ref{eq:economics:monthly_salary}), приняв значение тарифной ставки 1-го разряда
$ \text{Т}_{\text{м}_{1}} = \SI{\tariffRate}{\text{\byr}} $
и среднемесячную норму рабочего времени
$ \text{Ф}_{\text{р}} = \SI{\workingHoursInMonth}{\text{часов}} $
получаем
\begin{equation}
  \label{eq:economics:monthly_salary_analytic}
  \text{Т}_{\text{ч}}^{\text{мл. эксп.-граф.}} =
      \frac{ \num{\tariffRate} \cdot \num{\tariffFactorJuniorGraphologyAnalytic} }
           { \num{\workingHoursInMonth} }
    = \SI{\hourlySalaryAnalytic}{\text{\byr}/\text{час;}}
\end{equation}

Подставив ранее вычисленные значения и данные из таблицы~\ref{table:economics:graph_analytic} в формулу~(\ref{eq:economics:base_salary}) и приняв коэффициент премирования
$ \text{К} = \num{\bonusRate} $
получим
\begin{equation}
  \label{eq:economics:base_analytic_salary_calc}
  \text{З}_{\text{о}} = \hourlySalaryAnalytic \cdot \num{\workingDaysInYear} \cdot
                        \num{\workingHoursInDay}
                        \cdot \num{\bonusRate}
                        \cdot 2
                      = \SI{\baseSalaryAnalytic}{\text{\byr}} \text{\,.}
\end{equation}

Приняв норматив дополнительной заработной платы
$ \text{Н}_{\text{д}} = \num{\additionalSalaryRate\%} $
и подставив известные данные в формулу~(\ref{eq:economics:additional_salary}) получим
\begin{equation}
  \label{eq:econ:additional_salary_analytic_calc}
  \text{З}_{\text{д}} =
    \frac{\num{\baseSalaryAnalytic} \cdot 20\%}
         {100\%} = \SI{\additionalSalaryAnalytic}{\text{\byr}} \text{\,.}
\end{equation}

\FPeval{totalSalaryAnalytic}{\baseSalaryAnalytic + \additionalSalary}

% Отчисления в фонд социальной защиты
\FPeval{socialProtectionMoneyAnalytic}{round(\totalSalaryAnalytic * \socialProtectionRate / 100, 2)}

% Общая сумма
\FPeval{totalDirtyAnalyticExpense}{round(\totalSalaryAnalytic + \socialProtectionMoneyAnalytic, 2)}

\FPeval{totalAnalyticExpense}{round(\totalDirtyAnalyticExpense * (1 - 18/100), 2)}

Учитывая отчисления в фонд социального страхования населения и обязательного страхования получаем итоговые расходы на оплату труда.
\begin{equation}
  \label{eq:econ:total_salary_analytic_calc}
    \text{Р}_{\text{от}}=\num{\baseSalaryAnalytic} + \num{\additionalSalaryAnalytic} + \num{\socialProtectionMoneyAnalytic} = \num{\totalDirtyAnalyticExpense} {\text{\byr}} \text{\,.}
\end{equation}

Экономическия на оплату труда составит $ \text{Р}_{\text{от}}=\num{\totalDirtyAnalyticExpense} {\text{\byr}} $.

Подставив данные в формулу~(\ref{eq:economics:profit}) получим чистую прибыль равную $ \totalAnalyticExpense~\text{\byr} $, что для пользователя будет выступать в качестве экономического эффекта.

\def \depositRate{0.063}
\FPeval{discontOnSecondYear}  {round(1 / pow(2, 1 + \depositRate), 3)}
\FPeval{discontOnThirdYear}   {round(1 / pow(3, 1 + \depositRate), 3)}
\FPeval{discontOnFourthYear}  {round(1 / pow(4, 1 + \depositRate), 3)}
\FPeval{discontOnFivethYear}  {round(1 / pow(5, 1 + \depositRate), 3)}

\FPeval{secondYearProfit}{round(\totalAnalyticExpense * \discontOnSecondYear, 2)}
\FPeval{thirdYearProfit}{round(\totalAnalyticExpense * \discontOnThirdYear, 2)}
\FPeval{fourthYearProfit}{round(\totalAnalyticExpense * \discontOnFourthYear, 2)}
\FPeval{fivethYearProfit}{round(\totalAnalyticExpense * \discontOnFivethYear, 2)}

\FPeval{firstYearProfit}{round(\totalAnalyticExpense - \sellingPrice, 2)}
\FPeval{secondYearTotalProfit}{round(\firstYearProfit + \totalAnalyticExpense * \discontOnSecondYear, 2)}
\FPeval{thirdYearTotalProfit}{round(\secondYearTotalProfit + \totalAnalyticExpense * \discontOnThirdYear, 2)}
\FPeval{fourthYearTotalProfit}{round(\thirdYearTotalProfit + \totalAnalyticExpense * \discontOnFourthYear, 2)}
\FPeval{fivethYearTotalProfit}{round(\fourthYearTotalProfit + \totalAnalyticExpense * \discontOnFivethYear, 2)}

\begin{longtable}{| >{\raggedright}m{0.25\textwidth}
                  | >{\centering}m{0.12\textwidth}
                  | >{\centering}m{0.12\textwidth}
                  | >{\centering}m{0.12\textwidth}
                  | >{\centering}m{0.12\textwidth}
                  | >{\centering\arraybackslash}m{0.12\textwidth}|}
  \caption{Экономический эффект у покупателя}
  \label{table:econ:customer_profit}
  \endfirsthead
  \caption*{Продолжение таблицы \ref{table:econ:customer_profit}}\\
  \endhead

  \hline
    \multirow{2}{0.20\textwidth}{\centering{Показатель}}
    & \multicolumn{5}{c|}{\centering Расчетный период} \tabularnewline

  \cline{2-6}
       & { 1 }
       & { 2 }
       & { 3 } 
       & { 4 }
       & { 5 }\tabularnewline
  \hline
  РЕЗУЛЬТАТ & & & & & \\ \hline
  1. Экономический эффект & \num{\totalAnalyticExpense} & \num{\totalAnalyticExpense} & \num{\totalAnalyticExpense} & \num{\totalAnalyticExpense} & \num{\totalAnalyticExpense} \\ \hline
  2. Дисконтированный результат & \num{\totalAnalyticExpense} & \num{\secondYearProfit} & \num{\thirdYearProfit} & \num{\fourthYearProfit} & \num{\fivethYearProfit} \\ \hline
  ЗАТРАТЫ & & & & & \\ \hline
  4. Дисконтированные инвестиции & \num{\sellingPrice} & & & & \\ \hline
  5. Чистый дисконтированный доход & \num{\firstYearProfit} & \num{\secondYearProfit} & \num{\thirdYearProfit} & \num{\fourthYearProfit} & \num{\fivethYearProfit} \\ \hline
  6. Чистый дисконтированный доход нарастающим итогом & \num{\firstYearProfit} & \num{\secondYearTotalProfit} & \num{\thirdYearTotalProfit} & \num{\fourthYearTotalProfit} & \num{\fivethYearTotalProfit} \\ \hline
  Коэффициент дисконтирования & \num{1} & \num{\discontOnSecondYear} & \num{\discontOnThirdYear} & \num{\discontOnFourthYear} & \num{\discontOnFivethYear} \\ \hline
\end{longtable}

На четвертый года использования разработанное программное средство окупит себя и за 5 лет использования принесет прибыли в $ \fivethYearTotalProfit \text{ \byr} $.