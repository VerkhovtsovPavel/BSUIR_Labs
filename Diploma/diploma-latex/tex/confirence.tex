\section{Методы и подходы автоматизации графологического анализа}
\label{conf:voronesh:intro}

В данном разделе будет произведён обзор программных средств аналогичных разрабатываемому в рамках дипломного проекта – определение психологических параметров человека по образцу почерка, а так же литературных источников. Проанализированы преимущества и недостатки различных подходов к выделению и классификации признаков рукописного текста.

\subsection{Графология}
\label{conf:voronesh:grafologic}
\emph{Графология} – это учение, постулирующее наличие устойчивой связи месту почерком и индивидуальными особенностями личности.

Идея использования почерка для выявления психологических параметров личности впервые была предложена в 1622 в книге итальянского профессора Камилло Бальдо <<Как узнать природу и качества человека, взглянув на букву, которую он написал>>~\cite{kamillo_grafology}. Первым кто систематизировал знания стал Фландрэна аббат Мишон в 1872 году. Он проанализировал большое количество работ по графологии и образцов почерка и в своей книге <<Система графологии>> предложил \emph{метод Мишона}, он основывался на анализе штрихов, букв, слов, свободных движений, строк и пр.~\cite{mishon_grafology}

Начиная с середины 20-го века графология начала рассматриваться как псевдонаучное учение~\cite{graphology_wiki}. По результатам исследования профессиональным графологам не удалось достоверно оценить трудовые способности человека. В среднем профессиональные графологи давали такую же по степени достоверности оценку, как и люди «с улицы»~\cite{neter_shakhar_psevdograph, king_koehler_psevdograph}. В десятках исследований было показано отсутствие связи особенностей почерка с трудовыми способностями человека.

Тем не менее графология широко используется в современной практике отбора кадров~\cite{graphology_psyfactor}.

Основные признаки почерка, которые анализирует графологическая экспертиза:
\begin{enumerate}
  \item размер букв (очень маленькие, маленькие, средние, крупные);
  \item наклон букв (левый наклон, легкий наклон влево, правый наклон, резкий наклон вправо);
  \item направление почерка: (строчки ползут вверх, строчки прямые,  строчки ползут вниз);
  \item размашистость и сила нажима: (легкая, средняя, сильная, очень сильная);
  \item характер написания слов (склонность к соединению букв и слов, склонность к отдалению букв друг от друга, смешанный стиль);
  \item общая оценка (почерк старательный, почерк неровный, почерк небрежный, почерк неразборчивый).
\end{enumerate}

Перечисленные параметры почерка являются устойчивыми, но все же присутствует естественные отклонения параметров (длина, ширина, толщина, угол) от средних значений. Вариация становится наиболее заметной при изменение психологического состояния человека, например при страхе, беспокойстве, алкогольном опьянении.

\subsection{Существующие подходы}
\label{conf:voronesh:literary_sources}

Публикации и научные статьи на темы схожие с темой данной работы можно условно разделить по следующим признакам:
\begin{itemize}
  \item метод сегментации изображения;
  \item метод классификации признаков изображения.
\end{itemize}

Метод сегментации изображения является неотъемлемой частью алгоритмов обработки рукописного текста, будь это распознавание или анализ. От качества сегментации напрямую зависит качество работы всего алгоритма поэтому выбор метода сегментации является важным этапом.

В рассмотренной литературе предлагаются следующие алгоритмы сегментации рукописного текста:
\begin{itemize}
  \item преобразования Хафа~\cite{louloudis_gatos_pratikakis_halatsis};
  \item нечеткие интервалы~\cite{louloudis_gatos_pratikakis_halatsis};
  \item нечеткий и адаптивный рекурсивный метод наименьших \mbox{квадратов~\cite{louloudis_gatos_pratikakis_halatsis};}
  \item статистические методы (средний интервал между строками)~\cite{gomathi_umadevi_mohanavel};
  \item метод Лоулодиса-Гатоса-Пратикакиса-Халатсиса (LGPH)~\cite{louloudis_gatos_pratikakis_halatsis};
  \item проецирование контуров~\cite{louloudis_gatos_pratikakis_halatsis}.
\end{itemize}

Преобразования Хафа являются мощным инструментом компьютерного зрения, позволяющим извлекать элементы из изображения. Данный метод позволяет достичь высокой точности сегментации текста, однако требует больших вычислительных затрат в связи с объемом вычислений и использованием тригонометрических функций при вычислении.

Метод рекурсивных нечетких и адаптивных наименьших квадратов, так же как и преобразования Хафа, широко используется для выделения областей изображения благодаря высокому качеству работы и устойчивости к шумам и искажениям, однако вычислительная стоимость данных методов относительно велика.

Метод Лоулодиса-Гатоса-Пратикакиса-Халатсиса показывает очень хорошие результаты сегментации и по результатам исследований превосходит по качеству и времени работы все приведенные выше алгоритмы, однако он пока находится в состоянии исследования и не имеет реальных примеров реализации и использования в практических проектах.

Методы проецирования контуров и нечетких интервалов хорошо подходят для распознавания печатного текста, но дают плохие результаты при распознавании рукописного из-за динамически изменяющихся интервалов между символами,строками и словами.

Статические методы основаны на разбиении изображении в зависимости от распределения средней яркости частей изображения, качества работы данных методов ниже чем у преобразований Хафа или рекурсивных методов на основе наименьших квадратов, но время работы намного меньше благодаря меньшему количеству вычислений и быстрым операциям сложению и делению.

В разрабатываемом программном средстве не требуется высокая точность распознавания, как например для распознавания текста, в то время как программное средство может работать с большим количеством изображений одновременно и быстродействие важно. Исходя из этого в качестве алгоритма сегментации будет использоваться статический метод.

Не менее важным является алгоритм классификации, так как именно он будет устанавливать соответствие между параметрами текста и психологическими характеристиками, например силой нажима и степенью наклона символа.

В рассмотренной литературе предлагаются следующие алгоритмы классификации признаков рукописного текста:
\begin{itemize}
  \item нейронные сети~\cite{champa_ananda_kumar_ann, grewal_prashar, gabrani_solomon_dviwe,puri_lakhwani, dang_kumar, kathait_singh};
  \item мешок особенностей~\cite{rothacker_bag_of_features};
  \item метод опорных векторов~\cite{slideshare_khandelwal_garg, gabrani_solomon_dviwe, prasad_singh_sapre};
  \item метод основанный на правилах~\cite{champa_ananda_kumar_rule_base}.
\end{itemize}

Метод основанный на правилах заключается в последовательной проверке соответствия параметров почерка набору правил <<если"=то>>. Как пример можно привести правило <<Если строки наклонены влево и сила нажима слабая, то человек пессимист не склонный к выражению эмоций>>. Данный подход основан только на описании графологических метод, обладает высокой скоростью и не требует обучения. Однако требуется определение границ значений анализируемых параметров, а так же количество правил экспоненциально растет с количеством параметров и их возможных значений, так для двух параметров с тремя значениями каждого понадобиться 9 правил~\cite{champa_ananda_kumar_rule_base}, а для пяти параметров уже 243.

Метод основанный на <<мешке особенностей>> состоит в составлении <<словаря слов>> на основе большой базы изображений. Данный словарь будет содержать фрагмент изображения описанный каким-либо дескриптором, например SIFT, и частоту появления этого фрагмента. По сути данный метод рассматривает задачу определения психологических параметров как задачу категоризации. Основными недостатками данного метода является необходимость в сборе и ручной обработке, определения параметров личности экспертами графологами, огромной базы изображений, т.к. все образцы почерка достаточно похожи и сложно выделить отличительные признаки.

Использование нейронных сетей является хорошим решением благодаря способности сети обобщать данные, а использование обратного распространения ошибок позволяет добиться очень хорошего качества распознавания. Однако выбор оптимальной структуры сети и функции активации нейронов является нетривиальной задачей, так же время обучения и распознавания относительно велико.

Метод опорных векторов основан на нахождении границы классов максимально удаленной от их экземпляров. К его достоинствам относиться хорошее качество распознавания, высокая скорость обучения и классификации. Однако необходим большой объем обучающей выборки, такой чтобы каждый из возможных наборов признаков встречался хотя бы раз. Так же вопрос выбора оптимального типа ядер схож с выбором функции активации для нейронных сетей.

\subsection{Вывод}
\label{sec:domain:outro}
За рамками рассматриваемой темы осталось множество альтернативных параметров почерка и методов классификации, включая комбинацию методов, а также смежных областей применения, таких как установления авторства рукописного текста. Вышеперечисленное может стать основой для будущих научных исследований.

